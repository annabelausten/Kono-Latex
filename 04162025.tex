 \documentclass{assets/fieldnotes}

\title{Kono (Sierra Leone)}
\author{LING3020/5020}
\date{University of Pennsylvania, Spring 2025\\04/16/2025 Class Projects Week 4}

\setcounter{secnumdepth}{4} %enable \paragraph -- for subsubsubsections

\begin{document}

\maketitle

\maketitle
\tableofcontents

\section{Lex}
\exg. aŋ́ wù má\\
3-Pl short me.?\\
`they are shorter than me' 


\exg. wíí wù-ní-àmbè-m-bòò\\
deer short-?-AMBE-1SG-hand\\
`I have a short deer' 

\exg. wíí   wù-n-àmbè-m-bòò\\
      deer  short-?-AMBE-1SG-hand\\
`I have a short deer'

\exg. a-ŋ́ dʒímàtɛ̀ mù-mà\\
3-PL intelligent ?-me.?\\
`they are smarter (more intelligent) than me' 

\exg. a-ŋ́ ímàtɛ̀  mù-mà\\
3-PL intelligent ?-me.?\\
`they are smarter (more intelligent) than me' 

\exg. wíí-jímàtɛ̀ n-àmbè-m-bòò\\
deer-intelligent ?-AMBE-1SG-hand\\
`I have a smart deer'

\exg. a-ŋ́ t͡ʃèná-mà\\
3-PL old-me.?\\
`they are older than me'

\exg. a-ŋ́ t͡ʃèná-í-mà\\
3-PL old-2SG-?\\
`they are older than me'

\exg. a-ŋ́ t͡ʃèná-n-mà\\
3-PL old-1SG-?\\
`they are older than me'

\jf{Not sure if I need to add something to these compartive staments that marks the me as signifying "older than me"}
\jf{After re-listening to recordings realized that Tony adds in subject markers in compartive statements for They are older than you or they are older than me (he says taller in the recording), so the 'ma' probably has a different meaning. However, these subject markers are not present in all of the data.}

\exg. wíí t͡ʃènàmá bè-m-bòò\\
deer old BE-1SG-hand\\
`I have an old deer'

\exg. wíí còkwé-bè-m-bòò\\
deer old-BE-1SG-hand\\
`I have an old deer'


\exg. a-ŋ-wòò-mò-má\\
3PL-crazy-?-me\\
`they are crazier than me'

\exg. a-ŋ-wòò-mò-í-má\\
3PL-crazy-?-2SG-me??\\
`they are crazier than you'

\jf{I am highly skeptical that má refers to me here}

\exg. wíí-wòò-ná bè-m-bòò\\
deer-crazy-? BE-1SG-hand\\
`I have a crazy deer'

\exg. n-à wíínɛ̀ wòò-ɱfamu\\
1SG.POSS deer crazy-have??\\
`I have a crazy deer'

\exg. á-n t͡ʃínáín-mà\\
3-PL
they are taller than you 

\exg. wíí jã́sàmá bè-m-bòò\\
deer tall BE-1SG-hand\\
`I have a tall deer'

\exg. n-à-wíínɛ̀ jã́sã̀\\
1SG.POSS-deer tall\\
`My deer is tall'

\exg. a-ŋ́ bwò bwémù à bɛ̀má\\
3-PL body healthy POSS ?\\
`they are healthier than you'

\exg. n-à-wíínɛ̀ à bwò-bwémù\\
1SG.POSS-deer POSS body-healthy\\
`I have a healthy deer'

\exg. wíin bwò bwéná bè-m-bòò\\
deer body healthy BE-1SG-hand\\
`I have a healthy deer'


\ex. fíánɛ̀\\
`broom'

\ex. káínɛ̀\\
`gold'

\exg. n-à fíàn-ɔ káí\\
1SG.PST broom-3SG.OBJ break\\
`I broke the broom'

\exg. n-à fíànɛ̀ n-dò káí\\
1SG.PST broom   3Pl-OBJ break\\
`I broke the brooms'

\exg. n-à  fíànɛ̀ t͡ʃɛ̀-n-dò káí\\
1SG.PST broom DEM-3Pl-OBJ break\\
`I broke these brooms'

\exg. n-à sánù  fíànɛ̀ n-dò káí \\
1SG.PST gold broom 3Pl.OBJ break\\
`I broke gold brooms'


\section{Alex}

\exg.
m     bɛ     n-a       sani-n      dɔ    toŋ   \\
1SG   also   1SG-PST   bottle-PL   OBJ   close \\%
`I also closed the bottles.' \label{I also closed the bottles1}

\exg.
m     bɛ     n-a       sani-n      a     toŋ   \\
1SG   also   1SG-PST   bottle-PL   OBJ   close \\%
`I also closed the bottles.' \label{I also closed the bottles2}

\alex{It seems that either /a/, \ref{I also closed the bottles1}, or /ɔ/, \ref{I also closed the bottles2}, can occur as the object marker form with `close', }

\jal{Lex, this is a nice minimal pair for you}

\alex{I am suspecting that in many of these examples with `also', we have a kind of hanging topic that amounts to what Tony parsed as `even me'. So, these might be something like `Even me, I closed the bottles'.}

\jal{it's possible and would make the most sense of the patterns, but they don't have the rising intonation we expect, eh? maybe another A'-position? do we have `also' on an object?}

\exg.
sani-m      bɛ     na        tond-a  \\
bottle-PL   also   3PL.AUX   close-A \\%
`The bottles also closed.' \label{The bottles also closed}\\
`The bottles, too, they closed.'

\alex{I suspect the `hanging topic' idea might hold for the unaccusative sentences with `also' in them, too, as segmented in \ref{The bottles also closed}, and indicated in the second translation line.}

\exg.
sani     wɛ     a     tond-a  \\
bottle   also   OBJ   close-A \\%
`The bottle also closed.' \label{The bottle also closed}

\jal{Lex, another pair for you to look at}

\alex{\ref{The bottle also closed} suggests that there is not a final nasal segment in the `also' morpheme. Otherwise we might expect a nasal preceding the following /a/ vowel (whatever that is, either object marker or auxiliary...).}

\jal{agreed!}

\ex.
\ag.
n-a        tond-a  \\
1SG.AUX   close-A \\%
`I closed.'
\bg.
mbɛ        (a)   n-a          tond-a  \\
Also   (?)   1SG.AUX   close-A \\%
`I also closed.' \label{I also closed}

\alex{When Tony segmented \ref{I also closed}, he did not include the vowel /a/ after \textit{mbɛ}. Instead, he produced something like \textit{mbɛ na tonda}. In fact, when I repeated back to him \textit{mbɛ \textbf{a} na tonda}, he rejected it and insisted on \textit{mbɛ na tonda}. Not sure what to make of this yet...}

\jal{I didn't hear it when he pronounced it naturally either; I've got it annotated as (*a)}

\exg.
a-m    bɛ     an-a      tond-a  \\
3-PL   also   3-PL-AUX   close-A \\%
`They also closed.' \label{They also closed}

\exg.
sani-m       bɛ     ndɔ       te-a    \\
botttle-PL   also   3PL.OBJ   break-A \\%
`The bottles, too, they broke.'

\exg.
sani      wɛ     ɔ         te-a    \\
botttle   also   3SG.OBJ   break-A \\%
`The bottle, too, it broke.'

\exg.
*sani     ɔ         wɛ     te-a    \\
botttle   3SG.OBJ   also   break-A \\%
Int. `The bottle, too, it broke.'

\ex.
\ag.
a-n-dɔ      te-a    \\
3-PL-OBJ   break-A \\%
`They broke.' \label{They broke}
\bg.
a-m    bɛ     ɔ     te-a    \\
3-PL   also   OBJ   break-A \\%
`They also broke.' \label{They also broke}

\alex{Perhaps it is not always the case that we have a `hanging topic'-like strategy for the `also' constructions. If we did, then I would have predicted that \ref{They also broke} would instead look something like in \ref{They are also breaking} (and thus pattern more similarly to what we see in \ref{They also closed}). However, \ref{They also broke} suggests that we might indeed have a case of `also' intervening between the surface subject and tense/auxiliary/object marker.}

\exg.
a-m    bɛ     ndɔɔ           te-a    \\
3-PL   also   3PL.OBJ.AUX?   break-A \\%
`They are (also?) breaking.' \label{They are also breaking}

\alex{As for \ref{They are also breaking}, Tony said that while it is a well-formed sentence and it means ``the action is going on''---it does not have the past/completed meaning `They also broke' in \ref{They also broke}. Cf. the non-past/future construction in \ref{They will break} without the `also' (repeated from 04092025.tex).}
\jal{I wasn't clear this was a long vowel vs just him emphasizing the ndo b/c it was the part he was focusing on}

\exg.
a-n-dɔɔ         te-a      waN \\
3-PL-OBJ   break-A   FOC \\%
`They will break.' \label{They will break}

\exg.
m-bé       sani-n      a     tond-a    waN \\
1SG.NPST   bottle-PL   OBJ   close-A   FOC \\%
`I will close the bottles.'

\exg.
m-bɛ́     mbé       sani-n      a     tond-a    waN \\
1SG.NPST   also   bottle-PL   OBJ   close-A   FOC \\%
`I, too, will close the bottles.'

\exg.
m     bɛ́     m-b-ánà            tond-a    waN \\
  also   1SG.NPST.3PL.OBJ   close-A   FOC \\%
`I, too, will close them.'

\exg.
m     bɛ́     nɛ         ɛ     tond-a    waN   sina     \\
1SG   also   1SG.OBJ?   AUX   close-A   FOC   tomorrow \\%
`I, too, will close tomorrow.'

\exg.
a-m     bɛ́     anɛ    a   tond-a    waN \\
3-PL   also   3PL.AUX ?  close-A   FOC \\%
`They, too, will close.'


\exg.
e         an    dɔ    te-a      waN   sina     \\
3SG.NPST   3-PL   OBJ   break-A   FOC   tomorrow \\%
`He/she will break them tomorrow.'

\exg.
m     bɛ́     mbɔɔ              te-a      waN   sina     \\
1SG   also   1SG.NPST.3SG.OBJ   break-A   FOC   tomorrow \\%
`I, too, will break it tomorrow.'

\jal{not sure long?}

\exg.
m     bɛ́     mba       a-n    dɔ    te-a      waN   sina     \\
1SG   also   1SG.AUX   3-PL   OBJ   break-A   FOC   tomorrow \\%
`I, too, will break them tomorrow.'


\section{Giang}

\jal{let's get the currently most plausible segmentation here}
\g{o-series verbs: ``asked" and ``broke" \& object paradigm}

\exg. Bondu à Sàà ò tísá kùnù.\\
Bondu 3SG.PST Saa o ask yesterday.\\
``Bondu asked Saa yesterday''

\exg. Bondu a ɲondo tisa kunu\\
Bondu 3SG.PST who-o ask yesterday\\
``Who did Bondu ask yesterday?''

\exg. *Bondu a-ɱ fando tisa wã́ kunu\\
Bondu 1\textsc{AUX.PST} f-ando ask \textsc{foc} yesterday\\
Bondu asked me yesterday. 

\exg. Bondu a-ɱ fando tisa kunu\\
Bondu 1\textsc{3SG.PST} 1SG f-ando ask yesterday\\
Bondu asked me yesterday. 

\jal{eg perhaps this? LEX - notice the strong forms of the pronouns are iwã but ɱfã }
\exg. Bondu a ɱ-fan-do tisa kunu\\
Bondu \textsc{3SG.PST} 1\textsc{SG}-\textsc{foc}-\textsc{obj} ask yesterday\\
Bondu asked me yesterday.

\exg. Bondu a-m fáándo tisa kunu\\
Bondu 1\textsc{3SG.PST-m} f-ando ask yesterday\\
Bondu asked us yesterday.

\exg. Bondu a (j)i wando tisa kunu\\
Bondu \textsc{3SG.PST} 2\textsc{SG} w-ando ask yesterday\\
Bondu asked you(sg) yesterday.

\exg.  Bondu a wo w-ando tisa kunu\\
Bondu \textsc{AUX.PST} 2\textsc{PL} w-ando ask yesterday\\
Bondu asked you(pl) yesterday.

\exg. Bondu à ando tisa kunu\\
Bondu 3\textsc{AUX.PST} ando ask yesterday\\
Bondu asked him yesterday.

\exg. Bondu am fando tisa kunu\\
Bondu 3\textsc{AUX.PST} f-ando ask yesterday\\
Bondu asked them yesterday. 


\g{Future, o-series:}
\exg. Bondu Saa-o tisa wã́ sena\\
Bondu Saa-\textsc{o} ask \textsc{foc} tomorrow\\
Bondu will ask Saa tomorrow

\exg. Bondu ɲondo tisa a sena\\
Bondu who ask \textsc{fut?} tomorrow?\\
Who will Bondu ask tomorrow?

\exg. Bondu ɱ fando tisa a (*wã́) sena\\
Bondu 1\textsc{SG} f-ando ask \textsc{fut?} (*\textsc{foc?})  tomorrow\\
Bondu will ask me tomorrow.

\exg. Bondu moa ando tisa (*wã́) a sena\\
Bondu ? ando ask (*\textsc{foc?}) a tomorrow\\
Bondu will ask us tomorrow.

\exg. Bondu mò tisa (wã́) sena\\
Bondu ? ask \textsc{foc} tomorrow\\
Bondu will ask us tomorrow.
\g{More of a statement (no ando)}

\exg. Bondu ndo tisa (wã́)  sena.\\
Bondu ? ask \textsc{foc} tomorrow\\
Bondu will ask me tomorrow.

\exg. Bondu ndo tisa à sena.\\
Bondu ? ask \textsc{foc} tomorrow\\
Bondu will ask me tomorrow.

\exg. Bondu i-jo tisa (wã́) sena\\
Bondu 2SG-jo ask \textsc{foc} tomorrow\\
Bondu will ask you(sg) tomorrow.

\exg. Bondu i w-ando tisa a sena\\
Bondu 2SG w-ando ask \textsc{fut} tomorrow\\
Bondu will ask you(sg) tomorrow.

\exg. Bondu wo \textbf{wã́} n-do tisa a sena\\
Bondo 2PL \textsc{foc} ando ask \textsc{a} tomorrow\\
Bondu will ask you(pl) tomorrow.

\exg. Bondu wo tisa a sena\\
Bondo 2PL ask \textsc{a} tomorrow\\
Bondu will ask you(pl) tomorrow.

\exg. Bondu wo tisa wã́  sena\\
Bondo 2PL wa ando ask \textsc{a} tomorrow\\
Bondu will ask you(pl) tomorrow.


\section{Daniel}
\ds{Plan: Compare argument+adjunct fronting in different (non-past) tenses, check distribution of na/ni\\
Check more systematically if na/ni are actually clefts, distribution together with wã/fã\\
If possible, get more minimal pairs with w/f alternation}

\exg. Bondu-a na tombwe do kunu?\\
Bondu-3SG FOC/COP dance ? yesterday\\
`Was it Bondu that danced yesterday?'

\exg. Bondu-*(a) ni tombwe do kunu\\
Bondu-3SG ni dance \textit{v} yesterday\\
`It was Bondu who was meant to have danced yesterday'

\exg. (No,) Kai-a na/*ni tombwe do kunu\\
{} Kai-3SG FOC/ni dance \textit{v} yesterday\\
`No, it was Kai who danced yesterday'\\
\ds{Comment: cannot use \textit{ni} because now you are certain}

\exg. San\textipa{\textbardotlessj}e-mu Bondu tombwe dõ-da\\
right.now-COP Bondu dance \textit{v}-?\\
`Is it right now that Bondu is dancing?'

\exg. Bíí-mu Bondu tombwe dõ-da\\
today-COP Bondu dance \textit{v}-?\\
`Is it today that Bondu is dancing?'

\exg. Bondu tombwe dõ-téa bíí-wã?\\

`Is it today that Bondu is dancing?' \jf{has a defn similar to this}

\exg. Bíí-ni Bondu tombwe dõ-da\\
today-ni Bondu dance \textit{v}-?\\
`It is today that Bondu is meant to dance'

\exg. Sina-mu Bondu tombwe dõ-da?\\
tomorrow-COP Bondu dance \textit{v}-?\\
`Is it tomorrow that Bondu will dance?'

\exg. *Sina-ni B tombwe dõ-da\\
tomorrow-ni B dance \textit{v}-?\\

\exg. Kunu-ni Bondu-a tombwe du?\\
yesterday-ni Bondu-3SG dance \textit{v}\\
`Was it yesterday that Bondu danced?'

\exg. Kunu-mu Bondu-a tombwe du?\\
yesterday-ni Bondu-3SG dance \textit{v}\\
`Was it yesterday that Bondu danced?'

\exg. *Kunu-mu-ni/kunu-ni-mu Bondu-a tombwe du?\\
yesterday-mu-ni/yesterday-ni-mu Bondu-3SG dance \textit{v}\\

\exg. Bondu-a séé-ndo téé?\\
Bondu-3SG.PST sEE-obj? break\\
`Was it the sEE that Bondu broke?'

\exg. *Bondu-a séé-ni-o/séé-mu-o téé\\
Bondu-3SG.PST sEE-ni-OBJ?/sEE-COP-OBJ? broke\\

\exg. Bondu náá tombwe do Kai-a-na\\
Bondu ? dance \textit{v} Kai-?-with?\\
`Was it with Kai that Bondu danced?'

\exg. Kai-ni Bondu-a tombwe do-a\\
Kai-ni Bondu-3SG dance \textit{v}-?\\
`Was it Kai that Bondu danced with?'

\exg. Kai-mu Bondu-a tombwe do-a\\
Kai-COP Bondu-3SG dance \textit{v}-?\\
`Was it Kai that Bondu danced with?'

\exg. Kunu-ni Bondu-a tombwe du?\\
yesterday-ni Bondu-3SG dance \textit{v}\\
`Was it yesterday that Bondu danced?'

\exg. Jai-ni Bondu-a tombwe du\\
last.year-ni Bondu-3sg dance \textit{v}\\
`Was it last year that Bondu danced?'

\exg. Bondu a-na tombwe do kunu\\
Bondu 3SG-na dance \textit{v} yesterday\\
`It was Bondu that danced yesterday'

\exg. *Bondu a-mu tombwe do kunu\\
Bondu 3SG-COP dance \textit{v} yesterday\\
`It was Bondu that danced yesterday'

\exg. Bondu mfea Kai-a... \textipa{\textltailn}ona tombwe do kunu?\\
Bondu or Kai-3SG who dance \textit{v} yesterday\\
`Did Bondu or Kai dance yesterday?'\\
\ds{Literally: `Bondu or Kai -- who danced yesterday?'}



\section{Chun-Hung}

\chs{\textbf{A. Status of grammatical subjects}} \newline

\chs{0. Baseline for control domains; other verbs: cure, help}

\exg. Mw\`{e}d\'{a}nd\'{a}mw\`{e} \'{m}-bã́ṍ. \\
doctor 1SG-heal \\
`The doctor healed me.' 

\exg. Mw\`{e}d\'{a}nd\'{a}mw\`{e} \'{m}-b\'{a}\'{o}nd-a-wã́. \\
doctor 1SG-heal-A-WA \\
`The doctor will heal me.' 

\exg. Mw\`{e}d\'{a}nd\'{a}mw\`{e} t\'{u}-m\'{u} \`{a}n\`{i} \'{m}-bã́ṍ. \\
doctor like-MU 3SG 1SG-heal \\
`The doctor wants [to heal me].' \chs{subject as being controlled PRO}

\exg. \'{N}-t\'{u}-m\'{u} mw\`{e}d\'{a}nd\'{a}mw\`{e} nĩ̀(/*\'{m}-)-bã́ṍ.\\
1SG-like-MU doctor 1SG(/*1SG.SER1)-heal \\
`I want [to be healed by the doctor].'

\jal{I have nì \'{m}-b\'{ã}\'{õ}, it's the same ni as in the previous example, and the m- is the 1sg object;  probably shouldn't translate it as passive}

\ex. (*)I want [to the doctor treat]. \chs{object as being controlled PRO; expected to be ungrammatical} 

\chs{The embedded domain does not seem to be a control structure, given that the special pronominal morpheme can target those that are not grammatical subjects. The other two sentences could be included `The doctor wants to be healed by me.' and `I want to heal the doctor.' for comparisons.} \jal{not very natural sentences, I guess if you're a doctor too, but since these aren't control and not your primary focus, probably not worth spending time on}\newline

\chs{1. Predicative possessives}

\ex. I have a dog. 

\exg. W\'{u}\'{u} m\'{a} wã́ m-b\'{o}\`{o} (j\`{a}j). \\
dog FUT WA 1SG-hand (next.year) \\
`I will have a dog next year.' 

\exg. W\'{u}\'{u} m\'{a} \textipa{M}f\'{a}m-b\'{o}\`{o}. \\
dog FUT 1SG.STRONG-hand \\
`It will be me that will have a dog.' \chs{focus interpretation}

\exg. W\'{u}-t\textipa{S}\`{\textipa{E}} t\'{u}-m\'{u} \`{a}n\`{i} m\'{a} \'{m}-b\'{o}\`{o}. \\
dog-this like-MU 3SG FUT 1SG-hand \\
`The dog would like me to be his owner.' 
(`The dog wants to be owned by me.')

\exg. W\'{u}-t\textipa{S}\`{\textipa{E}} t\'{u}-m\'{u} \`{a}n\`{i} m\'{a} \textipa{M}f\'{a}m-b\'{o}\`{o}. \\
dog-this like-MU 3SG FUT 1SG-hand \\
`The dog would like me to be his owner.'
(`The dog wants to be owned by me.')

\exg. \'{N}-t\'{u}-m\'{u} w\'{u}\'{u} nĩ̀ m\`{a} m-b\'{o}\`{o}. \\
1SG.SER1-like-MU dog 1SG FUT 1SG-hand \\
`I want to have a dog next year (when I live alone).' \chs{the morpheme ni is not in the position right before `hand' as regular possessors are}


\chs{Though the position of ni is unexpected (may suggest that possessors raise higher than the original position???), this construction cannot tell which one is the grammatical subject, and I will set aside this test/construction for now.} \newline

\chs{\textbf{B. C-commanding relations between possessors and possessums} --- to test whether possessums are structurally higher than possessors as PP predication, or possessors (with delayed saturation) are structurally higher than possessums as proposed in Myler (2016)} \newline

\chs{1. Condition C}

\exg. B\`{o}nd\'{u}\textsubscript{i} (\`{a}) \`{a}\textsubscript{i}-d\`{e} jẽ̀ẽ̀. \\
Bondu (3SG.PST) 3SG-mother see \\
`Bondu\textsubscript{i} saw her\textsubscript{i} mother'. \chs{the brackets indicate that the morpheme should be there but indistinguishable from the surrounding}

\exg. \`{A}\textsubscript{?i/k}-d\'{e} \`{a} B\`{o}nd\'{u}\textsubscript{i} jẽ̀ẽ̀. \\
3SG-mother 3SG.PST Bondu see \\
`Her\textsubscript{?i/k} mother saw Bondu\textsubscript{i}'. \chs{Tony seems to like the pronoun to refer to a third person.}

\exg. B\`{o}nd\'{u}\textsubscript{i}-d\`{e} (\`{a}) \`{a}\textsubscript{i/k} jẽ̀ẽ̀. \\
Bondu-mother (3SG.SER1) 3SG.PST see \\
`Bondu\textsubscript{i}'s mother saw her\textsubscript{i/k}.'

\exg. B\`{o}nd\'{u}\textsubscript{i}-d\`{e} \`{a} B\`{o}nd\'{u}\textsubscript{i} jẽ̀ẽ̀. \\
Bondu-mother 3SG.PST Bondu see \\
`Bondu\textsubscript{i}'s mother saw Bondu\textsubscript{i}.' \chs{Tony prefers repeating the name to refer to the same person}


\section{Wesley}

\wml{As before, we have seen two patterns for conjunction: one uses \textit{ni} which Tony translates to \textit{and}, and the other uses \textit{féà} `two'.}\\

\wml{With \textit{ni}, we've seen just the two conjuncts and \textit{ni} intervening. If one of the conjuncts is pronominal, it is the strong pronoun:}

\exg. ɱ́-fã́ nì Bondu, ná-à kwɛ̀ tú kúnú\\
\textsc{1sg-strong} and {} \textsc{1EXCL-PST} rice pound yesterday\\
`Bondu and I, we pounded rice yesterday.’

\exg. Bondu nì ɱ́-fã́, ná-à kwɛ̀ tú kúnú\\
{} and \textsc{1sg-strong} \textsc{1EXCL-PST} rice pound yesterday\\
`Bondu and I, we pounded rice yesterday.’

\exg. í-wã́ nì Bondu, w-á kwɛ̀ tù kúnú\\
\textsc{2sg-strong} and Bondu, \textsc{2PL-PST} rice pound yesterday\\
`You (sg.) and Bondu, you (pl.) pounded rice yesterday.’

\exg. ó-wã́ nì Bondu w-á kwɛ̀ tú kúnú\\
\textsc{2pl-strong} and {} \textsc{2PL-PST} rice pound yesterday\\
`You (pl.) and Bondu, you pounded rice yesterday.’

\wml{With the \textit{féà} strategy, we see \textsc{ser1}\mss{i}-féà conjunct\mss{k} \textsc{ser2}\mss{k}.\\\\The \textsc{ser2} pronoun agrees with the second conjunct (or at least whatever is the second conjunct in English) while, interestingly, the first \textsc{ser1}\mss{i} agrees with the conjoined entity/plurality, e.g., in (\ref{bondu_and_i}), the pronominal prefix on \textit{fea} seems to be \textsc{1pl.excl} rather than \textsc{1sg} as I had initially heard---the difference is tonal. Similarly in (\ref{you_and_bondu}), it is `\textsc{2pl}-fea Bondu' to mean `\textsc{2sg} and Bondu'.}

\exg. ɱ̀-féà Bondu à, *(ná-à) kwɛ̀ tú kúnú\\
\textsc{1SG}-two {Bondu} \textsc{a} \textsc{1EXCL-PST} rice pound yesterday\\
`Bondu and I, we pounded rice yesterday.’\label{bondu_and_i}

\exg. ó-féà Bondu à w-á kwɛ̀ tú kúnú\\
\textsc{2pl}-two {Bondu} \textsc{a} \textsc{2PL-PST} rice pound yesterday\\
`You (sg.) and Bondu, you (pl.) pounded rice yesterday.’\label{you_and_bondu}

\wml{Further evidence is provided by (\ref{you_alex_bondu}). Here, the conjuncts are \textsc{2sg}, Alex, and Bondu; \textsc{2sg} and Alex are combined using the \textit{fea} strategy, and \textit{Alex} is followed by the \textsc{ser2} form \textit{ja} while \textit{fea} is preceded by \textsc{2pl.ser1}. This pattern reminds me of Greek which allows `we went with Stavro' (pigame me ton Stavro) to mean 'Stavro and I went'.}

\exg. ó-féà Alex j-á nì Bondu w-á kwɛ̀ tú kúnú\\
\textsc{2pl}-two {Alex} \textsc{2SG-PST} and {Bondu} \textsc{2PL-PST} rice pound yesterday\\
`You, Alex, and Bondu, you (pl.) pounded rice yesterday.’\label{you_alex_bondu}


\exg. à-ɱ-féà Bondu à, à-n-á kwɛ̀ tú kúnú\\
\textsc{3-pl}-two {Bondu} \textsc{a}, \textsc{3-PL-PST} rice pound yesterday\\
`They and Bondu, they pounded rice yesterday.’\\
\wml{Based on my above analysis, this should also be interpretable as `he and Bondu'.}

\wml{Relative clauses}

\exg. í-wɛ́nà Bondu sɔ̃̀, é nàà wã̀ sìná\\
\textsc{2sg}-\textsc{wena} Bondu know \textsc{2sg.NPST} come \textsc{wa} tomorrow\\
`You (sg.), who know Bondu, are coming tomorrow.’\hfill{(old)}

\exg. * í-mín-à Bondu sɔ̃̀, é nàà wã̀ sìná\\
\textsc{2sg}-\textsc{mi-na} Bondu know \textsc{2SG.NPST} come \textsc{wa} tomorrow\\
`You (sg.), who know Bondu, are coming tomorrow.’\\
\wml{The relative pronoun \textit{mi} cannot be used for pronominal heads, although we've seen that they can be used for appositives.}

\exg. í-*(wɛ́)*(nà) Bondu sɔ̃̀, é nàà wã̀ sìná\\
\textsc{2sg}-\textsc{wena} Bondu know \textsc{2SG.NPST} come \textsc{wa} tomorrow\\
`You (sg.), who know Bondu, are coming tomorrow.’\\
\wml{So both those pieces are obligatory!}

\exg. é nàà wã̀ sìná, *(í)-wɛ́nà Bondu sɔ̃̀ \jf{W/o í it doesn't work}
\textsc{2SG.NPST} come \textsc{wa} tomorrow \textsc{2sg}-\textsc{wena} Bondu know \\
`You (sg.) are coming tomorrow, you (sg.) who know Bondu,.’

\exg. g͡bòò féà à-m-bè Bondu bòò. ɛ̀ dòndwɛ́ kààndá {[} Kai à mím-bè à-mà {]}\\
book two \textsc{3-PL-NPST} {} hand \textsc{3SG.NPST} one read {} Kai \textsc{3SG.PST} \textsc{mi}-give \textsc{3sg}-to {}\\
`Bondu has two books. He is reading the one that Kai gave him.’

\exg. g͡bòò féà àmbè Bondu bòò. ɛ̀ g͡bòò kààndá {[} Kai à mím-bè à-mà kúnú ]\\
book two \textsc{3pl} {} hand \textsc{3SG.NPST} book read {} Kai \textsc{3SG.PST} \textsc{mi}-give \textsc{3sg}-to yesterday {}\\
`Bondu has two books. He is reading the book that Kai gave him yesterday.’

\wml{Here I was interested in seeing if a \textit{mi}-clause could be used with a nominal head as a fragment answer, to support the status of the entire phrase as a nominal.}

\ex. Q: Which book is Bondu\mss{k} reading?\\
\ag. A: Kai\mss{j} à\mss{j} g͡bòò mím-bè à\mss{k}-mà kúnú\\
{} Kai\mss{j} \textsc{3SG.PST}\mss{j} book \textsc{mi}-give \textsc{3sg}\mss{k}-to yesterday\\
`The book that Kai gave him yesterday.'\\

\section{Mingyang}
\begin{itemize}
    \item Follow-up on descriptive anaphora;
    \exg. kamwe kante-a. n kwa kunu (ã) kamwe tea.\\
        teacher place.of.learning-A 1.SG speak yesterday ANA teacher DEM P\\
   `There is a teacher in the place of learning. I talked to the teacher yesterday.'

   \exg. kamwe kante-a. (ã) kamwe (t͡ʃɛ) si-o gba.\\
   teacher place.of.learning-A ANA teacher DEM sit-O back\\ 
   There is a teacher in the place of learning. The teacher sits in the back.

   \exg. kamwe n-fea kande-a a-m-be kante-a. n kwa kunu (ã) kamwe (t͡ʃɛ) tea.\\
   teacher and student-A 3-PL-NPST place.of.learning-A 1.SG speak yesterday ANA teacher DEM P\\
   There is a teacher and a student in the place of learning. I talked to the teacher yesterday.
   
   \exg. kamwe n-fea kande-a a-m-be kante-a. t͡ʃe (ã) kamwe (t͡ʃɛ) si-o (ã) kande (t͡ʃɛ) bema.\\
   teacher and student-A 3-PL-NPST place.of.learning-A but ANA teacher DEM sit-O ANA student DEM near\\
   There is a teacher and a student in the place of learning. The teacher sits near the student.

   \item Intensionally independent reading of definites
   \exg. sand͡ʒa manse-a waj banda wa. jaj \{ã mansa t͡ʃɛ/ mansa t͡ʃɛ/ ã mansa\} ma sene-waj-t͡ʃe-mwe na.\\
    this.year chief-A work finish WA next.year ANA chief DEM chief DEM ANA chief MA farmer NA\\
   This year, the chief$_i$ will finish his work. Next year, the chief$_i$ will become a farmer.

    \mb{\textit{manse} and \textit{mansa} are probably both the bare form for `chief' and the difference is merely phonetic; I'll double check with Anthony.}
    

   \item Donkey anaphora 
    \exg. wuu-a sene-t͡ʃe-mwe g͡bɔɔ. sene-t͡ʃe-mwe t͡ʃɛ saj-boa wuu (t͡ʃɛ) a.\\
    dog-A farmer have farmer DEM play.with dog DEM A\\
    Intended elicitation: `If a farmer has a dog, he pets \textbf{the dog}.'
    
    \mb{Anthony didn't produce a conditional structure. Instead, the sentence he gave was something like `A farmer has a dog. The farmer plays with the dog.' Nonetheless, this could serve as a baseline for `every farmer who has a dog plays with the dog.'}

   
\end{itemize}

\section{Jan}

\subsection{[a] + [$\varnothing$i]}

\ex. gbô \\
`(to) curse'

\ex. $x$ à í gbô kunu \\
$x$ 3SG.PST 2SG curse yesterday \\
`$x$ cursed you yesterday.'

\exg. bóndú é gbòò kúnù \\
bondu 2SG.NPST curse yesterday \\
`Bondu cursed you yesterday.'

\jmt{Underlyingly [bóndú $\varnothing$à $\varnothing$í gbòò kúnù]. Also, [òò] lasts for 0.235s, where [ú] + [é] lasts for 0.189s, and [é] for approximately 0.082s. This suggests that [é] is not reduplicated.}

\ex. bóndú tʃéé í gbóò kùnù \\
bondu husband 2SG curse yesterday \\
`Bondu's husband cursed you yesterday.' 

\jmt{[í]: 0.118; [éé]: 0.190}

\ex. jéé í gbòò kùnù \\
jai. 2SG curse yesterday \\
`Jai cursed you yesterday.'

\jmt{It's not clear to me here whether this [é] is doubled or the [í] should be a [j].}

\ex. féà ì gbòò kùnù \\
fea 2SG curse yesterday \\
`Fea cursed you yesterday.'

\jmt{The [à] is really more of an [ɐ] or [ɛ] than a properly open [a].}

\ex. sá í gbòò kùnù \\
Saa 2SG curse yesterday \\
`Saa cursed you yesterday.'

\ex. í djè-j gbòò kùnù \\
2SG.POSS mother-2SG curse yesterday \\
`Your mother cursed you yesterday.'

\ex. í fáà ì gbòò kùnù \\
2SG.POSS father 2SG curse yesterday \\
`Your father cursed you yesterday.'

\subsection{[a] + [ii]}

\ex. íí \\
`water/river/brain'

\ex. n-dʒíì \\
`my brain'

\jmt{[íì]: 0.387s. Anthony also produced [n ndʒíì].}

\ex. (í) íì \\
`your brain'

\jmt{This lasted for 0.573s, which is considerably longer than other eliciations of [íì]. The final drop in tone lasted for about 0.12s}

\ex. ó íí \\
`you all's brains'

\jmt{This was produced with a markedly more closed [o]. The [íì] was also produced with a similar duration as the previous example; perhaps the additional [$\varnothing$í] did not affect the data.}

\ex. bóndú íì \\
`Bondu's brain'

\ex. bóndú tʃé íì \\
`Bondu's husband's brain' \label{husband_brain}

\ex. jéj íì \\
`Jai's brain'

\jmt{If we underlying predict that if the inalienable possessor for \textsc{3SG} is [à] + [à], there may be a difference here.}

\ex. n-a íì \\
1SG-POSS water\\
`my water' 

\jmt{[íì]: 0.376s}

\ex. j-á íì \\
2SG-POSS water\\
`your water'

\jmt{There was some rounding on the [íì] here, and [a] was again raised to [ɐ].}

\ex. $x$ à íí \\
$x$ 3SG water \\
`$x$'s water'

\ex. bóndú é íì \\
`Bondu's water'

\jmt{This is unexpected: we get [$\varnothing$a] + [íì] => [é íì]}

\ex. bóndú tʃéé íì \\
`Bondu's husband's water' 

\jmt{Interesting - in \ref{husband_brain}, we get that the [é] lasts about 0.16s, where here it's closer to 0.3s. So, in this example, the [à] sticks to the [tʃéé] and doesn't stand by itself.}

\ex. jéé íì \\
`Jai's water'

\jmt{It's unclear whether there's a [j] preceding [íì].}

\ex. féà jíì \\
`Fea's water'

\jmt{I transcribed a [j] here because the segment after the [á] was quite long - nearly 0.5s.}

\ex. sáá jíí \\
`Saa's water'

\ex. í djè íì \\
`Your mother's water'

\ex. í fàà íì \\
`Your father's water'

\subsection{Miscellany}

\exg. ń tó à gbè à wàn tì mà dòò nà \\
1SG ear 3SG.OBJ strong FOC ? COP? ? ? \\
`I am too stubborn sometimes.'

\jmt{[o] + [i]}

\exg. í tóó à gbé à wà tì mà dòò nà \\
2SG ear 3SG strong 3SG? FOC ? COP? ? ? \\
`You are too stubborn.'

\ex. ó tóó à gbè à wà tì mà dòò nò \\
2PL ear 3SG strong 3SG FOC ? COP? ? ? \\
`You all are too stubborn.'

\subsection{Time Permitting: pranking}

\exg. jéj á bóndú jáà \\
Jai 3SG.PST Bondu prank \\
`Jai pranked Bondu.'

\exg. jéj á sàà jà \\
Jai 3SG.PST Saa prank \\
`Jai pranked Saa.'

\exg. jáj á-á jáà \\
Jai 3SG.PST-3SG.OBJ prank \\
`Jai pranked him.'

\exg. jé $\varnothing$-í jà \\
Jai 3SG.PST-2SG.OBJ prank \\
`Jai pranked you.'

\jmt{This is strange - we heard an [a] in all cases except for this one. However, Anthony also said [jéj é í jà] the first time.}

\exg. bóndú é í jà \\
Bondu ? 2SG prank \\
`Bondu pranked you.'

\exg. bóndú á wó jà \\
Bondu 3SG.PST 2PL prank \\
`Bondu pranked you all.'

\jmt{It doesn't seem we get much change in [$\varnothing$a] + [$\varnothing$o]}

\exg. jé á wó jà \\
Jai 3SG.PST 2PL prank \\
`Jai pranked you all.'

\subsection{Testing [uu]}

\exg. jáj úú \\
Jai short \\
`Jai is short.'

\exg. bond-úú \\
Bondu-short \\
`Bondu is short.' \label{bondu_short}

\jmt{[u] is held for 0.56s here.}

\exg. í úú \\
2SG short \\
`You are short.'

\jmt{I could not decide whether to transcribe this as [júú] or not.}

\exg. jɛ́ɛ́ $\varnothing$ úú \\
Jai 3SG dog \\
`Jai's dog'

\jmt{This is the clearest example of raising to [ɛ]. Tony then says [jaj a uu] afterwards.}

\exg. bóndú á úú \\
Bondu 3SG dog \\
`Bondu's dog'

\exg. já úù \\
2SG dog \\ 
`Your dog'

\jmt{[uu] is held for a long time here: 0.81s and 0.77s, and with a different tonal structure. He also offered [já úú] + [ú] as an analysis.}

\jmt{The length of this [uu] is approximately the same length as \ref{bondu_short}, 0.54s.}




\section{Joey}



\subsection{One sentence or two:}

\exg. kai a feɱ fɔ-i-je?\\
Kai 3SG.PST what say-2SG-to\\
`What did Kai tell you?' \\

\exg. bondu a fen saŋ kunu?\\
Bondu 3SG.OBJ what buy yesterday\\
`What did Bondu buy yesterday?' \\

\exg. kai ɔ te bondu a fen sã kunu ni a fen dao kunu\\
Kai 3SG.OBJ say Bondu 3SG.PST what buy yesterday Q.C(?) AUX.3SG.PST what eat yesterday\\
`What did Kai tell you? What did Bondu buy yesterday?' \\

\jf{In order to get the double answer this construction is needed. This answer was crucially impossible with our general construction (ie: that construction is one question, not two)}\\

\jf{Also, NB: the "ni" here sounded longer and meaningfully different from the regular interrogative complementizer, to an extent. it paired more with the following "a" to sound like "nija" or maybe even "nii a"}\\

\jf{Answer:}

\exg. kai a suee-an san kunu\\
Kai 3SG.PST meat-FOC buy yesterday\\
`Bondu bought meat' \jf{(This is the answer to our general question, not the above double question)}\\


\exg. kwɛ̀\\
`rice'.\\

\exg. kai ɔ te a kwɛ̀ dao \\
Kai 3SG.OBJ say 3SG.PST rice eat   \\
`Kai said he ate rice.\\

\exg.  kàì ɔ̀ té á kwɛ̀ dàò / bòndú á swèè-àn sàŋ kùnù\\
Kai 3SG.OBJ say 3SG.PST rice eat / Bondu 3SG.PST meat-FOC buy yesterday\\
`Kai said he ate rice. Bondu bought meat yesterday'\\

\jf{Compare intonation of the above against that of:}\\

\exg. kai a feɱ fɔ-i-je bondu a fen saŋ kunu?\\
Kai 3SG.PST what say-2SG-to Bondu 3SG.PST what buy yesterday\\
`What did Kai tell you that Bondu bought yesterday?' \\

\jf{And compare answers:}\\

\exg. kai ɔ te bondu a suee-an san kunu\\
Kai 3SG.OBJ say Bondu 3SG.PST meat-FOC buy yesterday\\
`Kai said that Bondu bought meat' \\

\exg. kai a fɔ-n-je ɔ bondu a suee-an san kunu\\
Kai 3SG.PST say-1SG-to 3SG.OBJ Bondu 3SG.PST meat-FOC buy yesterday\\
`Kai told me that Bondu bought meat' \\

\jf{QUESTION: What is this ɔ doing here before Bondu?}


\jf{The following sentences can be tested against the later attempt at a Y/N embedded question:}

\exg. kàì á fèɱ fɔ̀-ì-jé? bòndú á swéé sáɱ-fáŋ kúnú\\
Kai 3SG.PST what say-2SG-to Bondu OBJ meat buy-FOC yesterday\\
`What did Kai tell you? Did Bondu buy meat yesterday?' \\

\subsection{Embed a yes/no question? (wh-scope markers cannot have this):}

\jal{Hindi can, Hungarian can't without an embedded wh-phrase}

\ex. ní \\
`whether' (interrogative C-head, seemingly) \jal{ah I have this transcribed as ní, could check formants for vowel height}\\ 

\exg. kai a feɱ fɔ-i-je bondu a fen saŋ kunu?\\
Kai 3SG what say-2SG-to Bondu 3SG.PST what buy yesterday\\
`What did Kai tell you that Bondu bought yesterday?' \\

\exg. kai a-ndo tisa bondu a fen sã kunu\\
Kai 3SG.PST-1SG.ɔSER ask Bondu 3SG.PST what buy yesterday\\
`Kai asked me what Bondu bought yesterday. \\ 

\exg. kai a-ndo tisa ni bondu a suee saɱ-fa kunu\\
Kai 3SG.PST-1SG.ɔSER ask Q.C Bondu 3SG.PST meat buy-FOC yesterday\\
`Kai asked me whether Bondu bought meat yesterday.\\ 

\exg. *kai a fɔ-n-d͡ʒe ne bondu a suee saŋ kunu?\\
Kai 3SG.PST say-1SG-to COMP.Q Bondu 3SG.PST meat buy yesterday\\
`Kai told me whether Bondu bought meat yesterday.' \\

\jf{Does not seem able to embed Y/N question with this matrix "what."}

\exg. n-ɔ soN-(fa) (*ni) bondu a suee-an saŋ kunu\\
1SG-3SG.ɔSER know-(FOC) (*Q.C) Bondu 3SG.PST meat-FOC buy yesterday\\
`I know that Bondu bought meat yesterday. ' \\

\jf{This is an interesting example of there being FOC in both matrix and embedded clasues. It is also notable that you cannot have an embedded Y/N Question under "know" apparently.}\\

\jf{Based on the above, can we make a sentence by adding matrix wh? If no, that suggests no yes/no embedding under matrix wh:}\\


\subsection{Reconfirm other apparent long-distance questions:}

\exg. kai a inat͡ʃii bondu a fen saŋ kunu\\
Kai 3SG? think Bondu 3SG.PST what buy yesterday\\
`Kai was thinking as to what Bondu bought yesterday' \\

\jf{This is not a question "Not a question, absolutely a statement."}\\



\exg. kai inat͡ʃii-ɔ kaama bondu a fen saŋ kunu\\
Kai think-3SG.OBJ concerning Bondu 3SG.PST what buy yesterday\\
`Kai is thinking through what Bondu bought yesterday/Kai is in the thought process about what Bondu bought yesterday'\\

\jf{This is not a question}\\


\jf{Try to elicit long-distance question with te:}

\exg. Kai ɔ te?\\
Kai OBJ say\\
`Kai said...' \jf{It's like starting a sentence, waiting for the interlocutor to finish it for you}\\

\exg. Kai ɔ te fẽ?\\
Kai OBJ say what\\
`What did Kai say?' \jf{(NB: This one was also paired with the English translation of the lingering, "Kai said..."} \\

\exg. bondu a fen sã?\\
Bondu 3SG what buy \\
`What did Bondu buy?' \\


\exg. Kàì ɔ̀ té bòndú á fèn sàŋ kùnù\\
Kai OBJ say Bondu 3SG.PST what buy yesterday\\
`What did Kai say that Bondu bought?'\\

\jf{If there is a way to do it, confirm that it's one utterance:}

\jf{Answer:}

\exg. Kàì ɔ̀ té bòndú á swèè-àn sàŋ kùnù\\
Kai OBJ say Bondu 3SG.PST meat-FOC buy yesterday\\
`Kai said that Bondu bought meat.'\\

\jf{This appears to be a proper example of a long-distance question}



\end{document}