\documentclass{assets/fieldnotes}

\title{Kono (Sierra Leone)}
\author{LING3020/5020}
\date{University of Pennsylvania, Spring 2025\\03/26/2025 Class Projects Week 1}

\setcounter{secnumdepth}{4} %enable \paragraph -- for subsubsubsections

\begin{document}

\maketitle

\maketitle
\tableofcontents


\section{Wh/focus - Daniel}
\exg. Bondu-a s\textipa{\`E\`E} o-téé kúnú\\
Bondu-AUX.PST sɛɛ AUX.PST.OBJ-break yesterday\\
`Bondu broke the sɛɛ yesterday'

\exg. Bondu-a s\textipa{\`E\`E} o-téé-a wã sìná\\
Bondu-3SG.NPST sɛɛ 3SG.OBJ-break-A \textit{wã} tomorrow\\
`Bondu will break the sɛɛ tomorrow'

\exg. Bondu-a na s\textipa{\`E\`E} o-téé?\\
Bondu-3SG COP sɛɛ AUX.PST.OBJ-break\\
`Did Bondu break the sɛɛ'?
\ds{Comment: `Was it Bondu that broke the sɛɛ', narrow-focused cleft}

\exg. Tiá-tiá Bondu a na s\textipa{\`E\`E} o-téé?\\
true Bondu AUX.PST COP sɛɛ AUX.PST.OBJ-break\\
`Is it true that Bondu broke the sɛɛ'?\\
\ds{Some kind of verum marker/`really'?}

\exg. Àà, Bondu-a na s\textipa{\`E\`E} o-téé\\
yes Bondu-AUX.PST COP sɛɛ AUX.PST.OBJ-break\\
`Yes, Bondu broke the sɛɛ'


\exg. Bondu s\textipa{\`E\`E} o-téé-a wã\\
Bondu sɛɛ OBJ-break-A \textit{wã}\\
`Did Bondu break the sɛɛ?'
 
\exg. Àà, Bondu s\textipa{\`E\`E} o-téé-a wã\\
yes Bondu sɛɛ AUX.PST.OBJ-break-A \textit{wã}\\
`Yes, Bondu will break the sɛɛ'

\exg. A-a, (Bondu i-sɛɛ o-téé-a.) é sa\textipa{N}gan do téé-a (*wã)\\
no Bondu \Neg{}-sɛɛ AUX.PST.OBJ-break-3SG. 3SG drum \textit{v} break-A \textit{wã}\\
`No, Bondu will not break a sɛɛ. He will break a drum.'

\exg. A-a, Bondu sa\textipa{N}gan ɲáán-d-a (*wã)\\
no Bondu drum fix-\textit{v}-AUX.PST.OBJ \textit{wã}\\
`No, Bondu will fix a drum'\\
\ds{Apparently not compatible with more narrow focus}

\exg. Bondu féé-ma (sina)\\
Bondu thing-\textit{ma} tomorrow\\
`What will Bondu do tomorrow?'

\exg. *Bondu féé-wã (sina)\\
Bondu thing-\textit{wã} tomorrow\\
Intended: `What will Bondu do tomorrow?'\\
\ds{No \textit{wã} with wh-}

\exg. Àà, Bondu o-téé kunu\\
yes Bondu 3SG.OBJ-break yesterday\\
`Yes, Bondu broke (it) yesterday'\\
\ds{Ellipsis/object drop and no wã in past}

\exg. Àà, Bondu o-téa wã sina\\
yes Bondu AUX.PST.OBJ-break \textit{wã} tomorrow\\
`Yes, Bondu will break it tomorrow'

\exg. Àà, Bondu o-téa sina\\
yes Bondu 3SG.OBJ-break tomorrow\\
`Yes, Bondu will break it tomorrow'\\
\ds{Comment: would really have to push it to be without wã}


\section{Wh/focus - Giang}

\g{Possible answers to questions:} %past tense, unerg
\exg. ɲɔ́-nà dí-t\textipa{S}\`{\textipa{E}} kùnù?\\
Who-\textsc{foc} cry-\textit{v} yesterday\\
``Who cried yesterday?''

\g{Established possible answers:}
\ex. Bondu.

\exg. Bondu á-nà dí-t\textipa{S}\`{\textipa{E}} kùnù\\
Bondu \textsc{a-foc} cry-\textit{v} yesterday.\\
``It was Bondu that cried yesterday.'' \g{This a response form.}

\g{You could answer with this even though it's really about a statement than a response, but it can also be used as a response:}
\exg. Bondu \'{a} d\'{i}-t\textipa{S}\`{\textipa{E}} k\`{u}n\`{u}. \\
Bondu A cry-\textit{v} yesterday \\
`Bondu cried yesterday.'


\exg. Bòndú á swéè dàó̃ kùnù\\
Bondu AUX.PST meat eat yesterday \\%
    `Bondu ate meat yesterday'

\exg. ɲɔ́nà swéè dàó̃ kùnù?\\
who meat eat yesterday\\
`Who ate meat yesterday?'

\g{Possible answers: (can also use the declarative form...)}
\ex. Bondu.

\exg. Bòndú á-nà swéè dàó̃ kùnù\\
Bondu \textsc{AUX.PST-foc} meat eat yesterday \\%
    `Bondu ate meat yesterday'\\
\g{Can say this out of the blue, but it's more of an answer to a question.}

\exg. Bòndú á fé(n) dàó̃ kùnù?\\
Bondu \textsc{AUX.PST} what eat yesterday\\
``What did Bondu eat yesterday?''

\g{Do I want to try Bondu a-na?}

\g{fé is from fénè}

\g{Possible answers?}

\exg. Bondu di-tfa w\'ã sena\\
Bondu cry-v \textsc{fut?} tomorrow\\
`Bondu will cry tomorrow'

\exg. ɲɔ́mbè dí-tfa a sénà?\\
Who cry-v a tomorrow\\
`Who will cry tomorrow?'

\exg. Bondu ambe di-tfe a sena\\
Bondu ? cry-v a tomorrow\\
`Bondu will cry tomorrow.' (Answer form)

\exg. ɲɔ́mbè ná Bájámà sénà?\\
Who arrive Baiama tomorrow\\
`Who will arrive in Baiama tomorrow?'

\exg. Bondu a-m-be na Baiamasena.\\
Bondu 3-PL-NPST arrive Baiama tomorrow\\
`Bondu will arrive in Baiama tomorrow.'
\g{Can the following be an answer?}

\exg. Bòndú ná (w\'ã) Bájámà sénà.\\
Bondu come (?) Baiama tomorrow\\
`Bondu will arrive in Baiama tomorrow.'

%future tense, transitive
\exg. Bondu (*ambe) swee daon da (w\'ã) sena\\
Bondu meet eat ? da (\textsc{fut}) tomorrow\\
`Bondu will eat meat tomorrow.'
\g{da is a form of \textsc{a} when the follows a word that ends in a nasal. Julie's theory: \textsc{a} is a future suffix.}

\exg. ɲɔ́mbè swee daon da (*w\'ã) sena?\\
Who meat eat ? (*\textsc{fut}) tomorrow\\
`Who will eat meat tomorrow?'

\exg. Bondu a-m-be swee daon da (*w\'ã) sena.\\
Bondu 3-PL-NPST meat eat ? (*\textsc{fut}) tomorrow\\
`Bondu will eat meat tomorrow' \g{(answer form)}


\g{Super important to note here: ambe and w\'ã CANNOT co-occur.}


\g{Possible answers?}

\exg. Bondu fen daon da sena?\\
Bondu what eat da tomorrow\\
`What will Bondu eat tomorrow?'
\g{Cannot add w\'ã anywhere.}

\g{Possible answers?}
\exg. Bondu swee a-n daon da\\
Bondu meat 3-PL eat ?\\
`It will be meat that Bondu eat.'

\exg. Bondu swee daon da w\'ã sena \\
Bondu meat ? eat fut? tomorrow\\
`Bondu will eat meat tomorrow.'

\section{Predicative possessives - Chun-Hung}

\chs{\textbf{Plan of this week}: try to get a descriptive sketch}

\chs{alienable relations (others: orange, medicine) (some of them are from previous elicitation)}

\exg. t\textipa{S}\'{\textipa{E}}n\`{e} wã́-m-b\'{o}\`{o}. \\
house WA-1SG.I-hand \\
`I have a house.'

\exg. t\textipa{S}\'{\textipa{E}}n\`{e} w\'{a}-\'{i}-b\'{o}\`{o}. \\
house WA-2SG.I-hand \\
`You have a house.'

\exg. t\textipa{S}\'{\textipa{E}}n\`{e} w\'{a}-\'{a}-b\'{o}\`{o}. \\
house WA-3SG.I-hand \\
`He has a house.'

\exg. t\textipa{S}\'{\textipa{E}}n\`{e} wã́-ã̀-b\'{o}\`{o}. \\
house WA-3PL.I-hand \\
`They have a house.'

\exg. t\textipa{S}\'{\textipa{E}}n\`{e} w\'{a}-B\`{o}nd\'{u}-b\'{o}\`{o}. \\
house WA-Bondu-hand \\
`Bondu has a house.'

\exg. t\textipa{S}\'{\textipa{E}}n\`{e} w\'{a}-k\`{a}n\'{i}nd\textipa{Z}\`{e}-b\'{o}\`{o}. \\
house WA-student-hand \\
`The student has a house.'

\exg. k\`{a}n\'{i}nd\textipa{Z}\`{e}, t\textipa{S}\'{\textipa{E}}n\`{e} w\'{a}-\'{a}-b\'{o}\`{o}. \\
student house WA-3SG.I-hand \\
`The student, he has a house.' \chs{could be topicalization}

\exg. t\textipa{S}\'{\textipa{E}}n\`{e} w\'{a}-k\`{a}n\'{i}nd\textipa{Z}\`{e}-n-b\'{o}\`{o}. \\
house WA-student-PL-hand \\
`These students have a house.' \chs{... *-nu- ... for PL}

\exg. k\`{a}n\'{i}nd\textipa{Z}\`{e}-nu, t\textipa{S}\'{\textipa{E}}n\`{e} wã́-ã̀-b\'{o}\`{o}. \\
student-PL house WA-3SG.I-hand \\
`The students, they have a house.' \chs{could be topicalization}

\exg. t\textipa{S}\'{\textipa{E}}n\`{e} m-b\'{e} wã́-m-b\'{o}\`{o}. \\
house 1SG-NPST WA-1SG.I-hand \\
`I have houses.'

\exg. t\textipa{S}\'{\textipa{E}}n-d\'{u} m-b\'{e} (wã́)-m-b\'{o}\`{o}. \\
house 1SG-NPST (WA-)1SG.I-hand \\
`I have five houses.' \chs{WA- seems to be optional with MBE, but needs double-checking}

\exg. t\textipa{S}\'{\textipa{E}}n-d\'{u} \^{a}mb\'{e} (*wã́)-m-b\'{o}\`{o}. \\
house A-M-BE (*WA)-1SG.I-hand \\
`I have five houses.' \chs{WA- seems to be prohibited with AMBE, but needs double-checking}


\exg. t\textipa{S}\'{\textipa{E}}n-d\textipa{Z}\'{i}\'{u} i-boo? \\
house-how.many 2SG.I-hand \\
`How many houses do you have?'

\chs{existential construction}:

\exg. W\'{u}-\`{a}n\`{e}. \\
dog-ANE \\
`This is a dog.' (\chs{elicitation: There is a dog.})

\exg. W\'{u}-\`{a}n\`{e} t\textipa{S}-\`{a}. \\
dog-ANE this-P \\
`This here is a dog.' (\chs{elicitation: There is a dog.})

\exg. W\'{u}-f-\`{a}n\`{e}. \\
dog-F-ANE \\
`These are dogs.' (\chs{need to check the translation})

\exg. W\'{u}-f-\`{a}n\`{e} t\textipa{S}-\`{a}. \\
dog-F-ANE this-P \\
`These here are dogs.' (\chs{need to check the translation})

\exg. t\textipa{S}\'{\textipa{E}}n-\`{a}n\`{e}. \\
house-ANE \\
`This is a house.' (\chs{elicitation: There is a dog.})

\exg. t\textipa{S}\'{\textipa{E}}n-\`{a}n\`{e} t\textipa{S}-\`{a}. \\
house-ANE this-P\\
`This here is a house.' (\chs{elicitation: There is a dog.})

\exg. W\'{u}-t\textipa{S}\`{\textipa{E}} t\textipa{S}\'{\textipa{E}}n-\`{a}. \\
dog-this house-P \\
`This dog is in the house.' (\chs{elicitation: There is a dog in the house.})

\exg. W\'{u}-t\textipa{S}\`{\textipa{E}} t\textipa{S}\'{\textipa{E}}n\`{e}-t\textipa{S}-\`{a}. \\
dog-this house-this-P \\
`This dog is in this house.' 

\section{Relative Clauses -- Wesley}


\exg. * Bondu à ẽ̀ẽ̀ {[} mwòkàmá\mss{k} mì-na\mss{k} sɛ̀ɛ̀ ɔ̀ té {]}\\
Bondu \textsc{3SG.PST} see {} man \textsc{mi}-\textsc{3sg} sɛɛ \textsc{3sg.obj} break {}\\
`Bondu saw the man who broke the sɛɛ.’

\exg. * Bondu à {[} mwòkàmá\mss{k} mì na\mss{k} sɛ̀ɛ̀ ɔ̀ té {]} ẽ̀ẽ̀ \\
Bondu \textsc{3SG.PST} {} man \textsc{mi}-\textsc{3sg} sɛɛ \textsc{3sg.obj} break {} see \\
`Bondu saw the man who broke the sɛɛ.’\\
\wml{Follow-up: Maybe put some adverbials after `see' to check if it sounds more acceptable with more material at the end. It could be a prosodic thing...? Also allows us to see how positioning differs if the adverb is modifying matrix vs. RC verb.}

\exg. * Bondu à sɛ̀ɛ̀ {[} mwòkàmá mì-ndò té {]} sã́\\
bondu \textsc{3SG.PST} sɛɛ {} man \textsc{mi}-\textsc{3SG.obj} break {} bought\\
`Bondu bought the sɛɛ that the man broke.'

\exg. Bondu à sɛ̀ɛ̀ sã́ {[} mwòkàmá mì-ndò té {]}\\
Bondu \textsc{3SG.PST} sɛɛ buy {} man \textsc{mi}-\textsc{3SG.OBJ} break {}\\
`Bondu bought the \textbf{sɛɛ} that the man broke.'

\exg. Bondu à sɛ̃̀ɛ̃́ sã́ {[} mwòkàmá mî-ndò té {]} \\
Bondu \textsc{3SG.PST} sɛɛ.\textsc{pl} buy {} man \textsc{mi}-\textsc{3pl.obj} break {}\\
`Bondu bought the \textbf{sɛɛs} that the man broke.'

\exg. Bondu à mwòkàmá ẽ̀ẽ̀ {[} \textbf{mì}-na sɛ̀ɛ̀ ɔ̀ té {]}\\
Bondu \textsc{3SG.PST} man see {} \textsc{mi}-\textsc{3sg} sɛɛ \textsc{3sg.obj} break {}\\
`Bondu saw the \textbf{man} who broke the sɛɛ.’\hfill{(old)}

\exg. Bondu à mwòkàmá-n dʒẽ̀ẽ̀ {[} \textbf{mîn-na} sɛ̀ɛ̀ ɔ̀ té {]}\\
Bondu \textsc{3SG.PST} man-\textsc{pl} see {} \textsc{mi}-\textsc{3sg} sɛɛ \textsc{3sg.obj} break {}\\ `Bondu saw the \textbf{men} who broke the sɛɛ.'

\exg. Bondu à mwòkàmá-n dʒẽ̀ẽ̀ {[} \textbf{mîn-na} sɛ̀ɛ̀ ndò té {]}\\
Bondu \textsc{3SG.PST} man-\textsc{pl} see {} \textsc{mi}-\textsc{3sg.obj} sɛɛ \textsc{3pl.obj} break {}\\ `Bondu saw the \textbf{men} who broke the \textbf{sɛɛs}.'

\exg. Bondu à mwòkàmá jẽ̀ẽ̀ {[} sɛ̀ɛ̀ ò té-à {]}\\
Bondu \textsc{3SG.PST} man see {} sɛɛ \textsc{3sg.obj} break-\textsc{a} {}\\ `Bondu saw a man breaking the sɛɛ.'

\exg. Bondu à mwòkàmá jẽ̀ẽ̀ {[} mí-mbɛ̀ sɛ̀ɛ̀ ò té-à {]}\\
Bondu \textsc{3SG.PST} man see {} \textsc{mi}-\textsc{3sg} sɛɛ \textsc{3sg.obj} break-\textsc{a} {}\\ `Bondu sees the man who will break the sɛɛ.'

\exg. Bondu à mwòkàmá-n dʒẽ̀ẽ̀ {[} mí-mbɛ́ sɛ̀ɛ̀ ò té-à {]}\\
Bondu \textsc{3SG.PST} man-\textsc{pl} see {} \textsc{mi}-\textsc{3pl} sɛɛ \textsc{3sg.obj} break-\textsc{a} {}\\ `Bondu sees the \textbf{men} who will break the sɛɛ.'

\section{Realization of nasals(inserted/production)-Lex}

\exg. tʃɛ́    wómfɛ́à mbè m-bòò\\
      house  7      1SG-NPST 1SG-hand\\
`I have 7 houses'

\jf{Unsure about m + bè present here, Chung Hung's data has it as MBE with analysis questioning co-presence with Wa.}

\exg. tʃɛ́ wɔ́ɔ́lwàè á bè-m-bòò\\
house  6  POSS BÉ-1SG-hand\\
`I have 6 houses'

\exg. \'{m}-m\'{a}m\`{a}-k\`{a}\`{a}\`{i} \\
1SG-grandparent-man \\
`my grandfather'

\exg. m-fáfá\\
1SG-grandfather\\
`my grandfather'

\exg. m-fáfá ni-à-wɛ́-fá\\
1SG-grandfather and-3SG-WƐ-father\\
`my grandfather's father'


\exg. úù\\
dog\\
`dog'

\exg. n-á úù\\
1SG.POSS dog\\
`my dog'

\exg. kòmbwɛ́\\
brain\\
`brain'

\exg. ìì\\
brain\\
`brain'

\exg. n-dʒìì\\
1SG-brain\\
`my brain'

 \exg. n-kòmbwɛ́\\
1SG-brain\\
`my brain'

\exg. tʃɛ́\\
husband\\
`husband'

\exg. n-tʃɛ́\\
1SG-husband\\
my husband

\exg.já\\
eye\\
`eye'

\exg. n-já\\
1SG-eye\\
`my eye'

\exg. n-dʒá\\
1SG-eye\\
`my eye'

\exg. wíì\\
blood\\
`blood'

\exg. n-wìì\\
1SG blood\\
`my blood'


\exg. pàâ\\
scar\\
`scar'

\exg. n-á pàâ\\
1SG.POSS scar\\
`my scar'

\exg. jã́sã̀\\
tall\\
`tall'

\exg. n-dʒã́sã̀\\
1SG-tall\\
`I am tall' \jf{Anthony always said pronunciation varied here, but tended to assimilate sound to [dʒ] from the original [j].}

\exg. n-jã́sã̀\\
1SG-tall\\
`I am tall'

\exg. ímátɛ̀\\
intelligent\\
`intelligent'

\exg. n-dʒímátɛ̀-mù\\
1SG-intelligent-?\\
`I am intelligent'

\jf{The mù could potentially be sometype of reflexive particle}

\exg. úú\\
short\\
`short'

\exg. kúndù\\
short\\
`short'

\exg. ŋ-wúú\\
1SG-short\\
`I am short'

\exg. n-kúndù\\
1SG-short\\
`I am short'

\exg. tɔ̀bɛ́\\
stubborn\\
`Stubborn'

\ex.n-tɔ̀bɛ́\\
1SG-stubborn\\
`I am stubborn'

\exg. wòò\\
crazy\\
`crazy'

\exg. ŋ-wòò-nɛ́\\
1SG-crazy-?\\
`I am crazy'

\exg. ŋ-wòò-nɛ́-mù\\
1SG-crazy-?-?\\
`I am crazy'

\jf{Tony said something about 'nɛ́-mú' referring to someone asserting that they are really crazy, not sure, could be part of reflexive again}

\exg. tʃíéndíà\\
to be afraid\\
`to (be) afraid’, `fear’\\

\exg. n-tʃíéndíɛ̀\\
1SG-afraid\\
`I am afraid'

\exg. n-tʃíéndɛ̀-mù\\
1SG-afraid-?\\
`I am afraid'

\jf{Potentially the reflexive marker again}


\section{Definiteness - Mingyang}
\mb{Plan: a descriptive sketch of how Kono encodes the two types of definites - unique and anaphoric.}
\subsection{Unique definites}
\begin{itemize}
    \item Immediate situation definites:
    \exg. wuu t͡ʃio te-a.\\
    dog street cross-A\\
    `\textbf{The dog} is crossing \textbf{the street}.'

    \exg. g͡boo tebu ma.\\
    book table on\\   
    `\textbf{The book} is on \textbf{the table}.'

    \exg. kajne banda waj ma.\\
    boy finish work MA\\
    `\textbf{The boy} finished \textbf{the work}.'
    
    \exg. Bondu banda busu ma.\\
    Bondu finish soup MA\\
    `Bondu finished \textbf{the soup}.'

    \mb{The status of particle \textit{ma} is unclear. One hypothesis is that `finish' is intransitive in Kono and \textit{ma} is a preposition.}

    \item Large situation definites:
    \exg. kao manda pinbi-o.\\
        moon light night-O\\
    `\textbf{The moon} shines at night.'

    \exg. tee manda ba.\\
        sun light back\\
        `\textbf{The sun} rises from the back.'
        
    \exg. mwe tumu tee-a.\\
        people like sun-A\\
        `People like \textbf{the sun}.'


    \item Part-whole bridging:
    \exg. n-a t͡ʃene sã. a-da kene.\\
        1SG.PST house buy 3.SG-door open\\
        `I bought a house. \textbf{The door} was open.' (Lit. `...its door is open.)

    \exg. n-a t͡ʃene sã. t͡ʃɛ̀ m-be a-da ɲaan tea wǎ.\\
        1SG.PST house buy bought 1SG-NPST 3.SG-door fix DUR WAN\\
        `I bought a house. I need to fix \textbf{the door}.' (Lit. `...fix its door.)

    \exg. mɔmɔmina t͡ʃene sã. a-m-be a-da ɲaan tea wǎ.\\
        everyone house buy 3-PL-NPST 3.SG-door fix DUR WAN\\
        `Everyone who bought a house needs to fix \textbf{the door}.' (Lit. `Everyone bought a house (every > a). They need to fix its door (co-variation reading).')

    \mb{\textit{tea} is glossed as an unspecified durative term here. Anthony: `tea doesn't stand on its own... here just means the process (of fixing) itself.}

\end{itemize}


\section{Long-Distance Questions - Joey}

\jf{Kinds of questions that have wh-initially:\\
-Subject of unaccusative (could just be in SpecTP)\\
-Subject of unergative (could just be SpecTP)\\
-Subject of transitive (could just be SpecTP)\\
-Based on the above examples, all of the wh-first questions we have seen an be explained by the wh-word being the grammatical subject and sitting in SpecTP.\\ 
-Possible wh-expletive/wh-scope-marking construction (`what did Julie say what did Bondu buy?')}\\


\exg. d͡ʒuli á feŋ fɔ-je bondu a fina-minã sa?\\
Julie 3SG what say-? Bondu 3SG.PST thing-which buy\\
`What thing did Julie say that Bondu bought?'\\ 

\jf{Verify the following question to hopefully prime questions that follow this pattern:}\\

\exg. d͡ʒuli a feŋ fɔ-je bondu a feŋ sa?\\
Julie 3SG what say-? Bondu 3SG.PST what buy\\
`What did Julie say that Bondu bought?'\\ 


\jf{Using Kono names and adding "yesterday":}\\

\exg. kai a feŋ fɔ-je bondu a feŋ sa-n kunu?\\
Kai 3SG what say-? Bondu 3SG.PST what buy-? yesterday\\
`What did Kai say that Bondu bought yesterday?' \\ 

\jf{It's not entirely clear to me why the "n" got inserted after buy above. Perhaps epenthetic? It is also not entirely clear what role the "-je" morpheme plays, but it seems necessary}\\

\jf{Replacing object "what" with object "which meat" (Context: you're wondering if it's beef, or pork, or chicken, etc.):}\\
\exg. swee-mina-mu kai ɔ te bondu a sa kunu\\
meat-which-? Kai OBJ say Bondu 3SG.PST buy yesterday\\
`Which meat did Kai say that Bondu bought yesterday?'\\ 

\jf{It's not entirely clear what this "mu" morpheme is, but we seem to have seen it in 03/05, ex. 32, and in the love series from 3/19. The ɔ may be object agreement with the embedded CP, perhaps it's combined with the subject agreement (presumably something like a-ɔ)?}\\

\jf{There is fronting in the above.}\\

\jf{Replacing object "which meat" with subject "who":}\\
\exg. kai-ja fɔ ɔ ɲo-na swee sa kunu\\
kai-3SG say 3SG.ɔSER who-FOC meat buy yesterday\\
`Who did Kai say bought meat yesterday?'\\ 

\exg. kai ɔ-te ɲo-na swee sa kunu\\
kai 3SG.ɔSER-say  who-FOC meat buy yesterday\\
`Who did Kai say bought meat yesterday?'\\ 

\jf{Assuming that the say is te and not ɔte, then we have object agreement with the CP in the second but not the first sentence above. Additionally, we have no subject agreement in the second sentence. It is unclear to me what the object agreement is doing in the first sentence, unless perhaps it's postverbal object CP agreement}\\

\exg. \ *kai-ja ɲo-na fɔ-je ɲo-na swee sa kunu\\
kai-3SG who-FOC say-? who-FOC meat buy yesterday\\
`Who did Kai say bought meat yesterday?'\\ 

\jf{The above attempt at a wh-expletive was not possible (Although, possibly I should try with just "what."}\\


\jf{When:}\\
\exg. te-mina-na-mu kai ɔ-te bondu a swee sa\\
time-which-FOC(?)-? Kai 3SG.ɔSER-say Bondu 3SG.PST meat buy\\
`When did Kai say that Bondu bought meat?'\\ 

\jf{Above example is fronted}\\

\jf{Why:}\\
\exg. kai (a) ɔ-te bondu a swee sa kunu asã feŋ kɔa?\\
Kai (3SG) 3SG.ɔSER-say Bondu 3SG.PST meat buy yesterday for(?) what reason(?)\\
`Why did Kai say that Bondu bought meat yesterday?'\\ 

\jf{It sounded in one of the instances of the sentences that he had the subject agreement precede the object agreement overtly. asã feŋ kɔa means "for what purpose," but exactly how to gloss it wasn't entirely clear.}\\

\jf{Base question (subject who):}\\
\exg. Kai ɔ-te ɲona bondu ie kunu\\
Kai 3SG.ɔSER-say who Bondu see yesterday\\
`Who did Kai say saw Bondu yesterday?'\\ 


\jf{Replacing subject "who" with object "who":}\\
\exg. Kai ɔ-te bondu a ɲond͡ʒ ie kunu\\
Kai 3SG.ɔSER-say Bondu 3SG.PST who see yesterday\\
`Who did Kai say Bondu saw yesterday?'\\ 

\jf{The previous pair of sentences suggests that there's no expletive construction for "who," regardless of whether it is the embedded subject or object}\\

\jf{Replacing object "who" with subject "what" (Context: You heard that some strange animal saw Bondu yesterday):}\\
\exg. kai ɔ-te fen-a bondu ee kunu   \\
Kai  3SG.ɔSER-say what-3SG.PST Bondu see yesterday\\
`What did Kai say saw Bondu yesterday?'\\ 

\exg. \ *kàì fèn ɔ̀-té fèn á bòndú èè kùnù   \\
Kai what 3SG.ɔSER-say what-3SG.PST Bondu see yesterday\\
`What did Kai say saw Bondu yesterday?'\\ 


\jf{Doesn't seem possible to use expletive wh with "te"}\\

\exg. kai-ja fɔ fen-a bondu ee kunu   \\
Kai-3SG say what-3SG.PST Bondu see yesterday\\
`What did Kai say saw Bondu yesterday?'\\ 

\jf{An interesting observation is that there is no object agreement for the CP with fɔ (except possibly in that one of "who did Kai say bought meat yesterday)}\\


\exg. kàì á fèɱ fɔ̀-ì-jé fèn á bòndú èè kùnù   \\
Kai 3SG.PST what say-2SG-to what 3SG.PST Bondu see yesterday\\
`What did Kai say saw Bondu yesterday?'\\ 

\exg. kai-ja feŋ fɔ-*(je) fen-a bondu ee kunu   \\
Kai-3SG what say-? what-3SG.PST Bondu see yesterday\\
`What did Kai say saw Bondu yesterday?'\\ 

\exg. kai-ja feŋ fɔ-*(je) fena-mina-na bondu ee kunu   \\
Kai-3SG what say-? thing-REL-NA Bondu see yesterday\\
`What did Kai say saw Bondu yesterday?'\\ 

\exg. kàì á fèɱ fɔ̀-ì-jé fènè-mìná-nà bòndú èè kùnù\\
Kai 3SG.PST what say-2SG-to thing-REL.SBJ-FOC Bondu see yesterday\\
`Which thing did Kai say saw Bondu yesterday?'\\ 

\jf{Anthony mentions that a comma should be inserted after "-je." Additionally, he mentioned that the "-je" is necessary and can't be removed when we have the wh-expletive, which begs the question as to what this "-je" is. These examples also show that wh-expletives can work with embedded subjects as well as objects.}\\ 


\section{Glides - Jan}

\exg. kóà \\
monkey\\
`monkey'

\exg. kóá wá \\
monkey big \\
`big monkey'

\exg. kɔ́wá \\
shoe\\
`shoe'
    
\exg. kɔ́á wá \\
shoe big \\
`big shoe'

\exg. úú \\
dog\\
`dog'

\exg. úú swè \\
dog meat\\
`dog meat'

\exg. úú wá \\
dog big \\
`big dog' 

\exg. úú íí jò\\
dog water in \\
`The dog is in the water.'

\exg. úú wíí à \\
dog boil A \\
`The dog is boiling.'

\exg. íí wíí à \\
water boil A \\
`The water is boiling.'

\exg. bóndú íí à wíí \\
bondu water OBJ boil\\
`Bondu, boil the water.'

\exg.  bóndú ʔ mó íí à wíí \\
bondu - 2PL.INC water A boil \\
`Bondu, let's boil the water.'

\exg. jay íí à wíí\\
Jai water A boil \\
`Jai, boil the water.'

\jmt{When a glottal stop separates `jai' and `water', it is as expected: [jaj íí à wíí]. But it seems that [y] surfaces when he says the sentence together.}

\exg. jáj mó íí à wíí \\
Jai 2PL.INC water A boil \\
`Jai, let's boil the water.'

\exg. íí dìç\\
water sweet \\
`Water is delicious.'

\jmt{This [ç] is interesting.}

\exg. ɔ́ɔ́lɔ́\\ 
six\\
`six'

\exg. úú wɔ́ɔ́lwè \\
dog six \\
`six dogs'

\exg. úú wɔ́ɔ́lwè m-bè-m-bòò \\
dog six 1SG.NPST-1SG-hand\\
`I have six dogs.'

\exg. úú wɛ́ m-bòò\\
dog big 1SG-hand\\
`I have a big dog.'

\exg. n-á éá ɔ́ɔ́lwèɪ í kìsì ò\\
1SG.PST peanut six see kitchen in\\
`I saw six peanuts in the kitchen.'

\exg. éjámbá ɔ́ɔ́lwè m-bòò\\
leaf six 1SG-hand \\
`I have six leaves.'

\exg. éjámbá éjáwà wɔ́ɔ́lwè m-bòò\\
leaf red six 1SG-hand \\
`I have six red leaves.'

\exg. éjáj   éjáwà wɔ́ɔ́lwè m-bòò\\
leaf red six 1SG-hand \\
`Jai has six red leaves.'

\exg. éágbásí\\
onion\\
`onion'

\exg. éábásí ɔ́ɔ́lwè\\
onion six \\
`six onions'

\exg. éábásí ɔ́ɔ́lwè m-bè-m-bòò \\
onion six 1SG-NPST-1SG-hand \\
`I have six onions. '

\exg. kàw \\
moon\\
`moon'

\exg. káw-ɔ́ɔ́lwè\\
moon-six\\
`six moons' (also sixth month)

\exg. káw-wa \\
moon-big \\
`big moon'

\section{Pronouns/auxiliaries - Alex}


\exg.
n-a    kaŋane-n   a     toŋ   \\
1SG.PST  door-PL    OBJ   close \\%
`I closed the doors.'

\exg.
n-a    sani     a     toŋ   \\
1SG.PST  bottle   OBJ   close \\%
`I closed the bottle.'

\exg.
n-a    sani-n      a     toŋ   \\
1SG.PST bottle-PL   OBJ   close \\%
`I closed the bottle.'

\exg.
n-á   á   toŋ \\
1SG.PST OBJ close \\%
`I closed it.'

\exg.
n-á    á-n-á       toŋ   \\
1SG.PST 3-PL-PST   close \\%
`I closed them.'

\exg.
mbáá              tond-a    waN \\
1SG.NPST.3SG.OBJ   close-A   FOC \\%
`I will close it'

\exg.
mb-ánà            tond-a    waN \\
1SG-NPST.3PL.OBJ   close-A   FOC \\%
`I will close them'

% unaccusative

\exg.
tʃɛna-da      a     tond-a  \\
house-mouth   OBJ   close-A \\%
`The door closed.'

\exg.
kaŋana   a     tond-a  \\
door     OBJ   close-A \\%
`The door closed.' \label{The door closed}

\alex{\ref{The door closed} demonstrates /ɛ/ -> /a/ change (which will be useful for arguing for the presence of the AUX in non-past sentences specifically when we have object pronouns (which seem to override the vowel quality of the auxiliary)), but this indeed might be more complicated, as we see in later examples that /a/ will (possibly) change to /ɛ/..?}

\exg.
a     tond-a  \\
3SG.PST.OBJ   close-A \\%
`It closed.'

\exg.
a-n-a   tonda   \\
3-PL-PST   close-A \\%
`They closed.'

\exg.
a   ɛ   tond-a   waN \\
3SG NPST close-A FOC \\%
`It will close.' \label{It will close}

\exg.
a-n-ɛ       tond-a    waN \\
3-PL-NPST   close-A   FOC \\%
`They will close.' \label{They will close}

\alex{Note that the form of the 3PL pronoung (the surface subject) in \ref{They will close} is not \textit{ambe}. Perhaps this is evidence for a syntactic restriction on morphophonological conditioning we observe. My hypothesis is that since the subject starts low, as a complement to the verb, and undergoes A-movement to the surface subject position, it blocks to the kind of morphophonological change /N + ɛ/ -> mbɛ we observe elsewhere, i.e., when the subject is, say, an agent. One question that remains is whether there is an ``object'' marker in these unaccusative constructions. The recording seems to suggest that in the sentence \ref{They will close}, there is indeed a vowel preceding the verb. However, in \ref{It will close}, there does not seem to be a vowel preceding the verb.}

Context: The speaker is talking to the door.

\exg.
j-a    tond-a  \\
2SG-PST   close-A \\%
`You closed.'

\exg.
j-a    ɛ     tond-a    waN   sina     \\
2SG-OBJ NPST close-A   FOC   tomorrow \\%
`You will close tomorrow.'

Context: The door is speaking.

\exg.
n-a    tond-a  \\
1SG-OBJ   close-A \\%
`I closed.'

\exg.
n-a    ɛ     tond-a    waN   sina     \\
1SG-OBJ   NPST   close-A   FOC   tomorrow \\%
`I will close tomorrow.'

\exg.
tʃɛnɛ-nɛ   tond-a  \\
house-PL   close-A \\%
`The doors/houses are closing.'

\exg.
kaŋganɛ-n oŋgo   ton-da  \\
door-PL     ONGO    close-A \\%
`The doors are closing'

\exg.
a-n-ɔŋɡɔ    tond-a  \\
3-PL-ONGO   close-A \\%
`They are closing'

\exg.
ɛŋɡɔ       tond-a  \\
3SG.ONGO   close-A \\%
`It is closing'

\exg.
n-ɔ        ɔ     te      kunu      \\
1SG.PST.3SG   OBJ   break   yesterday \\%
`I broke it yesterday.' \label{I broke it yesterday}

\alex{In \ref{I broke it yesterday}, we see the vowel /ɔ/ overriding the /a/ vowel for the 3SG object pronoun (which is distinct from the object marker), in addition to the /a/ vowel on the 1SG grammatical subject (arguably some Tense morpheme).}

\exg.
n-a    a-n.dɔ     te      kunu      \\
1SG.PST  3-PL.OBJ   break   yesterday \\%
`I broke them yesterday.'

\exg.
m-b-ɔɔ              te-a      waN   sina     \\
1SG-NPST-OBJ   break-A   FOC   tomorrow \\%
`I will break it tomorrow.'

\exg.
m-ba-a-n-dɔ           te-a      waN   sina     \\
1SG-NPST-3-PL-OBJ   break-A   FOC   tomorrow \\%
`I will break them tomorrow.'

\exg.
n-ɔŋɡɔ     ɔ     te-a    \\
1SG-ONGO   OBJ   break-A \\%
`I am breaking it.'

\exg.
n-ɔŋɡɔ     andɔ      te-a    \\
1SG-ONGO   3PL.OBJ   break-A \\%
`I am breaking them.'

\exg.
ɛnɡɔ       ɔ     te-a    \\
2SG.ONGO   OBJ   break-A \\%
`You are breaking it.'

\exg.
ɛnɡɔ       andɔ      te-a    \\
2SG.ONGO   3PL.OBJ   break-A \\%
`You are breaking them.'

\exg.
n-ɔŋɡɔ     taa.n.dɔ          te-a    \\
1SG-ONGO   calabash.PL.OBJ   break-A \\%
`I am breaking the calabashes.'

\exg.
n-ɔŋɡɔ     taa        ɔ     te-a    \\
1SG-ONGO   calabash   OBJ   break-A \\%
`I am breaking the calabash.'

\exg.
ni    taa        ɔ     te-a      kunu      \\
1SG   calabash   OBJ   break-A   yesterday \\%
`I was breaking the calabash yesterday...'

\exg.
ani   taa-n-dɔ          te-a      kunu,       mbɛ    sa   ana     \\
3PL   calabash-PL-OBJ   break-A   yesterday   then   Sa   arrived \\%
`I was breaking the calabash yesterday, then Sa arrived.'


\alex{Note we have what looks like another auxiliary form that affects/combines with the pronominal subject forms.}



\end{document}