\documentclass{assets/fieldnotes}


\title{Kono (Sierra Leone)}
\author{LING3020/5020}
\date{University of Pennsylvania, Spring 2025\\02/19/2025 Morphology II}

\setcounter{secnumdepth}{4} %enable \paragraph -- for subsubsubsections

\begin{document}

\maketitle
\tableofcontents

\jal{think about simple sentences, topics we've learned are likely naturally spoken of in Kono, coordinate with each other; try pronominal and simple nominal subjects   }

9\section{Nonverbal predication: AdjPs (Wesley)}

\wml{
\begin{itemize}
    \item Seems like emotion-related stage-level adjectives are expressed using body-part idioms, while type-level adjectives are actual adjectival predicates.
    \item Adjectival predicates seem to show morphological subject marking similar to the inalienable possessive series, although I only managed to elicit third-person items with nominal subjects.
    \item Attributive adjectives seem to be marked by -ma, while predicative ones do not bear this marker, only the subject marker.
\end{itemize}


\exg. ̩ḿ̩-bwó dí\\
\textsc{1sg}-body good\\
`I am happy.'

\exg. ̩ḿ̩-bwó fâ énè\\
\textsc{1sg}-body dead ?\\
`I am tired.'

\exg. ɱ̩́-fáà jón déɛ̀\\
\textsc{1sg}-heart poor ?\\
`I am sad.'

\exg. ̩ḿ̩-bwó má-dí\\
\textsc{1sg}-body \textsc{neg}-good\\
`I am not happy.'

\exg. mwókàmá tʃɛ̀ à-bwó má-dí\\
man this \textsc{3sg}-body \textsc{neg}-good\\
`This man is not happy.'

\exg. mwókàmá tʃɛ̀ à-fáà jón déɛ̀\\
man this \textsc{3sg}-heart poor ?\\
`This man is sad.'

\exg. mùsù tʃɛ̀ à-bwó má-dí\\
woman this \textsc{3sg}-body \textsc{neg}-good\\
`This woman is not happy.'

\exg. mwókàmá tʃɛ́-nù à-ɱ-fáà jón-déɛ̀\\
man this-\textsc{pl} \textsc{3-PL}-heart poor-?\\
`These men are sad.'

\exg. mwókàmá tʃẽ̀ĩ à-ɱ-fáà jón-déɛ̀\\
man these \textsc{3-PL}-heart poor-?\\
`These men are sad.'

\exg. mùsú tʃɛ́-nù à-M-bwó má-dì\\
woman this-\textsc{pl} \textsc{3-PL}-body \textsc{neg}-good\\
`These women are not good.'

\exg. mùsú tʃɛ́-nù àɱ-fáà jón déɛ̀\\
woman this-\textsc{pl} \textsc{3pl}-heart poor ?\\
`These women are sad.'

\exg. mwókàmá ɱ̩̀-fáà jón-déɛ̀\\
man \textsc{3pl}-heart poor-?\\
`(Some) men are sad.'

\exg. mùsú ɱ̩̀-fáà jón-déɛ̀\\
woman \textsc{3pl}-heart poor-?\\
`(Some) women are sad.'\\
\wml{Possible to express `women' with a bare noun plus a nasal prefix on the verb, perhaps a reduced form of the 3pl àɱ.}

\exg. mùsù jã̀sàmá tʃɛ̀\\
woman tall this\\
`this tall woman’

\exg. mùsù tʃɛ̀ jã̀sã̀\\
woman this tall\\
`This woman is tall.’

\exg. mùsù tʃɛ́-nù ã̀-jã̀sà\\
woman this-\textscpl{pl} \textsc{3pl}-tall\\
`These women are tall.'

\wml{Distinguishing predicative vs. attributive AdjPs}

\exg. mùsù jã̀sàmá tʃɛ̀ á-ɲí\\
woman tall this \textsc{3sg}-beautiful\\
`This tall woman is beautiful.'

\exg. mùsù ɲímá tʃɛ̀ á-jã̀sã̀\\
woman beautiful this \textsc{3sg}-tall\\
`This beautiful woman is tall.'

\wml{Rising intonation at the end of the DP---is that a hanging topic?}



\section{Nonverbal predication: PPs (Mingyang)} 
\exg. tʃénè\\
    house\\
    `house'

\exg. ḿ-bé tʃén-à\\
    1SG-NPST house-P\\
    `I am in the house.'

\mb{It might be the case that the postposition for `in' -a comes after tʃénè, so the result is tʃénà.}
   
\exg. é tʃén-à\\
    2SG.NPST house-P\\
    `You(sg.) are in the house.'

\exg. w-é tʃén-à\\
    2PL-NPST house-P\\
    `You(pl.) are in the house.'

\exg. ɛ̀ tʃén-à\\
    3SG.NPST house-P\\
    `She/he is in the house.'

\exg. à-m-bè tʃén-à\\
    3-PL-NPST house-P\\
    `They are in the house.'

\exg. bèé tʃénà\\
    1.PL.EXCL house\\
    `We(excl., du./pl.) are in the house.'

\exg. mw-ɛ̃̀  tʃénà\\
    1INCL.DU-NPST house\\
    `We(incl., du.) are in the house.'

\exg. mw-ɛ̃̂  tʃénà\\
    1INCL-NPST house\\
    `We(incl., pl.) are in the house.'

\exg. jàí tʃénà\\
    Jai house\\
    `Jai is in the house.'


\exg. kàâ\\
    snake\\
    `snake'

\exg. sènê\\
    stone\\
    `stone'

\exg. kàâ ɛ̀ sènê mà.\\
    snake 3SG.NPST stone on\\
    `The snake is on the stone.'

\mb{I suspect that there is a morpheme è for subjects (not sure if it's only for animate subjects), and kàâ è sounds like kàê.}

\exg. kàâ ɛ̀ sènê kɔ́ɔ́.\\
    snake 3SG.NPST stone under\\
    `The snake is under the stone.'

\exg. sènê kàâ kɔ́ɔ́.\\
    stone snake under\\
    `The stone is under the snake.'

\exg. kàâ ɛ̀ sènê jɔ̂.\\
    snake 3SG.NPST stone in.front.of\\
    `The snake is in front of the stone.'

\exg. kàâ ɛ̀ sènê bà.\\
    snake 3SG.NPST stone behind\\
    `The snake is behind the stone.'

Pronouns for `in the house':
\begin{center}
    \begin{tabular}{|c|c|c|c|}
    \hline
    Person/Number & SG & DU & PL \\ \hline
    1 & mbé & mùè(incl.) & mùê(incl.)/bèé(excl.) \\ \hline
    2 & éé & N/A & wéé \\ \hline
    3 & ɛ̀ & N/A & àmbè \\ \hline
    \end{tabular}
\end{center}
Prepositions:
\begin{center}
    \begin{tabular}{|c|c|}
    \hline
    on & mà\\ \hline
    under & kɔ́ɔ́\\ \hline
    in front of & jɔ̂\\ \hline
    behind & bà\\ \hline
    \end{tabular}
\end{center}

\section{Nonverbal predication: NPs (Daniel)}
\exg. kámw\textipa{\`E}\\
teacher\\
`teacher'

\exg. kámw\textipa{\`E} mu n-á\\
teacher COP 1SG-POSS\\
`I am a teacher'
\ds{Note order, pred-first -- is the morpheme at the end agreement or pron?}

\exg. kámw\textipa{\`E} mu j-á\\
teacher COP 2SG-POSS\\
`You (sg.) are a teacher'

\exg. kámw\textipa{\`E} mu w-á\\
teacher COP 2PL-POSS\\
`You (pl.) are teachers'

\exg. kámw\textipa{\`E} mu ná-à\\
teacher COP 1EXCL-POSS\\
`We (dual excl.) are teachers'

\exg. kámw\textipa{\`E} mu mwa\\
teacher COP 1PL.EXCL\\
`We (pl excl.) are teachers'


\exg. kà-mó\\
teach-person\\
`teacher’\\

\exg. Kai mu kámw\textipa{\`E\`E} na\\
Kai COP teacher-? 3SG?\\
`Kai is a teacher'
\ds{*Kai kámw\textipa{\`E\`E} na mu -- copula cannot be postposed}\\
\ds{Lengthening of predicate?}

\exg. Kámw\textipa{\`E\`E} mu Kai-à\\
teacher-? COP Kai-3SG.POSS\\
`Kai is a teacher' (unclear if this is equivalent)
\ds{*Kai kamw\textipa{\`E\`E}(na) -- copula cannot be dropped}


\exg. Kai mu kámw\textipa{\`E} \textipa{\textltailn}imá na\\
Kai COP teacher good 3SG?\\
`Kai is a good teacher'

\exg. táá\\
calabash\\
`calabash'

\exg. táá mu dawn-fen-àà na\\
calabash COP food-thing-? 3SG?\\
`A calabash is food'

\exg. táá mu dawn-f\textipa{\~E} \textipa{tS\~edE} na\\
calabash COP food-thing good 3SG?\\
`A calabash is good food'


\exg. táá mu dawn-f\textipa{\~E} \textipa{\textltailn}imá ná\\
calabash COP food-thing good 3SG?\\
`A calabash is good food'

\exg.  táá \textipa{tSe-nu} mu dawn-f\textipa{\~E} \textipa{tS\~edE} ana\\
calabashes this-PL COP food-thing good 3PL\\
`These calabashes are good food' (?)

\exg. táá-nu mu dawn-f\textipa{\~E} \textipa{tS\~edE} ana\\
calabashes COP food-thing good 3PL\\
`Calabashes are good food' (?)



\section{Intransitive verbal predicates: Past (Chun-Hung)}

\jal{use an adverb (e.g. yesterday) to make sure you're getting the temporal interpretation you want}

\chs{unergatives (others: run/walk, swim, sneeze, sing ...)}

\exg. K\`{a}\`{i} \`{a} d\`{i}-t\textipa{S}\`{\textipa{E}} k\`{u}n\`{u}. \\
Kai 3SG.PST cry-? yesterday \\
`Kai cried yesterday.' \chs{di with a slightly rising tone?}
 
\exg. D\'{e}n\'{e}-t\textipa{S}\`{\textipa{E}} \`{a} d\`{i}-t\textipa{S}\`{\textipa{E}} k\`{u}n\`{u}. \\
child-DEM 3SG.PST cry-? yesterday \\
`This child cried yesterday.'

\exg. N-\'{a} d\`{i}-t\textipa{S}\`{\textipa{E}} k\`{u}n\`{u}. \\
1SG.PST cry-? yesterday \\
`I cried yesterday.'

\exg. J-\'{a} d\`{i}-t\textipa{S}\`{\textipa{E}} k\`{u}n\`{u}. \\
2SG-PST cry-? yesterday \\
`You (sg.) cried yesterday.' 

\exg. \`{A} d\`{i}-t\textipa{S}\`{\textipa{E}} k\`{u}n\`{u}. \\
3SG.PST cry-? yesterday \\
`He/She cried yesterday.'

\exg. N\`{a}-\'{a} d\`{i}\'{e}n-t\textipa{S}\`{\textipa{E}} k\`{u}n\`{u}. \\
1EXCL-PST cry-? yesterday \\
`We (exclusive, dual/plural) cried yesterday.'

\exg. M\`{ɔ}\'{a} d\`{i}\'{e}n-t\textipa{S}\`{\textipa{E}} k\`{u}n\`{u}. \\
1INCL.DU-PST cry-? yesterday \\
`We (inclusive, dual) cried yesterday.'

\exg. M\'{ɔ}\'{a} d\`{i}\'{e}n-t\textipa{S}\`{\textipa{E}} k\`{u}n\`{u}. \\
1INCL-PST cry-? yesterday \\
`We (inclusive, plural) cried yesterday.'

\exg. W\'{a} d\`{i}\'{e}n-t\textipa{S}\`{\textipa{E}} k\`{u}n\`{u}. \\
2PL-PST cry-? yesterday \\
`You (pl.) cried yesterday.' 

\exg. \`{A}-n-\'{a} d\`{i}\'{e}n-t\textipa{S}\`{\textipa{E}} k\`{u}n\`{u}. \\
3-PL-PST cry-? yesterday \\
`They cried yesterday.'

\chs{unaccusative (others: come, leave, die ...)}
\jal{arrive/come/leave depend on the language having these as directed motion verbs, not all languages do. Since we don't have much time and die is a bit morbid, perhaps try fall?}

\exg. K\`{a}\`{i} b\'{e}\`{a} k\`{u}n\`{u}. \\
Kai fall yesterday \\
`Kai fell yesterday.'

\exg. D\'{e}n\'{i}n-t\textipa{S}\`{\textipa{E}} b\'{e}\`{a} k\`{u}n\`{u}. \\
child-DEM fall yesterday \\
`This child fell yesterday.'

\exg. \'{M} b\'{e}\`{a} k\`{u}n\`{u}. \\
1SG fall yesterday \\
`I fell yesterday.' \chs{*N\'{a} for 1SG cannot be used here.}

\exg. \'{I} b\'{e}\`{a} k\`{u}n\`{u}. \\
2SG fall yesterday \\
`You (sg.) fell yesterday.' \chs{*J\'{a} for 2SG cannot be used here.}

\exg. \`{A} b\`{e}\'{a} k\`{u}n\`{u}. \\
3SG fall yesterday \\
`He/She fell yesterday.'

\exg. \`{M} b\`{e}\'{a} k\`{u}n\`{u}. \\
1EXCL fall yesterday \\
`We (exclusive, dual/plural) fell yesterday.'

\exg. M\`{o} b\'{e}\`{a} k\`{u}n\`{u}. \\
1INCL.DU fall yesterday \\
`We (inclusive, dual) fell yesterday.'

\exg. M\'{o} b\'{e}\`{a} k\`{u}n\`{u}. \\
1INCL fall yesterday \\
`We (inclusive, plural) fell yesterday.'

\exg. W\'{o} b\'{e}\`{a} k\`{u}n\`{u}. \\
2PL fall yesterday \\
`You (pl.) fell yesterday.'

\exg. Ã́ b\'{e}\`{a} k\`{u}n\`{u}. \\
3PL fall yesterday \\
`They fell yesterday.' 

\section{Transitive verbal predicates: Past (Joey)} 

\exg. dàó̃\\
    eat\\
    `eat'

\exg. dàùnɛ̂\\
    eat\\
    `eat'


\exg. \textipa{sw\'e\`e} \\
meat\\
`meat'
  

\jf{When taken with data from the previous section, there appears to be a split-S ergative/absolutive system}\\

\exg. n-á swéè dàó̃ kùnù.\\
1SG.PST meat eat yesterday \\%
    `I ate meat yesterday'

\exg. j-á swéè dàó̃ kùnù\\
2SG-PST meat eat yesterday \\%
    `You (sg.) ate meat yesterday'

\exg. á swéè dàó̃ kùnù\\
3SG.PST meat eat yesterday \\%
    `He/She ate meat yesterday'

\exg. kàì á swéè dàó̃ kùnù\\
Kai AUX.PST meat eat yesterday \\%
    `Kai ate meat yesterday'

\exg. nà-á swéè dàó̃ kùnù\\
1EXCL-PST meat eat yesterday \\
    `We (exclusive, pl.) ate meat yesterday'

\exg. nà-á swéè dàó̃ kùnù\\
1EXCL-PST meat eat yesterday\\
    `We (exclusive, dual) ate meat yesterday'
\jf{Same as exclusive, pl. form}

\exg. múá swéè dàó̃ kùnù\\
1DU.INCL meat eat yesterday \\
    `We (inclusive, dual) ate meat yesterday'

\exg. múà swéè dàó̃ kùnù\\
1PL.INCL meat eat yesterday \\
    `We (inclusive, pl.) ate meat yesterday'

\exg. w-á swéè dàó̃ kùnù\\
2SG-PST meat eat yesterday \\
    `You (pl.) ate meat yesterday'

\exg. à-n-á swéè dàó̃ kùnù\\
3-PL-PST meat eat yesterday \\
    `They ate meat yesterday'

\exg. sàa ɱfea kàìjâ à-n-á swéè dàó̃ kùnù\\
Saa and Kai.? 3-PL-PST meat eat yesterday \\
    `Saa and Kai ate meat yesterday'

\exg. íễː\\
see\\
    `see'

\exg. n-á í íễː kùnù\\
1SG.PST 2SG see yesterday \\
    `I saw you (sg.) yesterday'

\exg. n-á à íễː kùnù\\
1SG.PST 3SG see yesterday \\
    `I saw him yesterday'

\exg. n-á sàà à íễː kùnù\\
1SG.PST Saa 3SG see yesterday \\
    `I saw Saa yesterday'

\exg. n-á wó íễː kùnù \jf{(When said quickly, can sound like ná wé íễː kùnù)}\\
1SG.PST 2PL see yesterday \\
    `I saw you (pl) yesterday'

\exg. n-á á̃ íễː kùnù \jf{(When said quickly, can sound like ná  ánd͡ʒ íễː kùnù)}\\
1SG.PST 3PL see yesterday \\
    `I saw them yesterday'

\exg. n-á sàà nì kàì̃ íễː kùnù\\
1SG.PST Saa and Kai.3PL see yesterday \\
    `I saw Saa and Kai yesterday'

\exg. n-á sàà ɱfea kàì̃ íễː kùnù\\
1SG.PST Saa and Kai.3PL see yesterday \\
    `I saw Saa and Kai yesterday'



\section{Nonverbal predication: Past (Lex)}%Lex


\exg. ʃopi\\
store\\
`store'

\exg. ʃopjɔ̀\\
at the store\\
`at the store'
\jf{ Sometimes Anthony pronounced it sopjɔ̀}

\exg. kùnù\\
yesterday\\
`yesterday'


\exg. m-fààyondɛ̀ níkùnù, m-bwɔ̀nímádìkùnù\\
1SG was sad yesterday\\
`I was sad, yesterday.'

\exg. m-fààyondɛ̀ níkùnù ʃopjɔ̀ \\
1SG was sad yesterday in the store\\
I was sad in the store, yesterday.\chs{( ʃopjɔ̀kùnù, m-fààyondɛ̀ní ) means the same thing}

\exg. m-fààyondɛ̀ níkùnùtʃɛ̀ná \\
1SG. was sad yesterday in the house\\
`I was sad in the house, yesterday.'
\chs{( tʃɛ̀nákùnù, m-fàà yondɛ̀ní ) means the same thing}

\exg. í-fààyondɛ̀ níkùnù\\
2SG. were sad yesterday\\
`You (sg.) were sad, yesterday.'

\exg. á-fààyondɛ̀ níkùnù\\
3SG. was sad yesterday\\
 ` He was sad, yesterday.'
 
\exg. á-fààyondɛ̀ níkùnù\\
3SG. was sad yesterday\\
`She was sad, yesterday'

\exg. áŋ-fààyondɛ̀ níkùnù\\
3PL. were sad yesterday\\
`They were sad, yesterday'

\exg. mm-fààyondɛ̀ níkùnù\\ 
1DU.EXCL. were sad yesterday\\
`We (\textsc{du.excl}) were sad, yesterday' 

\exg. mɔ-fààyondɛ̀ níkùnù\\
1DU.INCL. were sad yesterday\\
 `We (\textsc{du.incl}) were sad, yesterday' 

\exg. mɔ́-fààyondɛ̀ níkùnù\\
1INCL. were sad yesterday\\
 `We (\textsc{Pl.incl}) were sad, yesterday'


\exg. bànù\\
last year\\
`last year'



\section{Nonverbal predication: Future (Jan)} 

\jal{it's useful to have minimal pairs with other tenses; build on what we have. -- so use what other people have gotten before you  Use adverbs like "tomorrow" etc to help the temporal interpretation.}

\exg. kúíì \\
leopard\\
`leopard' 


\exg. kúíì cɛ̀ \\
leopard this \\
`this leopard'

\exg. sáándʒé \\
soon \\
`soon' (immediate)

\exg. flòflò \\
soon\\
`soon' (immediate)

\exg. kwíì tʃè koáà wà \\
leopard this big? FUT \\
`This leopard will be big.'

\exg. kwì tʃè kʷóáá wà sáándʒé \\
leopard this big? FUT soon \\
`This leopard will be big soon.'

\exg. kwì tʃɛ̀ ɛ́ dó éàwà \\
leopard this 3SG? small FUT? \\
`This leopard will be small.'

\exg. sáán tɛ̀mà pwì tʃà \\
Sahr POSS.FUT leopard this \\
`This leopard will be Sahr's.'

\exg. sàán tàmù \\
Sahr POSS-COP \\
`This is Sahr's' (certainty?)

\exg. sínà \\
tomorrow\\
`tomorrow' 

\exg. kwi tʃè-è màwàŋ kònè kóò sìnà \\
leopard this-? LOC.FUT tree under tomorrow \\
`This leopard will be under the tree tomorrow.' %\jal{doesn't strike me as very natural? think about a situation where you'd want to say something}

\exg. ii \\
river\\
`river'

\exg. kwí tʃè-è màwà íì fè sìnà  \\
leopard this-? LOC.FUT river beside tomorrow \\
`This leopard will be by the river tomorrow.'

\exg. kwí tʃè màwà nè sìnà \\
leopard this LOC.FUT here tomorrow \\ 
`This leopard will be here tomorrow.'


\exg. kwí tʃè dàònfènà són dá wà sìnà \\
leopard this food POSS? POSS FUT tomorrow \\
`This leopard will have food tomorrow.'


\exg. m-bátʃɛ̀nɛ́ \\
neighbor\\
`neighbor'

\exg. káé màwà mbátʃènà \\
Kai POSS-FUT neighbor \\
`Kai will be my neighbor.'

\exg. gbándìmá \\
soon\\
`soon' (not immediate)

\exg. káé màwà mbátʃènà àgbándìmá\\
Kai POSS-FUT neighbor soon \\
`Kai will be my neighbor soon.'

\jmt{Here, Anthony noted that using [sáándʒé] would be incorrect, since the `immediate' meaning does not fit.}

\exg. káé màwà sáà batʃènà \\
Kai POSS.FUT Sahr neighbor \\
`Kai will be Sahr's neighbor.'

\exg. káé bàtʃènè mà són dà-wà \\
Kai neighbor POSS POSS POSS-FUT \\
`Kai will have a neighbor.'


\section{Verbal predication: Progressive (Alex)} %Alex

\exg.
ḿ-bé        béjà \\
1SG.NPST   fall \\%
`I am falling.'

\exg.
é         béjà \\
2SG.NPST   fall \\%
`You(sg) are falling.'

\exg.
ɛ̀         bèjà \\
3SG.NPST   fall \\%
`He/she is falling.'

\exg.
mw-ɛ̃̀    bèjà \\
1INCL.DU-NPST   fall \\%
`We(dual) are falling.'

\exg.
mw-ɛ̃̂         bèjà \\
1INCL-NPST  fall \\%
`We(incl) are falling.'

\exg.
m-bè         bèjà \\
1EXCL-NPST   fall \\%
`We(excl) are falling.' \label{We(excl) are falling}

\alex{not sure about the tone on \ref{We(excl) are falling}.}

\exg.
w-é        béjà \\
2PL-NPST   fall \\%
`You(pl) are falling.'

\exg.
à-m-bè      bèjà \\
3-PL-NPST   fall \\%
`They are falling.'

\exg.
n-a    jaɡbasi   a     beja \\
1SG.PST   onion     OBJ   fall \\%
`I dropped the onion.'

\exg.
ḿ-bé      jaɡbasi   a     beja \\
1SG.NPST   onion     OBJ   fall \\%
`I am dropping the onion.'

\exg.
ḿ-bé      jaɡbasi   a     beja   waN   timotima   \\
1SG.NPST   onion     OBJ   fall   FOC   every.time \\%
`I always drop the onion.' \label{I always drop the onion}

\exg.
timotima     ḿ-bé       jaɡbasi   a     beja   waN \\
every.time   1SG.NPST   onion     OBJ   fall   FOC \\%
`I always drop the onion.'

\alex{The non-past/progressive/imperfective form is also used to express habituals, as shown by \ref{I always drop the onion}.}

\alex{Nice unaccusative/transitive (inchoative/causative) minimal pair with \textit{beja} `fall'.}

\alex{Object marker /a/ shows up in both past and non-past clauses}

\exg.
é         jaɡbasi   a     beja \\
2SG.NPST   onion     OBJ   fall \\%
`You(sg) are dropping the onion.'

\exg.
ɛ̀         jaɡbasi   a     beja \\
3SG.NPST   onion     OBJ   fall \\%
`He/she is dropping the onion.'

\exg.
a     jaɡbasi   tʃɛ-n-a       beja \\
3SG.PST   onion     this-PL-OBJ   fall \\%
`He/she dropped these onions.'

\exg.
ɛ̀         jaɡbasi   tʃɛ-n-a       beja \\
3SG.NPST   onion     this-PL-OBJ   fall \\%
`He/she is dropping these onions.'


\section{Verbal predication: Future (Giang)} %Giang


\g{Unaccusatives, most likely:}

\exg. báj.á.mà\\
Baiama\\
``Baiama''

\exg. Sàà é ná wá Bájámà sénà.\\
Saa ?(3SG.NPST??) come \textsc{fut} Baiama tomorrow\\
`Saa will arrive in Baiama tomorrow.' 
\g{is is e, does it carry tone?}


\exg. m-bé nà wà Bájámà sénà.\\
1\textsc{SG-NPST} come \textsc{fut} Baiama tomorrow\\
`I will arrive in Baiama tomorrow.'

\exg. éé nà wà Bájámà sénà.\\
You (2SG.NPST??) come \textsc{fut} Baiama tomorrow\\
`You will arrive in Baiama tomorrow.'

\exg. wéé nà wà Bájámà sénà.\\
2\textsc{PL.NPST} come \textsc{fut} Baiama tomorrow\\
`You(pl) will arrive in Baiama tomorrow.'

\exg. mw-è ná wà Bájámà sénà.\\
1.\textsc{INCL.DU-NPST}.\textsc{incl} come \textsc{fut} Baiama tomorrow\\
`We(dual, inclusive) will arrive in Baiama tomorrow.'

\exg. m-bé ná wà Bájámà sénà.\\
1.\textsc{EXCL-NPST}.\textsc{exl} come \textsc{fut} Baiama tomorrow\\
`We(dual, exclusive) will arrive in Baiama tomorrow.'

\exg. á-m-bè ná wà Bájámà sénà.\\
3-\textsc{PL-NPST} come \textsc{fut} Baiama tomorrow\\
`They will arrive in Baiama tomorrow.'


\g{Unergative}
\exg. tómbwè\\
to dance\\
`to dance'

\exg. íí-kuma\\
by the river\\
`by the river'

\exg. sànàmà bwódjè\\
New years\\
`New years'

\exg. Sàà é tómbwè dòndá-wà ííkùmà sénà\\
Saa (3SG.NPST)? dance ? by.the.river tomorrow\\
`Saa will dance by the river tomorrow.'

\exg. m-bé tómbwè dòndá-wà ííkùmà sénà\\
1\textsc{SG-NPST} ? dance ? by.the.river tomorrow\\
`I will dance by the river tomorrow.'
\g{Is dòndáwà another way to mark \textsc{fut}?}

\exg. mw-è tómbwé dòndá-wa ííkùmà sénà\\
1\textsc{INCL.DU-NPST} ? dance ? by.the.river tomorrow\\
`We(dual, incl) will dance by the river tomorrow.'


\g{Transitive}
\exg. fí.án.né\\
broom, sweep\\
`broom,sweep'
\g{né is nominalization.}

\exg. Sàà é cénà fíándà sénà\\
Saa ?(3SG.NPST) house sweep tomorrow\\
`Saa will sweep the house tomorrow.'
\g{house might be house+obj marker.}


\exg. é cénà fíándà sénà\\
2\textit{SG.NPST} house sweep tomorrow\\
`You will sweep the house tomorrow.'

\exg. é cénà fíándà wá sénà\\
2\textit{SG.NPST} house sweep have.to tomorrow\\
`You(sg) will have to sweep the house tomorrow.'

\exg. W-é cénà fíándà sénà\\
2\textit{PL-NPST} house sweep tomorrow\\
`You(pl) will sweep the house tomorrow.'

\exg. mw-è cénà fíándà sénà\\
1.\textit{INCL.DU-NPST} house sweep tomorrow\\
`We(dual, incl) will sweep the house tomorrow.'

\exg. m-bé cénà fíándà sénà\\
1.\textit{EXCL-NPST} house sweep tomorrow\\
`We(dual, incl) will sweep the house tomorrow.'

\exg. mw-ê cénà fíándà sénà\\
1.\textit{INCL-NPST} house sweep tomorrow\\
`We(pl, incl) will sweep the house tomorrow.'

\exg. á-m-bè cénà fíándà sénà\\
3-\textsc{PL.NPST} house sweep tomorrow\\
`They will sweep the house tomorrow.'

\end{document}}