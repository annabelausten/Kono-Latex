\documentclass{assets/fieldnotes}

\title{Kono (Sierra Leone)}
\author{LING3020/5020}
\date{University of Pennsylvania, Spring 2025\\05/14/2025 Class Projects Week 8}

\setcounter{secnumdepth}{4} %enable \paragraph -- for subsubsubsections

\begin{document}

\maketitle

\maketitle
\tableofcontents


\section{Joey}

\subsection{negative data elicitation:}

\exg. *bòndú  tàí ɔ́ té\\
Bondu story 3SG.OBJ say\\
`Bondu told the story' \jf{(Confirmed bad for standard past tense reading - interestingly, though, works as a command, but this took a long time after he forced meaning...)}

\subsection{different IO's with te:}

\exg. bòndú ɔ́ té kàì á swèè sàŋ kùnù\\
Bondu 3SG.OBJ say Kai 3SG.PST meat buy yesterday\\
`Bondu said that Kai bought meat yesterday' \jf{NO OPTION FOR IO}


\subsection{te with deixis:}

\exg. bòndú ɔ́ té n-á swèè sàŋ kùnù\\
Bondu 3SG.OBJ say 1SG.PST meat buy yesterday\\
`Bondu said that I bought meat yesterday' 

\jf{Clarify above who bought the meat, the speaker or Bondu - ie: is it quotative, or not?}

\exg. bòndú ɔ́ té á swèè sàŋ kùnù\\
Bondu 3SG.OBJ say 3SG.PST meat buy yesterday\\
`Bondu\textsubscript{1} said that she\textsubscript{1}  bought meat yesterday' 


\jf{(NOT EXCLUSIVELY QUOTATIVE, REFERENTS DON'T HAVE TO CHANGE)}

\subsection{Reconfirm:}

\exg. j-á fɔ̀ sɔ́jɔ́ bòndú á swèè sã̀ \\
2SG-PST say when Bondu 3SG.PST meat buy\\
`when did you say that Bondu bought meat?' \jf{(long-distance adjunct question }

\jf{If this is a Question, then answer:}\\

\exg. n-á fɔ̀-ì-jè bòndú á swèè sàŋ kùnù-fã̀\\
1SG.PST say-2SG-to Bondu 3SG.PST meat buy yesterday-FOC\\
`I told you that Bondu bought meat yesterday.'

\exg. sɔ́jɔ́ j-á fɔ̀ bòndú á swèè sã̀ \\
2SG-PST say when Bondu 3SG.PST meat buy\\
`when did you say that Bondu bought meat?' \jf{(Confirmed to work)}

\jf{If this is a Question, then answer:}\\

\exg. n-á fɔ̀-ì-jè bòndú á swèè sàŋ kùnù-fã̀\\
1SG.PST say-2SG-to Bondu 3SG.PST meat buy yesterday-FOC\\
I told you that Bondu bought meat yesterday. \jf{Same as above}


\exg. í-n-á fɔ̀ bòndú-ní ím-á fèn sã̀ -nì kùnù\\
C.Q(?)2SG-1SG.PST say Bondu-NI NEG-3SG.PST what buy-NEG yesterday\\
`You told me that Bondu did not anything yesterday.' \jf{NOT A Q}


\subsection{raising verb:}

\exg. m-wájɔ̀ bòndú á swèè-án dáò\\
1SG-seem Bondu 3SG meat-FOC eat \\
`I think/It seems to me that Bondu eats meat' \jf{Can't be used as a question}

\exg. í-í-mɔ̀ bòndú á fèn sã̀  kùnù \\
2SG-mind-in Bondu 3SG.PST what buy yesterday\\
`What did it seem that Bondu bought?' \jf{Lit: "In you're mind..."}

\subsection{Indirect Question with "te":}

\exg. kàì á fɔ̀-ì-jé bòndú á fèn sàŋ kùnù.\\
Kai 3SG.PST say-2SG-to Bondu 3SG.PST what buy yesterday\\
`Kai told you what Bondu bought yesterday' \jf{(For reference)}


\exg. kàì ɔ́ té bòndú á fèn sàŋ kùnù\\
Kai 3SG.OBJ say Bondu 3SG.PST what buy yesterday\\
`Kai said what did Bondu buy yesterday.' \jf{(With te - this is more quotative - not an indirect question)}

\subsection{Conjoined embedded questions (try with te and fo:}

\exg. bòndú á fèn d͡ʒé\\
Bondu 3SG.PST what see\\
`What did Bondu see?'

\exg. bòndú á fèn sã̀\\
Bondu 3SG.PST what buy\\
`What did Bondu buy?'

\ex. ní\\
`And'


\exg. kàì á fèɱ fɔ̀-ì-jé bòndú á fèn d͡ʒé ní bòndú á fèn sã̀\\
Kai 3SG.PST what say-2SG-to Bondu 3SG.PST what see and Bondu 3SG.PST what buy\\
`What did Kai say what did Bondu see and what did Bondu buy?' (Fo with SM)

\jf{Answer:}\\

\exg. kàì á fɔ̀-n-d͡ʒé bòndú á kònén d͡ʒé ní á swèè-àn sã̀\\
Kai 3SG.PST say-1SG-to Bondu 3SG.PST tree see and 3SG.PST meat-FOC buy\\
`Kai told me that Bondu saw a tree and that she bought meat'

\exg. kàì á fèɱ fɔ̀-ì-jé bòndú á fèn d͡ʒé ní  á fèn sã̀\\
Kai 3SG.PST what say-2SG-to Bondu 3SG.PST what see and 3SG.PST what buy\\
`What did Kai say what did Bondu see and what did Bondu buy?' (Fo with SM)

\exg. *kàì ɔ́ té bòndú á fèn d͡ʒé ní bòndú á fèn sã̀\\
Kai 3SG.OBJ say Bondu 3SG.PST what see and Bondu 3SG.PST what buy\\
`What did Kai say what did Bondu see and what did Bondu buy?' (te with LDQ - CAN'T HAPPEN!! -- ONLY COULD WORK IF A QUOTE)

\subsection{double-embedded:}

\exg. \\
\\
`What did you hear that Kai said that Bondu bought?' \\ 

\section{Wesley}

\exg. Kai à swèé mín tàwá, Bondu à dã̀ũ̀\\
{} \textsc{3SG.PST} meat \textsc{mi} cook {} \textsc{3SG.PST} eat\\
`The meat that Kai cooked, Bondu ate it.’ \hfill{(05-14-2025, 14:33)}

\exg. Kai à swèé mín tàwá, Bondu à swèé tʃɛ̀ dã̀ũ̀\\
{} \textsc{3SG.PST} meat \textsc{mi} cook {} \textsc{3SG.PST} meat \textsc{dem} eat\\
`The meat that Kai cooked, Bondu ate [that meat].’ \hfill{(05-14-2025, 14:57)}

\exg. Kai à swèé mín tàwá, Bondu à án swèé àn dã̀ũ̀\\
{} \textsc{3SG.PST} meat \textsc{mi} cook {}  \textsc{3SG.PST} \textsc{anp} meat \textsc{foc} eat\\
`The meat that Kai cooked, Bondu ate [that meat].’ \hfill{(05-14-2025, 15:27)}

\wml{Sentence was volunteered to consultant without focus-marking \textit{an}; he corrected the second half to include \textit{an}.}
\exg. swèé tʃɛ̀, Kai à mín tàwá, Bondu à dã̀ũ̀\\
meat \textsc{dem} {} \textsc{3SG.PST} \textsc{mi} cook {} \textsc{3SG.PST} eat\\
`That meat, the one that Kai cooked, Bondu ate it.’ \hfill{(05-14-2025, 15:46)}

\exg. swèé tʃɛ̀, Bondu à dã̀ũ̀, Kai à mín tàwá \\
meat \textsc{dem} {} \textsc{3SG.PST} eat {} \textsc{3SG.PST} \textsc{mi} cook \\
`That meat, Bondu ate it, the one that Kai cooked.’ \hfill{(05-14-2025, 16:20)}

\exg. Kai à swèé tʃɛ̀ mín tàwá, Bondu à dã̀ũ̀\\
{} \textsc{3SG.PST} meat \textsc{dem} \textsc{mi} cook {} \textsc{3SG.PST} eat \\
`That meat which Kai cooked, Bondu ate it.’ \hfill{(05-14-2025, 16:44)}

\exg. Bondu à g͡bòò tʃɛ̀ màsɔ̃̌ kàmá\\
{} \textsc{3SG.PST} book \textsc{dem} get how\\
`How did Bondu get the book?’ \hfill{(05-14-2025, 17:10)}

\exg. Bondu à g͡bòò màsɔ̃̌ kàmá\\
{} \textsc{3SG.PST} book get how\\
`How did Bondu get the book?’ \hfill{(05-14-2025, 17:26)}

\exg. Kai àn-á g͡bòò bè à-mà\\
{} \textsc{foc}-\textsc{aux} book give \textsc{AUX.PST}-to\\
`\textsc{kai} gave her the book.’ \hfill{(05-14-2025, 18:33)}

\exg. Kai à g͡bòò mím-bè à-mà\\
{} \textsc{3SG.PST} book \textsc{mi}-give \textsc{3sg}-to\\
`The book that Kai gave her.’ \hfill{(05-14-2025, 19:00)}

\exg. à ɱféà Bondu à à-n-náá kúnú Baiama\\
\textsc{3sg} and {Bondu} \textsc{a} \textsc{3pl}-come yesterday {}\\
`He and Bondu came to Baiama yesterday.’ \hfill{(05-14-2025, 20:40)}

\exg. à-wã́ nì Bondu à-n-náá kúnú Baiama\\
\textsc{3sg-strong} and {} \textsc{3-PL}-come yesterday {}\\
`He and Bondu came to Baiama yesterday.’ \hfill{(05-14-2025, 21:22)}

\exg. Kai à mwòkàmá mĩ́-sɔ̃̂ nì Bondu à-n-náá kúnú Baiama\\
{} \textsc{3SG} man \textsc{mi}-know and {} \textsc{3-PL}-come yesterday {}\\
`The man whom Kai knows and Bondu came to Baiama yesterday.’ \hfill{(05-14-2025, 21:51)}

\exg. Kai à mwòkàmá mĩ́-sɔ̃̂, à-wã́ nì Bondu à-n-náá kúnú Baiama\\
{} \textsc{3SG} man \textsc{mi}-know and {} \textsc{3-PL}-come yesterday {}\\
`The man whom Kai knows and Bondu came to Baiama yesterday.’ \hfill{(05-14-2025, 23:04)}

\exg. Jai á jẽ̀ẽ̀ nì Bondu\\
{} \textsc{3SG.PST} see and {}\\
`Jai saw him and Bondu.’ \hfill{(05-14-2025, 23:30)}

\exg. Jai á jẽ̀ẽ̀ à ɱféà Bondu à \\
{} \textsc{3SG.PST} see \textsc{3SG} and {} \textsc{?}\\
`Jai saw him and Bondu.’ \hfill{(05-14-2025, 23:50)}

\exg. Jai á mwòkàmá jẽ̀ẽ̀ Kai à mĩ́-sɔ̃̂ nì Bondu \\
{} \textsc{3SG.PST} man see Kai \textsc{3SG.PST} \textsc{mi}-know and {}\\
`Jai saw him and Bondu.’ \hfill{(05-14-2025, 24:25)}

\wml{Consultant comment: in this case, Kai knows the man, but not necessarily Bondu.}
\exg. Kai à mwòkàmá mĩ́-sɔ̃̂, Jai á jẽ̀ẽ̀ à ɱféà Bondu à\\
{} \textsc{3sg} man \textsc{mi}-know {} \textsc{3SG.PST} see \textsc{3sg} and {} \textsc{3sg}\\
`The man that Kai knows, Jai saw him and Bondu.’ \hfill{(05-14-2025, 26:24)}

\exg. Kai à mwòkàmá mĩ́-sɔ̃̂, Jai á jẽ̀ẽ̀ nì Bondu\\
{} \textsc{3sg} man \textsc{mi}-know {} \textsc{3SG.PST} see and {}\\
`The man that Kai knows, Jai saw him and Bondu.’ \hfill{(05-14-2025, 26:40)}

\exg. mwòkàmá jã̀sã̀ Bondu má\\
man tall {} on\\
`The man is taller than Bondu.’ \hfill{(05-14-2025, 27:30)}

\exg. n-á mwòkàmá jẽ̀ẽ̀ mĩ́-jã̀sã̀ Bondu má\\
\textsc{1SG.PST} man see \textsc{mi}-tall {} on\\
`I saw the man who is taller than Bondu.’ \hfill{(05-14-2025, 27:40)}

\exg. Bondu jã̀sã̀ mwòkàmá má\\
{} tall man on\\
`Bondu is taller than the man.’ \hfill{(05-14-2025, 28:03)}

\exg. n-á mwòkàmá jẽ̀ẽ̀ Bondu jã̀sã̀ mĩ́-má\\
\textsc{1SG.PST} man see {} tall \textsc{mi}-on\\
`I saw the man Bondu is taller than.’ \hfill{(05-14-2025, 28:18)}

\exg. Bondu túmú mwòkàmá-à\\
{} love man-\textsc{a}\\
`Bondu loves the man.’ \hfill{(05-14-2025, 28:37)}

\exg. n-á mwòkàmá jẽ̀ẽ̀ Bondu túmú mín-à\\
\textsc{1SG.PST} man see {} love \textsc{mi}-\textsc{a}\\
`I saw the man whom Bondu loves.’ \hfill{(05-14-2025, 28:57)}

\exg. Bondu túmú mwòkàmá mín-à ná jẽ̀ẽ̀\\ 
{} love man \textsc{mi}-\textsc{a} \textsc{1sg} see\\
`I saw the man whom Bondu loves.’ \hfill{(05-14-2025, 29:16)}

\section{Daniel}
\exg. \textipa{\textltailn\'on\`a} \textipa{sw\`E\`E} \textipa{b\`E} \textipa{Bondu} (\textipa{Kai}) \textipa{m\'a?}\\
who meat give Bondu {} to?\\
`Who gave meat to Bondu (Kai)?'

\exg. *Bondu nì sw\textipa{\`E\`E} bè Kai-má kùnù\\
Bondu COP.PST meat give Kai-to yesterday\\
Intended: `It was Bondu who gave meat to Kai yesterday'

\exg. Bondu nì // miná sw\textipa{\`E\`E} bè Kai-má kùnù\\
Bondu COP.PST {} REL.SBJ meat give Kai-to yesterday\\
`It was Bondu who gave meat to Kai yesterday'

\exg. sw\textipa{\`E\`E} nì (*mì/*miná/*mindò) Bondu-à bè Kai-má kùnù\\
meat COP.PST REL/REL.SBJ/REL.OBJ Bondu-3SG give Kai-to yesterday\\
`It was meat that Bondu gave Kai yesterday'

\exg. kùnù nì Bondu-à sw\textipa{\`E\`E} bè Kai-má\\
yesterday COP.PST Bondu-3SG.PST meat give Kai-to\\
`It was yesterday that Bondu gave the meat to Kai'

\exg. kùnù tìmá-mìná-nà-nì Bondu-à sw\textipa{\`E\`E} bè Kai-má\\
yesterday time-REL.SUBJ-FOC-COP.PST Bondu-3SG.PST meat give Kai-to\\
`What time yesterday did Bondu give the meat to Kai?'

\exg. tìmá-mìná-nà-nì kùnù Bondu-à sw\textipa{\`E\`E} bè Kai-má\\
time-REL.SUBJ-FOC-COP.PST yesterday Bondu-3SG.PST meat give Kai-to\\
`What time yesterday did Bondu give the meat to Kai?'

\exg. morning\\
`Sɔmá'

\exg. during the day\\
`tɛ̀áná'

\exg. à swèè bɛ̀ àmá tɛ̀áná\\
3SG.PST meat give ? during the day\\
`She gave the meat during the day'

\exg. à swɛ̀ɛ̀ bɛ̀ àmá ɛ̀mɛ́ tɛ̀áná\\
`3SG.PST meat give ?  in the evening
`She gave him the meat in the evening'

\exg. tìmá-mìná-nà-nì Bondu-à sw\textipa{\`E\`E} bè Kai-má\\
time-REL.SUBJ-FOC-COP.PST Bondu-3SG.PST meat give Kai-to\\
`What time did Bondu give the meat to Kai?'


\exg. tìmá-mìná-nà-nì kùnù Bondu-à bè Kai-má\\
time-REL.SUBJ-FOC-COP.PST yesterday Bondu-3SG.PST give Kai-to\\
`What time yesterday did Bondu give (something) to Kai?'

\exg. mwòkámá t\textipa{S}á-n-à s\textipa{\`E\`E}-ò téé\\
man this-FOC-3SG.PST s\textipa{\`E\`E}-OBJ break\\
`\textsc{This} man broke the s\textipa{\`E\`E}'

\exg. mwòkámá t\textipa{S}\`E á\textipa{n-à } s\textipa{\`E\`E}-ò téé\\
man this-FOC-3SG s\textipa{\`E\`E}-OBJ break\\
`\textsc{This} man broke the s\textipa{\`E\`E}'

\exg. n-a mwòkámá j\textipa{\~{\'e}} *(mina) s\textipa{\`E\`E}-ò téé\\
1SG.PST man see REL.SUBJ \textipa{s\`E\`E}-OBJ break\\
`I saw a man who broke the \textipa{s\`E\`E}'

\exg. n-a mwòkámá j\textipa{\~{\'e}} s\textipa{\`E\`E}-ò té-à\\
1SG.PST man see \textipa{s\`E\`E}-OBJ break-AUX\\
`I saw the man breaking the \textipa{s\`E\`E}'

\exg. Bondu-à sw\textipa{\`E\`E}-àN dáõ kò ta\textipa{N}gá?\\
Bondu-3SG meat-FOC eat or cassava\\
`Did Bondu eat the meat or the cassava?'

\exg. Bondu-à sw\textipa{\`E\`E}-àN dáõ. Kai-àN-à ta\textipa{N}gá dáõ.\\
Bondu-3SG.PST meat-FOC eat Kai-FOC-3SG.PST cassava eat\\
`Bondu ate the \textsc{meat}. \textsc{Kai} ate the cassava.'

\exg. Bondu-à sw\textipa{\`E\`E} dáõ. Kai-à ta\textipa{N}gá-àN dáõ.\\
Bondu-3SG.PST meat-FOC eat Kai-3SG.PST cassava-FOC eat\\
`Bondu ate the \textsc{meat}. Kai ate the \textsc{cassava}.'

\exg. *Bondu-à sw\textipa{\`E\`E}-àN dáõ. Kai-àN-à ta\textipa{N}gá-àN dáõ.\\
Bondu-3SG.PST meat-FOC eat Kai-FOC-3SG.PST cassava-FOC eat\\
`Bondu ate the \textsc{meat}. \textsc{Kai} ate the \textsc{cassava}.'

\exg. *sw\textipa{\`E\`E} mù Bondu-àN dáõ. ta\textipa{N}gá mù Kai-àN-à dáõ.\\
meat COP.PRS Bondu-FOC eat cassava COP.PRS Kai-FOC-AUX.PST eat\\
Intended: `It was the meat that \textsc{bondu} ate. It was the cassava that \textsc{Kai} ate.'

\exg. Bondu-à kw\textipa{\'E\'E}-àN dáõ?\\
Bondu-3SG rice-FOC eat\\
`Did Bondu eat rice?'

\exg. À-à, sw\textipa{\`E\`E} mù.\\
no meat COP\\
`No, (it was) meat.'

\exg. \textipa{\textltailn ónà} fé\textipa{\~{\'e}} dáõ?\\
who what ate\\
`Who ate what?'

\exg. Bondu-à sw\textipa{\`E\`E}-àN dáõ. Kai-à ta\textipa{N}gá-àN dáõ.\\
Bondu-3SG.PST meat-FOC eat Kai-3SG.PST cassava-FOC eat\\
`Bondu ate the meat. Kai ate the cassava.'

\exg. *sw\textipa{\`E\`E} nì Bondu-à dáõ.\\
meat-COP.PST Bondu-3SG.PST eat\\

\exg. *sw\textipa{\`E\`E} nì Bondu-àN-à dáõ.\\
meat COP.PST Bondu-FOC-3SG.PST eat\\

* = means not acceptable

\section{Giang (5:30 zoom)}


\g{Testing for movement: Coordinate Structure Constraint}

47:00
\exg. Bondu  mfea  ɲona kaŋaná tõ kúnú\\
Bondu and who door close yestderday\\
`Bondu and who closed the door yesterday?'


\exg. ɲonfea Bondu-á kaŋane tõ kúnú\\
who-and Bondu 3SG.PST door close yesterday\\
`Who and Bondu closed the door yesterday?'


\g{What are the possible answers? Can you answer with just ``Saa'' or does it have to be ``Bondu and Saa''/``Saa and Bondu.''}

\exg. Bondu mfea Saa\\
`Bondu and Saa'

\exg. àn fánà kaŋane tõ kúnú\\
3PL    ?   door close yesterday\\
`They closed the door yesterday'

\jf{Can answer with just Saa}


\g{Multiple wh-questions}

\exg. ɲòná fèná tõ kúnú \\
who ? close yesterday\\
`Who closed what yesterday?'


\g{If the question is allowed, check for answers. Can it be a single pair or does it have to be a pair-list question?}

\g{Even if question is not allowed, is it allowed in echo contexts?}

\g{We have established that the following are acceptable answers to wh questions:}

\exg. B\textipa{\`ond\'u} \`a \textipa{k\'aNg\'an\`e} n\`a \textipa{t\'{\~o}} \textipa{k\`un\`u}.\\
Bondu \textsc{3SG.PST} door \textsc{pl-obj} close yesterday.\\
``Bondu closed door yesterday.''

\exg. B\textipa{\`ond\'u} \textipa{\`an-\'a} \textipa{k\'aNg\'an\`a} \textipa{t\'{\~o}} \textipa{k\`un\`u}.\\
Bondu \textsc{foc-3SG.PST} door-\textsc{obj} closed yesterday.\\
``It was Bondu who closed the door yesterday.''

Would the following be acceptable:

* \ex. B\textipa{\`ond\'u} \textipa{\`an-\'a} \textipa{k\'aNg\'an\`e} n\`a \textipa{t\'{\~o}} \textipa{k\`un\`u}.
``\textsc{Bondu} close the \textsc{door} yesterday.""


\section{Chun-Hung (5:45 zoom)}

\chs{\textbf{0. Some checking}}

\exg. B\`{o}nd\'{u}\textsubscript{i/k}-\`{a}-k\'{i} w\'{a} B\`{o}nd\'{u}\textsubscript{i}-b\'{o}\`{o}. \\
Bondu-3SG.POSS-key WA Bondu-hand \\ 
Bondu\textsubscript{i} has Bondu\textsubscript{i/k}'s key. \chs{can be in a scenario where only one Bondu is present (high likely) or where two (different) girls with the same name Bondu are present (less likely)}

\exg. n-\'{a}-d\'{e}-m\'{u}s\`{u}-\`{a}-w\'{u}\`{u} \\
1SG.POSS-child-female-3SG-dog \\
`my daughter's dog (the dog of my daughter)' 

\exg. n-\'{a}-d\'{e}-m\'{u}s\`{u}-n\v{u}-\`{a}-n-\`{a}-w\'{u}\`{u} \\
1SG.POSS-child-female-PL-3-PL-POSS-dog \\
`my daughter' dog (the dog of my daughter)' 

\exg. n-\'{a}-d\'{e}-m\'{u}s\`{u}-n\v{u}-\`{a}-n-\`{a}-w\'{u}\`{u}-n\`{u} \\
1SG.POSS-child-female-PL-3-PL-POSS-dog-PL \\
`my daughters' dogs (the dogs of my daughter)'

\chs{\textbf{A. Word order with functional categories} --- to establish that grammatical subjects precede functional categories, and non-subjects do not (but may be wrong)}

\chs{PP predication -- baseline} \newline

\chs{I'm guessing that the negation morpheme for PP predication is like vowel lengthening? So it has to make the pronominal-like auxiliary overt to attach to it or attach to the past tense morpheme?}

\exg. W\'{u} B\`{o}nd\'{u}-t\'{e}\`{a}. \\
dog Bondu-with \\
The dog with Bondu. 

\exg. W\'{u} \'{\textipa{E}}\'{\textipa{E}} B\`{o}nd\'{u}-t\'{e}\`{a}. \\
dog 3SG.NPST.NEG Bondu-with \\
The dog is not with Bondu. 

\exg. W\'{u} n\`{i} (w\'{a}) B\`{o}nd\'{u}-t\'{e}\`{a}. \\
dog PAST (WA) Bondu-with \\
`The dog was with Bondu.' \chs{I heard a slightly nasalized vowel for wa when I re-listened.}

\exg. W\'{u} n\`{i}\`{i} B\`{o}nd\'{u}-t\'{e}\`{a}. \\
dog PAST.NEG Bondu-with \\
`The dog was not with Bondu.'

\exg. W\'{u} (*m\'{a}) (*wã́) B\`{o}nd\'{u}-t\'{e}\`{a}. \\
dog (*FUT) (*WAN) Bondu-with \\
`The dog will be with Bondu.' \chs{The two morphemes in parentheses are both required.}

\exg. W\'{u} \'{\textipa{E}}\'{\textipa{E}}  m\'{a} B\`{o}nd\'{u}-t\'{e}\`{a}. \\
dog 3SG.NPST.NEG FUT Bondu-with \\
`The dog will not be with Bondu.'

\chs{verbal predication (negation) --- additional baseline}

\exg.B\`{o}nd\'{u} (*n\'{i}) \`{a}-d\`{e} jẽ̀ẽ̀. \\
Bondu (*PAST) 3SG.PST-mother see \\
`Bondu saw her mother.' 

\exg. B\`{o}nd\'{u} (n\'{i}) m\'{a} (a-)d\`{e} jẽ̀ẽ̀-n\'{i}. \\
Bondu (PAST) NEG (3SG.PST-)mother see-NEG \\
`Bondu didn't see her mother.'

\chs{predicative possessives --- to see whether possessums pattern with subjects of PP predication in occupying a \textit{presumably} Spec,TP position as preceding other functional categories}

\exg. W\'{u} w\`{a} B\`{o}nd\'{u}-b\'{o}\`{o}. \\ 
dog WA Bondu-hand \\
`Bondu has a dog.' 

\exg. W\'{u} \'{\textipa{E}}\'{\textipa{E}} B\`{o}nd\'{u}-b\'{o}\`{o}. \\ 
dog 3SG.NPST.NEG Bondu-hand \\
`Bondu doesn't a dog.' 

\exg. W\'{u} n\`{i} wã́ B\`{o}nd\'{u}-b\'{o}\`{o}. \\ 
dog PAST WAN Bondu-hand \\
`Bondu had a dog.'  

\exg. W\'{u} n\`{i}\`{i} B\`{o}nd\'{u}-b\'{o}\`{o}. \\ 
dog PAST.NEG Bondu-hand \\
`Bondu didn't have a dog.'  

\exg. W\'{u} m\`{a} wã́ B\`{o}nd\'{u}-b\'{o}\`{o}. \\ 
dog FUT WAN Bondu-hand \\
`Bondu will have a dog.'  

\exg. W\'{u} \'{\textipa{E}}\'{\textipa{E}} m\`{a} B\`{o}nd\'{u}-b\'{o}\`{o}. \\ 
dog 3SG.NPST.NEG FUT Bondu-hand \\
`Bondu will not have a dog.'  


\section{Alex}


\exg.
à             tón-dà  \\
3SG.OBJ.PST   close-A \\%
`It closed.'

\exg.
à-n-à           tón-dà  \\
3-PL-PST   close-A \\%
`They closed.'

\exg.
n-á            tón-dà  \\
1SG.PST   close-A \\%
`I closed.'

\exg.
j-á            tón-dà  \\
2SG.PST   close-A \\%
`You closed.'

\exg.
ɔ̀             té-à    \\
3SG.OBJ  break-A \\%
`It broke.'

\exg.
à-n-dɔ̀          té-à    \\
3-PL-PST.3SG.OBJ   break-A \\%
`They broke.'

\exg.
n-dɔ́           té-à    \\
1SG-PST.3SG.OBJ  break-A \\%
`I broke.'

\exg.
íɔ́            té-à    \\
2SG.OBJ.PST   break-A \\%
`You broke.' \label{44430}

\exg.
à-ɛ́             tòn-dà    waN \\
3SG-3SG.NPST   close-A   FOC \\%
`It will close.'

\exg.
à-n-ɛ̀ɛ́           tòn-dà    waN \\
3-PL-3SG.NPST   close-A   FOC \\%
`They will close.'

\exg.
n-ɛɛ            ton-da    waN \\
1SG-3SG.NPST   close-A   FOC \\%
`I will close.' (\textit{fast speech})

\exg.
n-á-ɛ́            tòn-dà    waN \\
1SG.PST-3SG.NPST   close-A   FOC \\%
`I will close.' (\textit{careful speech})

\exg.
j-á-ɛ́            tòn-dà    waN \\
2SG-PST-3SG.NPST   close-A   FOC \\%
`You will close.'

\exg.
ɔ̀ɔ́             tè-à      waN \\
3SG.OBJ  break-A   FOC \\%
`It will break.'

\exg.
à-n-dɔ̀ɔ́          tè-à      waN \\
3-PL-OBJ   break-A   FOC \\%
`They will break.'

\exg.
nɔ́ɔ́            tè-à      waN \\
1SG.OBJ.NPST   break-A   FOC \\%
`I will break.'

\exg.
nɔ́ɛ́            tè-à      waN \\
1SG.OBJ.NPST   break-A   FOC \\%
`I will break.' \label{53108}

\alex{Anthony gave multiple instances of \ref{53108}, where the non-past auxiliary is more clearly segmented out from the subject, not just in careful speech.}

\exg.
n-dɔ́ɔ́           tè-à      waN \\
1SG-OBJ   break-A   FOC \\%
`I will break.' \label{39569}

\alex{Interesting alternation between \ref{53108} and \ref{39569} concerning the /d/ insertion between /N/ and /ɔ/.}

\exg.
íɔ́ɔ́            tè-à      waN \\
2SG.OBJ.NPST   break-A   FOC \\%
`You will break.'

\exg.
à-n-à       ní   tòn-dà,    mbɛ    bondu   na   \\
3-PL-PST   NI   close-A,   then   Bondu   came \\%
`They were closing, then Bondu came.'

\exg.
ànɛ̀ɛ́           tòn-dà,    mbɛ    bondu   na   \\
3PL.OBJ.NPST   close-A,   then   Bondu   came \\%
`They were closing, then Bondu came.'


\exg.
à           nì   tón-dà,    mbɛ    bondu   na   \\
3SG(OBJ?)   NI   close-A,   then   Bondu   came \\%
`It was closing, then Bondu came.'

\exg.
n-á        ní   tòn-dà,    mbɛ    bondu   na   \\
1SG.PST   NI   close-A,   then   Bondu   came \\%
`I was closing, then Bondu came.'

\exg.
nɛ́ɛ́            tòn-dà,    mbɛ    bondu   na   \\
1SG.PST.?  close-A,   then   Bondu   came \\%
`I was closing, then Bondu came.'

\exg.
j-á        ní   tòn-dà,    mbɛ    bondu   na   \\
2SG.PST   NI   close-A,   then   Bondu   came \\%
`You were closing, then Bondu came.'

\exg.
j-á        ní   tón     tè-à    wáN,   mbɛ    bondu   na   \\
2SG.PST   NI   close   $v$-A   FOC,   then   Bondu   came \\%
`You were closing, then Bondu came.'

\exg.
j-áɛ́            tòn-dà,    mbɛ    bondu   na   \\
2SG.PST.?   close-A,   then   Bondu   came \\%
`You were closing, then Bondu came.'

\exg.
ɔ̀         ní   tè-à,      mbɛ    bondu   na   \\
3SG.OBJ   NI   break-A,   then   Bondu   came \\%
`It was breaking, then Bondu came.'

\exg.
à-n-dɔ́      ní   tè-à,      mbɛ    bondu   na   \\
3-PL-OBJ   NI   break-A,   then   Bondu   came \\%
`They were breaking, then Bondu came.' \label{33574}

\exg.
à-n-dɔ́-ɛ́         tè-à,      mbɛ    bondu   na   \\
3-PL-OBJ-NPST   break-A,   then   Bondu   came \\%
`They were breaking, then Bondu came.' \label{39200}

\exg.
à-n    ní   tè-à,      mbɛ    bondu   na   \\
3-PL   NI   break-A,   then   Bondu   came \\%
`They were breaking, then Bondu came.' \label{58506}

\alex{Anthony gave \ref{58506} as a possible form, which suggest that the ɔ-object marker might indeed be ``optional'' in some cases (cf. \ref{33574}).}

\exg.
n-dɔ́       ní   tè-à,      mbɛ    bondu   na   \\
1SG-OBJ   NI   break-A,   then   Bondu   came \\%
`I was breaking, then Bondu came.'

\exg.
n-dɔ́       ɛ́      tè-à,      mbɛ    bondu   na   \\
1SG-OBJ   NPST   break-A,   then   Bondu   came \\%
`I was breaking, then Bondu came.'

\exg.
j-ɔ́        ní   tè      tè-à    wáN,   mbɛ    bondu   na   \\
2SG-OBJ   NI   break   $v$-A   FOC,   then   Bondu   came \\%
`You were breaking, then Bondu came.' \label{25043}

\alex{Still not sure whether the verb in \ref{25043} is reduplication (as Anthony has reported, similar to \textit{ɛŋɡwai ka-ka-ka} `thank you very very much'), especially considering the presence of the focus marker, or whether this a complex predicate with /te/ as the light verb/verbalizer}

\exg.
j-ɔ́        ní   tè-à,      mbɛ    bondu   na   \\
2SG-OBJ   NI   break-A,   then   Bondu   came \\%
`You were breaking, then Bondu came.'

\exg.
j-ɔ́ɔ́            tè-à,      mbɛ    bondu   na   \\
2SG-OBJ   break-A,   then   Bondu   came \\%
`You were breaking, then Bondu came.' \label{14489}

\alex{Anthony reported that \ref{14489} could also be interpreted as the `past/completed' construction (cf. \ref{44430}). If this is indeed the case, then it gives further reason to suppose that in the simple past examples with the ɔ-object marker on the subject, the past marker is indeed there and contributes to some vowel lengthening. Then, the primary difference might come down to how likely is the /a/ vowel (past) or the /ɛ/ vowel (nonpast) to remain and contribute some lengthening, since we see /a/ (past) as quite short(ened) in such cases.}

\exg.
n-á        káŋánàà    tòŋ   \\
1SG.PST   door.OBJ   close \\%
`I closed the door.'

\exg.
à-á        káŋánàà    tòŋ   \\
3SG.PST   door.OBJ   close \\%
`He/she closed the door.'

\exg.
j-á        káŋánàà    tòŋ   \\
2SG.PST   door.OBJ   close \\%
`You closed the door.'

\exg.
à-n-á       káŋánàà    tòŋ   \\
3-PL-PST   door.OBJ   close \\%
`They closed the door.'

\exg.
à-n-á       n-á        tòŋ   \\
3-PL-PST   1SG.PST   close \\%
`They closed me.'

\exg.
ń-á        à           tòŋ   \\
1SG.PST   3SG.PST   close \\%
`I closed it.'

\exg.
ń-á        fɛ̀nàà       tòŋ   \\
1SG.PST   thing.OBJ   close \\%
`I closed it.'

\exg.
ń-á        j́-à        tòŋ   \\
1SG.PST   2SG.PST   close \\%
`I closed you.'

\exg.
ń-á        à-n-à       tòŋ   \\
1SG.PST   3PL.PST   close \\%
`I closed them.'

\exg.
à-n-á       ń-dɔ̀       tè    \\
3-PL-PST   1SG.OBJ   break \\%
`They broke me.'

\exg.
àá        ń-dɔ̀       te    \\
3SG.PST   1SG.OBJ   break \\%
`He/she broke me.'

\exg.
m-bé        káŋánàà    tòn-dà    waN \\
1SG.NPST   door.OBJ   close-A   FOC \\%
`I will close the door.'

\exg.
ɛ̀ɛ̀         káŋánàà    ton-da    waN \\
3SG.NPST   door.OBJ   close-A   FOC \\%
`He/she will close the door.' \label{16309}

\alex{In \ref{16309}, I definitely hear the pronoun + aux as a long vowel, supporting the idea that there are indeed two distinct morphemes}

\exg.
éé         káŋánàà    tòn-dà    waN \\
2SG.NPST   door.OBJ   close-A   FOC \\%
`You will close the door.'

\exg.
à-m-bè       káŋánàà    tòn-dà    waN \\
3-PL-NPST   door.OBJ   close-A   FOC \\%
`They will close the door.'

\exg.
ḿ-b-á        à           tòn-dà    waN \\
1SG-NPST-?   3SG(.OBJ)   close-A   FOC \\%
`I will close it.'

\exg.
m-bé        fɛ́nà-à      tòn-dà    waN \\
1SG-NPST   thing-OBJ   close-A   FOC \\%
`I will close it.'

\exg.
m-bé        j́á        tòn-dà    waN \\
1SG-NPST   2SG.OBJ   close-A   FOC \\%
`I will close you.'

\exg.
mbá        ànà       tòn-dà    waN \\
1SG.NPST   3PL.OBJ   close-A   FOC \\%
`I will close them.'

\exg.
à-n-á       ná        tòŋ   \\
3-PL-PST   1SG.OBJ   close \\%
`They closed me.'

\exg.
à-m-bè       ńdɔ́       té-à      waN   sina     \\
3-PL-NPST   1SG.OBJ   break-A   FOC   tomorrow \\%
`They will break me tomorrow.'


ní       káŋánàà    tòn-dà,    mbɛ    bondu   na   \\
1SG.NI   door.OBJ   close-A,   then   Bondu   came \\%
`I was closing the door, then Bondu came.'

\exg.
m-bé        káŋánàà    tòn-dà,    mbɛ    bondu   na   \\
1SG-NPST   door.OBJ   close-A,   then   Bondu   came \\%
`I was closing the door, then Bondu came.'

\exg.
ànì      káŋánàà    tón-dà,    mbɛ    bondu   na   \\
3SG.NI   door.OBJ   close-A,   then   Bondu   came \\%
`He was closing the door, then Bondu came.' \label{22883}

\exg.
íní      káŋánàà    tón-dà,    mbɛ    bondu   na   \\
2SG.NI   door.OBJ   close-A,   then   Bondu   came \\%
`You were closing the door, then Bondu came.'

\exg.
óní      káŋánàà    tón-dà,    mbɛ    bondu   na   \\
2PL.NI   door.OBJ   close-A,   then   Bondu   came \\%
`You all were closing the door, then Bondu came.'

\exg.
à-n-ní     káŋánàà    tón-dà,    mbɛ    bondu   na   \\
3-PL.NI   door.OBJ   close-A,   then   Bondu   came \\%
`They were closing the door, then Bondu came.' \label{60205}

\alex{Very cool, if you compare the tone on the NI auxiliary between \ref{22883} and \ref{60205}, you'll see that NI appears to take on the low tone of the 3SG pronoun in \ref{22883}, whereas in \ref{60205}, NI has a high tone, despite 3PL having a low tone. To me, this looks like the plural/nasal segment of the 3PL might be blocking tone spreading onto to the auxiliary NI. In the case of \ref{60205} where NI has a high tone, perhaps this reflects the underlying tone of this auxiliary. I am pretty sure we see this same pattern with other auxiliaries, but we'll need to check this pattern. Also, it does not seem like the high tone on NI is coming from the high tone on `door' (cf. \ref{28222} where the 3SG pronoun following NI has a low tone, and NI remains high.)}

\exg.
à-n-ní     ná        tón-dà,    mbɛ    bondu   na   \\
3-PL.NI   1SG.OBJ   close-A,   then   Bondu   came \\%
`They were closing me, then Bondu came.'

\exg.
à-n-ní     ńdɔ́       té-à,      mbɛ    bondu   na   \\
3PL.NI   1SG.OBJ   break-A,   then   Bondu   came \\%
`They were breaking me, then Bondu came.'

\exg.
à-n-ní     à(a)        tón-dà,    mbɛ    bondu   na   \\
3-PL.NI   3SG(.OBJ)   close-A,   then   Bondu   came \\%
`They were closing it, then Bondu came.'\label{28222}

\exg.
à-n-ní     ànà       tón-dà,    mbɛ    bondu   na   \\
3PL.NI   3PL.OBJ   close-A,   then   Bondu   came \\%
`They were closing them, then Bondu came.'

\exg.
à-n-ní     à-n-dɔ̀      té-à,      mbɛ    bondu   na   \\
3PL.NI   3PL.OBJ   break-A,   then   Bondu   came \\%
`They were breaking them, then Bondu came.'

\exg.
m-b-ɔ́        ɔ̀         té-à      waN \\
1SG.NPST-?   3SG.OBJ   break-A   FOC \\%
`I will break it.'

\exg.
m-b-á        àndɔ̀      té-à      waN \\
1SG.NPST-?   3PL.OBJ   break-A   FOC \\%
`I will break them.'

\exg.
nɔ́        ɔ̀         tè    \\
1SG.PST   3SG.OBJ   break \\%
`I broke it.'

\exg.
n-á        àndɔ̀      tè    \\
1SG.PST   3PL.OBJ   break \\%
`I broke them.'

\exg.
n-á        j-ɔ́        tè    \\
1SG.PST   2SG-OBJ   break \\%
`I broke you.'

\alex{In slow speech, I hear something like in \ref{14438}, where the 2SG object has a high tone, and the ɔ-object marker has a slightly lower tone. Note, though, that I provide \ref{14438} as an analysis, not necessarily what is precisely reflected in the recording:}

\exg.
n-á        í     ɔ̀     tè    \\
1SG.PST   2SG   OBJ   break \\%
`I broke you.' \label{14438}

\exg.
m-bé        j-ɔ́        té-à      waN \\
1SG.NPST   2SG.OBJ   break-A   FOC \\%
`I will break you.'
09oioi


\end{document}