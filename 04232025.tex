\documentclass{assets/fieldnotes}

\title{Kono (Sierra Leone)}
\author{LING3020/5020}
\date{University of Pennsylvania, Spring 2025\\04/23/2025 Class Projects Week 5}

\setcounter{secnumdepth}{4} %enable \paragraph -- for subsubsubsections

\begin{document}

\maketitle

\maketitle
\tableofcontents

\section{Chun-Hung}

\chs{\textbf{A. C-commanding relations between possessors and possessums} --- to test whether possessums are structurally higher than possessors as PP predication, or possessors (with delayed saturation) are structurally higher than possessums as proposed in Myler (2016)} \newline

\chs{1. Reflexives/Reciprocals --- to confirm that possessums bind possessors but that possessors cannot bind possessums}


\exg. \`{O}-d\'{o}nd\`{o} b\'{e}m\`{u}, m-b\`{e} w\'{a} \textipa{M}-f-\'{a}nd\`{i}-b\'{o}\`{o}. \\
O-one BEMU 1SG-NPST WA 1SG-self-hand \\
`Even though it's just me, I have myself.' 



\exg. ??\textipa{M}-f-\'{a}nd\`{i} w\'{a} \`{m}-b\'{o}\`{o}. \\
1SG-self WA 1SG.SER1-hand \\
`I have myself.' 


\exg. K\`{a}nd\'{i}\`{e} mb-\^{a} \textipa{\textltailn}ṍ-b\'{o}\`{o}. \\
student PL-A each.other-hand \\
The students have each other. (in the context of `They can support each other.') \chs{the part mb- may be from the plural marking for possessums as in PP predication, but I am not sure of the morpheme a-}

\exg. *\textipa{\textltailn}ṍ mb-\^{a} k\`{a}nd\'{i}\`{e}-b\'{o}\`{o}. \\
each.other PL-A student-hand \\
`The students have each other.'

\chs{2. Condition C --- the violation is seen on a par with transitives, but more restrictions are found with predicative possessives, and need to be checked again!}

\exg. B\`{o}nd\'{u}\textsubscript{i} \`{a}\textsubscript{i}-k\'{o}-t\textipa{S}\`{\textipa{E}}n\`{a}m\'{a}-b\'{o}\`{o}. \\
Bondu 3SG.POSS-brother-big-hand \\
`Her\textsubscript{i} older brother has Bondu\textsubscript{i}.' = `Bondu\textsubscript{i}'s older brother has her\textsubscript{i}.' \chs{Unlike literal English translation, no Condition C violation is present in Kono, which suggests the possessum is higher.}

\exg. \`{A}\textsubscript{*i/k}-k\'{i} w\`{a} B\`{o}nd\'{u}\textsubscript{i}-b\'{o}\`{o}. \\
3SG.POSS-key WA Bondu-hand \\
`Bondu\textsubscript{i} has her\textsubscript{*i/k} keys.' \chs{Tony may not like R-expressions are weakly bound, similar to transitive clause above.}

\exg. B\`{o}nd\'{u}\textsubscript{i}-\`{a}-k\'{i} w\'{a} \`{a}\textsubscript{i/*k}-b\'{o}\`{o}. \\
Bondu-A-key WA 3SG.POSS-hand \\
`She\textsubscript{i/*k} has Bondu\textsubscript{i}'s keys.' 
= `Bondu has her keys.' \chs{It is surprising that the pronoun cannot refer to someone else (unless the possessor is structurally higher at some point of derivation), and this needs to be double-checked and be asked how to say `He/She\textsubscript{k} has Bondu\textsubscript{i}'s key.'}

\exg. \`{\textipa{E}}\textsubscript{*i/k} B\`{o}nd\'{u}\textsubscript{i}-k\'{o}-t\textipa{S}\'{\textipa{E}}n\`{a}m\`{a}-b\'{o}\`{o}. \\
3 SG.SER3 Bondu-brother-big-hand \\
`Bondu\textsubscript{i}'s older brother has it\textsubscript{*i/k}.' \chs{The pronoun is interpreted most likely as an inanimate object and also as a baby according to Tony, but even if Bondu is a baby, the pronoun cannot refer to Bondu, and hence the Condition C violation may be present.}

\exg. B\`{o}nd\'{u}\textsubscript{i}, \`{a}\textsubscript{k}-k\'{i} w\'{a} \`{a}\textsubscript{i}-b\'{o}\`{o}. \\
Bondu A-key WA 3SG.POSS-hand \\
`Bondu\textsubscript{i} has her\textsubscript{k} keys.' \chs{This sentence is identical to (7) but with a unexpected interpretation, and I can only think of topicalization (with the segmentation here) that may derive this meaning, but also needs double-checked.}

\chs{3. Variable binding}

\exg. K\`{a}nd\'{i}\`{e} n\^{a}n-d\'{e}-d\textipa{Z}\`{e}\`{e}. \\
student NAN-mother-see \\
`Students saw their(?) mother.' \chs{originally targeted as `Every student\textit{\scriptsize{i}} saw his\textit{\scriptsize{i}} (own) mother.'}

\jal{nan = n-an   PL-3plAux}

\exg.  K\`{a}nd\'{i}\`{e} mb\'{e} n\^{a}n(/*\`{a})-\textipa{\textltailn}ṍ-d\'{e} d\textipa{Z}\`{e}\`{e}. \\ 
student PL NAN(/*A)-each.other-mother see \\
`Students saw their each other's(?) mother.' \chs{originally targeted as `Every student\textit{\scriptsize{i}} saw his\textit{\scriptsize{i}} (own) mother.'}

\ex. His\textit{\scriptsize{i}} mother saw every student\textit{\scriptsize{i}}. (meaning Every student's mother saw him.)

\ex. Every student\textit{\scriptsize{i}} has his\textit{\scriptsize{i}} (own) schoolbag.

\ex. Every boy\textit{\scriptsize{i}} has his\textit{\scriptsize{i}} (own) dog.

\ex. His\textit{\scriptsize{i}} mother has every student\textit{\scriptsize{i}} (with her). (meaning Every student's mother has him with her.) 

\ex. His\textit{\scriptsize{i}} boy has every dog\textit{\scriptsize{i}}. 

\exg. K\`{a}nd\'{i}\`{e} k\`{a}ṍkã̀ m\'{a} d\`{e}-jẽ́ẽ́-n\'{i}. \\
student none.of.them NEG mother-see-NEG \\
`No student\textit{\scriptsize{i}} saw (his\textit{\scriptsize{i}}) mother.' \chs{double negation? The word was translated as none-of-them, and the final ã̀  may be 3PL?}

\ex. His\textit{\scriptsize{i}} mother saw no student\textit{\scriptsize{i}}.

\exg. K\`{a}nd\'{i}\`{e} k\`{a}ṍkã̀, (\`{a}n\`{a}/*\`{a}-)b\'{a}g\`{i} \'{e} ã̀-b\'{o}\`{o}. \\
student none.of.them (3PL.SER2/*3SG.SER2-)bag 3SG.NPST 3PL.SER1-hand \\
`No student\textit{\scriptsize{i}} has (their\textit{\scriptsize{i}}) backpack.'

\exg. B\'{a}g\`{i} k\`{a}nd\'{i}\`{e} k\`{a}ṍkã̀-b\'{o}\`{o}. \\
bag student none.of.them-hand \\
`No student\textit{\scriptsize{i}} has (their\textit{\scriptsize{i}}) backpack.' \chs{I'll put a pronoun for the word `bag'.}

\ex. His\textit{\scriptsize{i}} mother has no student\textit{\scriptsize{i}} (with her). 

\chs{I may continue with the negative quantifier though the pronoun which comes with it is also third plural, and will try to figure out whether `student' + `none.of.them' is a constituent or not and whether the word none.of.them is internally complex (compared to none of us or the like)} \newline

\chs{\textbf{B. Predicative possessives, PP predication, NP predication} --- to argue that predicative possessives pattern with PP predication (in line with Creissels (2025) who described `hand' as an adposition) not NP predication} \newline

\chs{0. Baseline}


\chs{2. Vowel change --- The adjective for the possessum in predicate possessives and subject with PP predication has a different final vowel than that for the subject with NP predication. (I expected t\textipa{S}\textipa{E}n\textipa{E}me though...)}

\exg. W\`{u}-t\textipa{S}\'{\textipa{E}}\`nam\^{e} w\'{a} B\`{o}nd\'{u}-b\'{o}\`{o}. \\
dog-big WA Bondu-hand \\
`Bondu has a big dog.' 

\jf{add example with tsenama}

\exg. W\`{u}-t\textipa{S}\'{\textipa{E}}\`nam\^{e}
 B\`{o}nd\'{u}-t\'{e}\`{a}. \\
 dog-big Bondu-with \\
 `The big dog is with Bondu.'

\exg. W\`{u}-t\textipa{S}\'{\textipa{E}}n\`{a}m\`{a} ǎ-m\`{u} B\`{o}nd\'{u}-\`{a}-s\'{o}f\'{e}n(\`{e})-ǎ. \\
dog-big A-COP Bondu-3SG.POSS-pet-A \\
`The big dog is Bondu's pet.' \chs{I'm not sure what the a- before the copula is.} 

\chs{\textbf{C. Covert postposition in predicative possessives} --- to test whether there's a covert postposition after `hand' in predicative possessives or it's `hand' that serves as a postposition} \newline



\section{Wesley}

\wml{The questions in \ref{question-book} and \ref{question-woman} target an object DP. I tested possible answers to these questions and among them are \ref{answer_book} and \ref{answer_woman} respectively, supporting that these structures are relativised DPs and not full clauses.}

\textbf{Question}
\exg. \label{question-book}
    Bondu g͡bòò mín-à à-ŋ kààntèà?\\
    {} {} book \textsc{mi-a} \textsc{3-PL} read\\
    `Which book is Bondu reading?\hfill{(04-22-25, 43:27)}

\textbf{Answers to \ref{question-book}}
    
\exg. Kai à g͡bòò mím-bè Bondu mà, Bondu à kàándà\\
{} \textsc{3SG.PST} book \textsc{mi}-give {} to {} \textsc{3sg} read\\
`The book that Kai gave to Bondu, Bondu is reading it.’\hfill{(04-22-25, 43:56)}
    
\exg. Bondu g͡bòò míŋ-kààntèà, Kai ànà bè à-mà\\
{} book \textsc{mi}-read {} \textsc{3sg.foc} give \textsc{3SG.PST}-to\\
`The book that Bondu is reading, Kai gave it to him.’\hfill{(04-22-25, 44:14)}

\exg. Kai à g͡bòò mím-bè à-mà\\
{} \textsc{3SG.PST} book \textsc{mi}-give \textsc{3sg}-to\\
`The book that Kai gave him.’\hfill{(04-22-25, 44:38)}

\exg. Kai *(à) mím-bè à-mà\\
{} \textsc{3sg} \textsc{mi}-give \textsc{3sg}-to\\
`The one that Kai gave him.’\hfill{(04-22-25, 44:38)}\label{answer_book}

\exg. * g͡bòò Kai à mím-bè à-mà\\
book {} \textsc{3SG.PST} \textsc{mi}-give \textsc{3sg}-to\\
\trans `The book that Kai gave him.’\hfill{(04-23-25, 1:35:38)}

\noindent{\rule{\textwidth}{1pt}}

\textbf{Question:}
\exg.
Kai à mùsù mín-à ã̀ sɔ̃̂?\\
{} \textsc{3SG.PST} woman \textsc{mi-3sg} \textsc{foc} know\\
`Which woman does Kai know?’\hfill{(04-23-25, 1:35:52)} \label{question-woman}

\textbf{Answers to \ref{question-woman}}

\exg. Kai à mùsù sɔ̃̂, mín-à jàg͡básì sã̂\\
{~} \textsc{3SG.PST} woman know \textsc{mi-3sg} onion buy\\
`Kai knows the woman who bought an onion.’\hfill{(04-23-25, 1:36:16)}

\exg. mùsù mín-à jàg͡básì sã̂, Kai à ã̀ sɔ̃̂\\
woman \textsc{mi-3sg} onion buy {} \textsc{3SG.PST} \textsc{foc} know\\ 
`Kai knows the woman who bought an onion.’\hfill{(04-23-25, 1:36:27)}

\exg. mùsù mín-à jàg͡básì sã̂\\
woman \textsc{mi-3sg} onion buy\\
`The woman who bought an onion.’ \hfill{(04-23-25, 1:36:50)}\label{answer_woman}

\exg. mín-à jàg͡básì sã̂\\
\textsc{mi-3sg} onion buy\\
`The one who bought an onion.’\hfill{(04-23-25, 1:37:37)}

\noindent{\rule{\textwidth}{1pt}}

\wml{\textbf{Hypothesis: [wɛ́nà] = wɛ̃́ + a}}\\

\wml{For RCs with pronominal heads (e.g., \textit{You who know...}), we've been seeing the element wɛ́nà or fɛ́nà. I propose that this is composed of [wɛ̃́], functioning a relative marker analogous to [mĩ́], that is unique to pronominal heads. (Could it be `also'?) In this example, [wɛ̃́] with the series 2 pronoun [j\'{a}] gives [í-wɛ́n-à]; the nasalisation on [wɛ̃́] surfaces as [n]. This gives supporting evidence that the series 2 pronouns are indeed decomposed to series 1 + -a. Here, the [wɛ̃́] morpheme intervenes between them.}

\exg. í-wɛ́n-à Bondu sɔ̃̀, é nàà wã̀ sìná\\
\textsc{2SG}-\textsc{we}-\textsc{a}\textsc{} Bondu know \textsc{2SG.NPST} come \textsc{wa} tomorrow\\
`You (sg.), who know Bondu, are coming tomorrow.’\hfill{(04-02-2025)}

\wml{That [wɛ̃́] is a morpheme by itself is also supported by the unaccusative data below, where we expect the series 1 pronouns and no -\textit{a}. Indeed, we see [wɛ̃́] attached directly to the verb in \ref{you_who_fell}, [í wɛ́m-bèà]. If the morpheme were [wɛ̃́n\'{a}], here we would expect *[í-wɛ̃́n\'{a} béà].}

\exg. í béà\\
\textsc{2SG} fall\\
`You (sg.) fell.’\hfill{(04-23-25, 1:38:14)}

\exg. í táá t͡ʃɛ́nà\\
\textsc{2SG} return home\\
`You (sg.) went home.’\hfill{(04-23-25, 1:38:27)}

\exg. ó wɛ́m-bèà, wé táá wã̀ t͡ʃɛ́n-à\\
\textsc{2PL} \textsc{we}-fall, \textsc{2PL-NPST} return \textsc{wa} home-to\\
`You (pl.) who fell, you (pl.) will go home.’\hfill{(04-23-25, 1:38:51)}\label{you_who_fell}

\exg. í wɛ́m-bèà, é táá wã̀ t͡ʃɛ́n-à\\
\textsc{2SG} \textsc{we}-fall, \textsc{2SG.NPST} return \textsc{wa} home-to\\
`You (sg.) who fell, you (sg.) will go home.’\hfill{(04-23-25, 1:39:22)}

\exg. í wɛ́m-bèà, í tàà t͡ʃɛ́n-à\\
\textsc{2SG} \textsc{we}-fall, \textsc{2SG} return home-to\\
`You (sg.) who fell, you (sg.) went home.’\hfill{(04-23-25, 1:39:50)}

\exg. à wɛ́m-bèà, à tàà t͡ʃɛ́n-à\\
\textsc{3SG} \textsc{we}-fall, \textsc{3SG} return home-to\\
`He who fell, he went home.’\hfill{(04-23-25, 1:40:05)}

\exg. à wɛ́m-bèà, à tàà t͡ʃɛ́n-à\\
\textsc{3SG} \textsc{we}-fall, \textsc{3SG} return home-to\\
`He who fell, he went home.’\hfill{(04-23-25, 1:40:05)}

\exg. í wɛ́m-bèà, táá t͡ʃɛ́n-à\\
\textsc{2SG} \textsc{we}-fall, return home-to\\
`You (sg.) who fell, go home.’ (imperative) \hfill{(04-23-25, 1:40:26)}

\exg. à-ɱ fɛ́m-bèà, à-m-bè tàà wã̀ t͡ʃɛ́n-à\\
\textsc{3-PL} \textsc{we}-fall, \textsc{3-PL-NPST} return home-to\\
`They who fell, they went home.’\hfill{(04-23-25, 1:41:06)}

\noindent{\rule{\textwidth}{1pt}}

\exg. nì mòmó bèà, é tàà wã̀ t͡ʃɛ́n-à\\
\textsc{ni} everyone fall \textsc{2SG.NPST} return \textsc{wa} home-to\\
`Whoever falls, you (sg.) will go home.’ \hfill{(04-23-25, 1:42:23)}

\exg. nì mòmó bèà, wé tàà wã̀ t͡ʃɛ́n-à\\
\textsc{ni} everyone fall \textsc{2PL-NPST} return \textsc{wa} home-to\\
`Whoever falls, you (pl.) will go home.’ \hfill{(04-23-25, 1:42:33)}

\exg. Kai féŋ-go-fẽ́n dàùɱ-fã̀, Bondu à min-tawa\\
{} thing-\textsc{go}-thing eat-\textsc{wa} {} \textsc{3SG} \textsc{mi}-cook\\
`Kai eats whatever Bondu cooks.’ \hfill{(04-23-25, 1:43:01)}

\exg. Bondu à mín-táwá, Kai à dàùɱ-fã̀.\\
{} \textsc{3SG} \textsc{mi}-cook {} \textsc{3SG} eat-\textsc{wa}\\
`What Bondu cooks, Kai will eat.’ \hfill{(04-23-25, 1:43:40)}

\exg. Bondu à féŋ-go-fẽ́n táwá, Kai à dàùɱ-fã̀.\\
{} \textsc{3SG} thing-\textsc{go}-thing cook {} \textsc{3SG} eat-\textsc{wa}\\
`What(ever) Bondu cooks, Kai will eat.’ \hfill{(04-23-25, 1:44:18)}

\exg. Bondu à fénè mín-táwá, Kai à dàũ̀.\\
{} \textsc{3SG.PST} thing \textsc{mi}-cook {} \textsc{3SG.PST} eat\\
`What Bondu cooked, Kai ate.’ \hfill{(04-23-25, 1:45:00)}

\exg. Bondu à tàwà mín-tʃɛ̀, Kai à dàũ̀.\\
{} \textsc{3SG.PST} cook \textsc{mi-?} {} \textsc{3sg.PST} eat\\
`What Bondu cooked, Kai ate.’ \hfill{(04-23-25, 1:45:29)}\\
\wml{I wonder if this tʃɛ̀ is the determiner we’ve been seeing---something like, `that which Bondu cooked’?}

\exg. Bondu à tàwà mín-tʃɛ̀, Kai à dàũ̀.\\
{} \textsc{3sg.PST} cook \textsc{mi-det} {} \textsc{3sg.PST} eat\\
`What Bondu cooked, Kai ate.’ \hfill{(04-23-25, 1:45:29)}

\exg. mòmó mín-à Kai sɔ̃́, à-m-bè nàà wã̀ sínà\\
everyone \textsc{mi-3SG} {} know \textsc{3-PL-NPST} come \textsc{wa} tomorrow\\
`Whoever knows Kai will come tomorrow.’ \hfill{(04-23-25, 1:46:16)}

\section{Joey}

\subsection{Hear:}

\exg. àmí(N)\\
`Hear' \\

\jf{There appears to be a final Nasal, but I don't notice it in all instances}

\exg. èj-á fèn àmì\\
2SG-PST what hear\\
`What did you hear?'\\

\exg. n-á àmí bòndú á swèè sã̀  kùnù\\
1SG.PAST hear Bondu 3SG.PST meat buy yesterday\\
`I heard that Bondu bought meat yesterday' \jf{(Declarative throught)}\\

\exg. èj-á àmí-fã̀  bòndú á swèè sã̀  kúnú\\
2SG-PST hear-FOC Bondu 3SG.PST meat buy yesterday\\
`Did you hear that Bondu bought meat yesterday?' \jf{(Matrix interrogative, embedded declarative)}\\

\jf{The above is "more like a statement," per Tony. It's used to confirm information. Seems similar to making a statement then sticking "right?" at the end.}

\jal{I understood it as only the intonation indicated it was a question}

\exg. í-n-á ámíɱ-fã̀  bòndú á swèè sã́  kúnú\\
2SG-1SG.PST hear-FOC Bondu 3SG.PST meat buy yesterday\\
`Did you hear that Bondu bought meat yesterday?' \jf{(Matrix interrogative, embedded declarative)}\\

\jf{This is more of a question. NOTE: This is somewhat unexpected. have we seen different morphology before based on whether something is a question or a statement? Namely, what is this "n" in between 2SG and the aux? Is it a manifestation of the Q C-head (seen before as "ni")?}

\jf{Answer to the above (can be a response to either of the above Qs):}

\exg. n-á ámíɱ-(fã̀ ) bòndú á swèè sã̀  kùnù\\
1SG.PST hear-(FOC) Bondu 3SG.PST meat buy yesterday\\
`I heard that Bondu bought meat yeterday' \\

\exg. n-á àmíɱ-(fã̀ ) bòndú á fèn sã̀  kùnù\\
1SG.PST hear-(FOC) Bondu 3SG.PST what buy yesterday\\
`I heard what Bondu bought yesterday?' \jf{(matrix declarative, embedded interrogative}\\

\exg. èj-á fèn àmím bòndú á fèn sã̀  kùnù\\
2SG-PST what hear Bondu 3SG.PST what buy yesterday\\
`What did you hear that Bondu bought yesterday?' 

\jf{(Interrogative throughout - long distance wh}\\

\jf{Answer to the above:}

\exg. n-á àmí bòndú á swèè-àn sã̀  kùnù\\
1SG.PST hear Bondu 3SG.PST meat-FOC buy yesterday\\
`I heard that Bondu bought meat yesterday' \\


\subsection{Declarative:}

\exg. kàì ɔ́ té bòndú á swèè sã̀\\
Kai 3SG.PST.3SG.OBJ say Bondu 3SG.PST meat buy\\
`Kai said that Bondu bought meat' \jf{(declarative throughout}\\

\exg. kàì á fɔ̀-n-d͡ʒè...\\
Kai 3SG.PST say-1SG-to \\
`Kai told me...' \jf{(This can be resumed by the same embedded CP as the above, per Tony's suggestion)}\\

\jf{To get actual DECL word order, need a non-CP obj (maybe "story," or just "words."}

\exg. *kàì n-ɔ́ té ...\\
Kai 1SG-ɔSER say \\
`Kai told me...' \jf{(NB: I did forget to include the epenthetic d after n, but since I theoretically said the underlying representation, and since the below with "j" is not workable, that suggests that IO's don't work here)}\\

\exg. *kàì j-ɔ́ té ...\\
Kai 2SG-PST.3SG.OBJ say \\
`Kai told you...' \\

\exg. bòndú á fɔ̀-kàì-jè\\
Bondu 3SG.PST say-Kai-to\\
`Bondu told Kai'\\

\exg. bòndú á fèɱ fɔ̀\\
Bondu 3SG.PST what say\\
`What did Bondu say?\\

\exg. bòndú ɔ́ té...\\
Bondu 3SG.PST.3SG.OBJ say\\
`What did Bondu say? \jf{(We elicited this in 4/16 as well. It has that "Bondu said..." interpretation.)}\\

\exg. bòndú á fèn(è) ɔ́ té...\\
Bondu 3SG.PST something 3SG.PST.3SG.OBJ break\\
`Bondu broke something' \jf{(te (as say) can't take non-CP DO, apparently. Or at least can't take a wh-phrase as object)}\\



\section{Alex}

\exg.
ani      kaŋan-a    tond-a    kunu,        bɛ     saa   ana  \\
3PL.NI   door-OBJ   close-A   yesterday,   then   Sah   came \\%
`They were closing the door, then Sah came.'

\exg.
ni       a         toN     te-a   kunu,        bɛ     saa   ana  \\
1SG.NI   OBJ   close   v-A    yesterday,   then   Sah   came \\%
`I was closing it, then Sah came.' \label{62301}

\exg.
ni       a         tond-a    kunu,        bɛ     saa   ana  \\
1SG.NI   OBJ   close-A   yesterday,   then   Sah   came \\%
`I was closing it, then Sah came.' \label{32115}

\alex{Good minimal pair with verbalizing morphology \textit{toN tea} in \ref{62301} vs. \textit{tonda} in \ref{32115}.}

\exg.
ni       a-n-a      tond-a    kunu,        bɛ     saa   ana  \\
1SG.NI   3-PL-OBJ   close-A   yesterday,   then   Sah   came \\%
`I was closing them, then Sah came.'

\exg.
ni       a-n-a      toN     te-a   waN   kunu,        bɛ     saa   ana  \\
1SG.NI   3-PL-OBJ   close   c-A    FOC   yesterday,   then   Sah   came \\%
`I was closing them, then Sah came.'

\exg.
saa   a     N     taN    kunu,        mban-a            tond-a  \\
Sa    3SG.PST   1SG   meet   yesterday,   1SG.AUX.3PL-OBJ   close-A \\%
`Sah met me yesterday, (while) I was closing the door.'

\exg.
kaŋan-a    ni   ton     te-a,   bɛ     sa   ana  \\
door-OBJ   NI   close   ?-A,    then   Sa   came \\%
`The door was closing, then Sah came.'

\exg.
kaŋan-a    ni   tond-a,    bɛ     sa   ana  \\
door-OBJ   NI   close-A,   then   Sa   came \\%
`The door was closing, then Sah came.'

\exg.
kaŋanɛ-n-a    ni   toN     te-a,   bɛ     sa   ana  \\
door-PL-OBJ   NI   close   v-A,    then   Sa   came \\%
`The doors were closing, then Sah came.'

\exg.
kaŋanɛ-n-a    ni   tond-a,    bɛ     sa   ana  \\
door-PL-OBJ   NI   close-A,   then   Sa   came \\%
`The doors were closing, then Sah came.'

\exg.
kaŋanɛ-n-e.a      tond-a,    bɛ     sa   ana  \\
door-PL-OBJ.AUX   close-A,   then   Sa   came \\%
`The doors were closing, then Sah came.'

\exg.
a-ni     tond-a,    bɛ     sa   ana  \\ 
3SG-NI   close-A,   then   Sa   came \\%
`It was closing, then Sah came.' \label{10893}

\exg.
a-n-ni     tond-a,    bɛ     sa   ana  \\ 
3-PL-NI   close-A,   then   Sa   came \\%
`They were closing, then Sah came.' \label{59676}

\alex{Interesting minimal pair between \ref{10893} and \ref{59676}. Note sure whether the only difference between these two is tone, or whether there might also be gemination of the nasal segment.}

\exg.
a-n-ni    ton     te-a,   bɛ     sa   ana  \\
3-PL-NI   close   v-A,    then   Sa   came \\%
`They were closing, then Sah came.'

\exg.
mani-a     di-tʃɛ.   bondu-we-a       di-tʃɛ \\
Mani-3SG.PST   cry-v.    Bondu-also-3SG.PST   cry-v  \\%
`Mani cried. Bondu also cried.'

\exg.
ɲom.bɛ   a     di-tʃɛ \\
who.also     3SG.PST   cry-v  \\%
`Who also cried?'

\exg.
bondu-we-an-a        di-tʃɛ \\
Bondu-also-FOC-3SG.PST   cry-v  \\%
`It was Bondu who also cried.'

\exg.
bondu-we-a       di-tʃɛ   waN \\
Bondu-also-3SG.PST   cry-v    FOC \\%
`It was Bondu who also cried.'

\exg.
mani     di-tʃɛ-a    waN.   bondu-we     di-tʃɛ-a    waN \\
Mani.?   dry-v-FUT   FOC.   Bondu-also   cry-v-FUT   FOC \\%
`Mani will cry. Bondu will also cry.'

\alex{Not sure if there is an auxiliary after `Mani' in the first clause; unsure about vowel length at the end of `Mani'.}

\exg.
ɲom.bɛ     a     di-tʃa    \\
who.also   3SG.PST   cry-v.FUT \\%
`Who will also cry?'

\exg.
bondu-we-a     di-tʃɛ-a    waN \\
Bondu-also-?   cry-v-FUT   FOC \\%
Answer: `Bondu will also cry.'

\exg.
taa        ɔ     te-a    \\
calabash   OBJ   break-A \\%
`The calabash broke.'

\exg.
fen     dɔ    te-a    \\
thing   OBJ   break-A \\%
`What broke?'

\exg.
taa        an    dɔ    te-a    \\
calabash   FOC   OBJ   break-A \\%
A: `It was the calabash that broke.'

\exg.
sani     an    dɔ    te-a    \\
bottle   FOC   OBJ   break-A \\%
A: `It was the bottle that broke.'

\exg.
sani-n      fan   dɔ    te-a    \\
bottle-PL   FOC   OBJ   break-A \\%
A: `It was the bottles that broke.'


\section{Lex}

\jf{Looking at lexical w/ double nasal environment}

\exg.
sánì-m      bɛ́     nà        tònd-à  \\
bottle-PL   also   3PL.AUX   close-A \\%
`The bottles also closed.' \label{The bottles also closed}\\
`The bottles, too, they closed.'

\exg. sánì wɛ́ bɛ́à bɛ̀á tònd-á\\
bottle also fell-a  (?and)   close-A \\
`the bottle also fell and closed'

\exg. sánì-m-bɛ̀ ná tònd-á bɛ̀án bɛ̀á\\
bottle-Pl-also 3PL.AUX close-A  (and/Also-3PL?) fall-a\\
`the bottles also closed and fell'

\exg. sánì-n á tònd-á bɛ̀án bɛ̀á\\
bottle-PL OBJ(.AUX?) close-A  (?and/ Also-3Pl) fall-A\\
`the bottles also closed and fell'

\exg. sánì kòmá wɛ̀á tònd-á\\
bottle old also close-A\\
`the old bottle also closed'

\exg. sánì kòmá-nù-ná tòndá\\
bottle old-PL-3Pl.Aux close-A\\
`the old bottles also closed'

\exg. sánì kòmá-m-bɛ̀  ná   tònd-á\\
    bottle old-Pl-Also 3Pl.AUX close-A\\
`the old bottles also closed'

\exg. sánì tʃénámà ni dà: tʃénámà tònd-á\\
bottle     big   and pot big    close-A\\
`the big bottle and pot  closed'

\exg. sánì tʃénámà ni dà: tʃénámà ná tònd-á\\
     bottle big   and pot big   3Pl.Aux close-a\\
`the big bottle and pot  closed'

\exg. sánì-wá ɱfɛ̀á ni dá-wá-na tònd-á\\
bottle-big   (?)  and pot-big-3Pl.AUX close-A\\
`the big bottle and pot  closed'

\exg. sánì-wá ni dà:-wá aná tònd-á\\
bottle-big and pot-big 3PL close-A\\
`the big bottles and pots closed' 

\exg. sánì-wá ɱfɛ̀á dà:-wá ná tònd-á\\
bottle-big   (?) pot-big 3PL.AUX close-A\\
`the big bottle and pot closed' 

\ex. sánì-wá-nù ni dà:-wá-nù ànà tònd-á\\
bottle-big-PL and pot-big-PL 3Pl close-A\\
`the big bottles and pots closed'

\exg. wúú m-bɔ̀\\
dog 1SG-hand\\
`I have a dog'

\exg. wúú á-m bɔ̀\\
dog POSS-1SG-hand\\
`I have a dog'

\exg. W\'{u}\'{u} m\'{a} \textipa{M}-f\'{a}m-b\'{o}\`{o}. \\
dog FUT 1SG.STRONG-hand \\
`It will be me that will have a dog.' \chs{focus interpretation}

\exg. ɱ-fámbɛ̀ wúú másɔ̀n dà\\
1SG-STRONG-?FUT?? dog get  ?\\
`It will be me that will get a dog'

\exg. ɱ-fámbɛ̀ wúú sànd-à\\
1SG-STRONG.? dog buy-A\\
`It will be me that will buy a dog'

\ex. kunɛ\\
`to awaken'

\ex. akunɛ\\
`wake him/her'

\exg. Bondu à ɱ-fán-dò tísá kùnù\\
Bondu \textsc{3SG.PST} 1\textsc{sg}-\textsc{foc}-\textsc{obj} ask yesterday\\
Bondu asked me yesterday.

\exg. Bondu à-n dɛ̀má kùnù\\
Bondu   ?-1SG  help   yesterday\\ 
`Bondu helped me yesterday'

\exg. Bondu í dɛ̀má kùnù\\
Bondu 2SG help yesterday\\
`Bondu helped you (sg.) yesterday'

\exg. Bondu wò dɛ̀má kùnù\\
Bondu 2Pl help yesterday\\
`Bondu helped you (pl.) yesterday' 

\exg. Bondu à-n fáníàfomà \\
Bondu ?-1SG lie \\
`Bondu accused me yesterday/ falsely accused'

\exg. Bondu à faníàfò-í-má kùnù\\
Bondu 3SG lied-2SG-1SG??  yesterday\\
`Bondu accused you (sg) yesterday'

\ex. faníá\\
`lie'
 

\ex. Bondu ándò tísà kùnù\\
Bondu ando question yesterday\\
`Bondu questioned you yesterday'


\section{Jan}

\subsection{[-ɔ] + [$\varnothing$í]}

\ex. kɔ \\
`older brother/sister'

\exg. bondu ko-$\varnothing$ $\varnothing$-jẽẽ  \\
bondu brother-AUX.3SG 2SG.OBJ-see \\
`Bondu's older brother saw you.'

\jmt{expected: kɔ (à) í jẽẽ}

\exg. jaj ko $\varnothing$-e jẽẽ kunu \\
Jai brother.older  AUX.3SG-2SG.OBJ \\
`Jai's older brother saw you.'

\jmt{expected: kɔ í jẽẽ  wã sínà}

\exg. bondu ko $\varnothing$-jẽẽnda wã sina \\
bondu brother.older 2SG.OBJ-see FUT tomorrow \\
`Bondu's older brother will see you tomorrow.'

\ex. jaj ko $\varnothing$-jẽẽnda ma \\
Jai brother.older 2SG.OBJ-see FUT \\
`Jai's older brother will see you tomorrow.'

\subsection{[$\varnothing$ɔ] + [$\varnothing$í]}

\jmt{Context. The mirror has been shattered by Saa.}

\exg. meme i te kunu \\
mirror 2SG cut yesterday \\
`The mirror cut you yesterday.'

\exg. a i te kunu \\
3SG.PST 2SG cut yesterday \\
`It cut you yesterday.'

\jmt{Now talking to me and Lex.}

\ex. `The mirror cut you all yesterday.'

\ex. a wo te kunu \\
3SG.PST 2PL cut yesterday \\
`It cut you all yesterday.'

\ex. The mirror will cut you all tomorrow

\jmt{Context. The mirror is still broken, and you want to warn Saa that he could be cut by it.}

\ex. `The mirror will cut you.'

\ex. `It will cut you.'

\ex. `The mirror will cut you all.'

\exg. e wo te a wã sina \\
3SG.NPST 2PL cut AUX FUT tomorrow \\
`It will cut you all.'

\ex. o saa
`Where is Saa? (to another person)' 

\jmt{This was [o saa]. Further context: it's dark.}

\exg. e mende
`where are you'

\ex. o ja \\
`Where are you?'

\jmt{Not series 1!}

\ex. o wa
`Where are you all?'

\jmt{Saa fell over and hurt himself.}

\ex. `Poor Saa!'

\exg. a ma tʃi n-dʒi de mu \\
3SG COP? ? 1SG-? ? ?\\
`Poor you!'


\section{Giang}
%Contrastive Focus
\exg. \textipa{\textltailn\'On\`a} \textipa{d\'i-tS\`e} \textipa{k\`un\`u}.\\
Who cry-v yesterday\\
``Who cried yesterday?''

\exg. S\textipa{\`a\`a} \textipa{\'an-\`a} \textipa{d\'i-tS\`e} \textipa{k\`un\`u}\\
Saa \textsc{foc}-\textsc{3sg} cry-v yesterday\\
``It was Saa that cried yesterday.''

\exg. \textipa{\`a\'a}, B\textipa{\`ond\'u} \textipa{\'an-\`a} \textipa{d\'i-tS\`e} \textipa{k\`un\`u}.\\
No, Bondu \textsc{foc}-\textsc{3sg} cry-v yesterday\\
``No, it was Bondu that cried yesterday.''

\exg. \textipa{\`a\'a}, S\textipa{\`a\`a} \textipa{m\v{a}}. B\textipa{\`ond\'u} \textipa{\'an-\`a} \textipa{d\'i-tS\`e} \textipa{k\`un\`u}.\\
No, Saa \textsc{neg?}. Bondu \textsc{foc}-\textsc{3sg} cry-v yesterday\\
``No, it's not Saa. It was Bondu that cried yesterday.''

\exg. \textipa{\textltailn\'On\`a} \textipa{sw\'e\'e} \textipa{d\`a\`o} \textipa{k\`un\`u}.\\
Who meat eat yesterday\\
``Who ate meat yesterday?''

\exg. S\textipa{\`a\`a} \textipa{\'a} \textipa{sw\'e\'e} \textipa{d\`a\`o} (*\textipa{w\'{\~a}}) \textipa{k\`un\`u}.\\
Saa \textsc{3SG.PST} meat eat (*wa) yesterday\\
``Saa ate meat yesterday.''

\exg. S\textipa{\`a\`a} \textipa{\'an-\`a} \textipa{sw\'e\'e} \textipa{d\`a\`o} \textipa{k\`un\`u}.\\
Saa \textsc{foc-3SG.PST} meat eat yesterday\\
``It was Saa who ate meat yesterday.''

\exg. \textipa{\`a\'a}. S\textipa{\`a\`a} \textipa{m\v{a}}. B\textipa{\`ond\'u} \textipa{\'an-\`a} \textipa{sw\'e\'e} \textipa{d\`a\`o} \textipa{k\`un\`u}.\\
No, Saa \textsc{neg}. Bondu \textsc{foc-3SG.PST} meat eat yesterday\\
``No! It's not Saa. It was Bondu who ate meat yesterday!''

\exg. B\textipa{\`ond\'u} \'a f\'e\'{\~e} \textipa{d\`a\`o} \textipa{k\`un\`u}.\\
Bondu \textsc{3sg} what eat yesterday\\
``What did Bondu eat yesterday?''

\exg. B\textipa{\`ond\'u} \textipa{\`a} \textipa{\textltailn\'E\'E} \textipa{\'En} \textipa{d\`a\`o} \textipa{k\`un\`u}.\\
Bondu \textsc{3SG.PST} fish \textsc{foc} eat yesterday.\\
``Bondu ate fish yesterday.''

\exg. \textipa{\`a\'a}, (B\textipa{\`ond\'u} m\'a \textipa{\textltailn\'E\'E} \textipa{d\`a\`o} n\'i.) \'a \textipa{sw\'e\'e} \textipa{\'En} \textipa{d\`a\`o} \textipa{k\`un\`u}.\\ 
No Bondu \textsc{neg} fish eat ?. 3\textsc{SG.PST} meat \textsc{foc} eat yesterday.\\
``No! Bondu didn't eat fish, she ate meat yesterday!''

\g{Now I am wondering whether it's \textipa{\'En} or \textipa{\'an}...}

\exg. B\textipa{\`ond\'u} a \textipa{f\'e\~{\'e}} \textipa{d\`ot\'e} \textipa{k\`un\`u}.\\
Bondu \textsc{3SG.PST} what break yesterday\\
``What did Bondu break yesterday?''

\exg. B\textipa{\`ond\'u} a \textipa{s\'E\'E} \textipa{\'En} \textipa{d\`ot\'e} \textipa{k\`un\`u}.\\
Bondu \textsc{3SG.PST} see \textsc{foc} break yesterday\\
``Bondu broke the see yesterday.''

\exg. \`a \'a. (B\textipa{\`ond\'u} m\'a \textipa{s\'E\'E} \textipa{d\`ot\'e} n\`i.) \'a \textipa{s\'a\ng{}b\`a\ng} \textipa{d\`ot\'e} \textipa{k\`un\`u}.\\
No, Bondu \textsc{neg} see break ?. \textsc{3SG.PST} drum break yesterday\\
No, Bondu didn't break the see yesterday, Bondu broke the drum yesterday.


\exg. ɲɔ́-n-à dí-t\textipa{S}\`{\textipa{E}} (*wã́) kùnù?\\
Who-\textsc{foc-AUX} cry-\textit{v} ? yesterday\\
``Who cried yesterday?''

\exg. wo wán-à \textipa{d\'i-tS\`e} \textipa{k\`un\`u}\\
2pl \textsc{foc-3SG.AUX} cry-v yesterday\\
``It was you(pl) who cried yesterday.''

\ex. \textipa{w\-'a} \textipa{d\'i-tS\`e} (\textipa{w\'{\~a}}) \textipa{k\`un\`u}\\
2\textsc{PL-PST} cry-v \textsc{foc} yesterday
``You(pl) cried yesterday.'' 


\section{Daniel}
\ds{Distribution of -mu/-na -- arguments vs. adjuncts}

\ex. It was with Kai that Bondu danced yesterday

\exg. Bondu-a tombwe do Kai-aN-a\\
Bondu-3SG dance \textit{v} Kai-FOC-3SG\\
`Bondu danced with Kai yesterday'

\ex. *Bondu-a tombwe do Kai-mu.

\ex. *Bondu-a tombwe do Kai-a-mu.

\exg. Kai-ni Bondu-a tombwe do-a\\
Kai-FOC.PST Bondu-3SG dance \textit{v}-with?\\
`It is with Kai that Bondu danced'

\exg. Bondu tombwe dõ-da Kai-aN-a sina\\
Bondu dance \textit{v}-FUT Kai-FOC-3SG tomorrow\\
`Bondu will dance with Kai tomorrow'

\ex. *Bondu tombwe dõ-da Kai-mu sina\\

\ex. *Bondu tombwe dõ-da Kai-a-mu sina\\

\ex. *Bondu tombwe dõ-da Kai-mu-a sina\\

\exg. *Kai-mu Bondu tombwe dõ-da sina\\
Kai-FOC Bondu dance \textit{v}-FUT tomorrow\\
Intended: `It is with Kai that Bondu will dance tomorrow'

\exg. *Kai-aN-a Bondu tombwe dõ-da sina\\
Kai-FOC-3SG Bondu dance \textit{v}-FUT tomorrow\\
Intended: `It is with Kai that Bondu will dance tomorrow'

\ex. Bondu síí-a Baiama wã\\
Bondu live-FUT Baiama \textit{wã}\\
`Bondu will live in Baiama'

\ex. Baiama-mu Bondu síí-a (wã)\\
Baiama-FOC Bondu live-FUT \textit{wã}\\
`It is in Baiama that Bondu will live'

\ex. *Baiama-na/ni Bondu síí-a (wã)\\
Baiama-FOC Bondu live-FUT \textit{wã}\\
`It will be with Kai that Bondu will dance tomorrow'

\ex. Baiama-mu Bondu táá\\
Baiama-FOC Bondu go\\
`It is to Baiama that Bondu will go'

\ex. Bondu táá Baiama-(*mu) wã\\
Bondu go Baiama-FOC \textit{wã}\\
`Bondu will go to Baiama'

\exg. \textipa{\textltailn}ona fene daõ kunu\\
who what ate yesterday\\
`Who ate what yesterday?'

\exg. Bondu-a swéé daõ\\
Bondu-3SG.PST meat ate\\
`Bondu ate meat' (in response to `Who ate what?')

\exg. Bondu-aN-a swéé daõ\\
Bondu-FOC-3SG.PST meat ate\\
`Bondu ate meat'

\exg. Bondu-a swéé-aN daõ\\
Bondu-3SG.PST meat-FOC ate\\
`Bondu ate meat'

\ex. *Bondu-a-na swéé-aN daõ

\ex. *Bondu-a-na swéé-na/mu daõ

\exg. *Swéé-aN Bondu-aN-a daõ\\
meat-FOC Bondu-FOC-3SG ate\\
Intended: `Bondu (was the one who) ate meat'

\ex. Swéé-ni Bondu-a daõ\\
meat-FOC.PST Bondu-3SG ate\\
`It was meat that Bondu ate'

\exg. *Swéé-ni Bondu-aN-a daõ\\
meat-FOC.PST Bondu-FOC-3SG ate\\

\exg. \textipa{\textltailn}ombe tombwe dõ-da kééna?\\
who dance \textit{v}-FUT where\\
`Who will dance where?'

\exg. Bondu tombwe dõ-da Baiama wã\\
Bondu dance \textit{v}-FUT Baiama \textit{wã}\\
`Bondu will dance in Baiama'

\exg. Baiama-mu Bondu tombwe dõ-da\\
\jf{Needs transcription}
`Bondu will dance in Baima'

\ex. *Baiama-mu Bondu-aN tombwe dõ-da\\

\ex. *Bondu-a-na tombwe dõ-da Baiama\\

\ex. Bondu-aN-a tombwe do Baiama\\
Bondu-FOC-3SG dance \textit{v} Baiama\\
`Bondu will dance in Baiama'(?)

\section{Mingyang}
\begin{itemize}
    \item Follow-up on donkey anaphora:

    \mb{Two ways to say `farmer': \textit{sene-sa-mwe}; \textit{sene-t͡ʃe-mwe}}.
    
    \exg. sene-sa-mwe mbɛɛ wuu ambã bɔɔ. (ã) sene-sa-mwe t͡ʃɛ-nu ambe sajboa \#(ana) wuu a/na.\\
    farmer all dog 3.PL? have ANA farmer DEM-PL 3.PL play.with 3.PL dog A\\
    `Every farmer who has a dog plays with the dog.' (Lit. `All farmers have a dog ($\forall > \exists$). These farmers play with their dog.')

    \item More on part-whole bridging:
    \exg. Bondu kongo min-do; \#(ana) mansa jansan.\\
    Bondu village REL-DO 3.PL chief tall\\
    `Bondu lives in a village. The chief is tall.' (Lit. `The village that Bondu lives in - their chief is tall.')

\mb{The demonstrative is not compatible with `the chief' in this case.}
    
Alternatively:
\exg. Bondu kongo min-do; \#(Bondu na kongo) mansa jansan.\\
    Bondu village REL-DO Bondu NA village chief tall\\
    `Bondu lives in a village. The chief is tall.' (Lit. `The village that Bondu lives in - the chief of Bondu's village is tall.')

\exg. Bondu kongo min-do; mwe min-be bondu-a kongo antumu \#(ana) mansa.\\
    Bondu village REL-DO people REL-BE Bondu-A village like 3.PL chief\\
    `Bondu lives in a village. People like the chief.' (Lit. `The village that Bondu lives in - the people in Bondu's village like their chief.')

Alternatively:
\exg. Bondu kongo min-do; mwe min-be no antumu \#(ana) mansa.\\
    Bondu village REL-DO people REL-BE there like 3.PL chief\\
    `Bondu lives in a village. People like the chief.' (Lit. `The village that Bondu lives in - the people there like their chief.')

    \mb{\textit{ana mansa} can co-occur with the demonstrative, but only the exophoric use of the demonstrative is available: \textit{ana mansa t͡ʃɛ} means something like `(pointing) this chief of theirs'.}
    
    \exg. mwe mim-be kongo t͡ʃo o, antumu \#(ana) mansa.\\
        people REL-BE village DEM O like 3.PL chief\\
    Intended elicitation: `Everyone who lives in a village likes the chief. (Co-variation reading)'\\
    Actual sentence: `Everyone who lives in this village likes their chief. (Specific reading)'

\jf{ Could add end "this chief" info}
    \item Relational bridging:
    \exg. n-a gbo nama sã. \#(a) ki ko.\\
        1SG.PST lock new buy 3SG.PST key big\\
        `I bought a new lock. The key is big.' (Lit. `...Its key is big.')

    \exg. n-a gbo nama sã. \#(a) ki Bondu boo.\\
        1SG.PST lock new buy 3SG.PST key Bondu have\\
        `I bought a new lock. Bondu has the key.' (Lit. `...Bondu has its key.')

    \exg. mɔmɔ min-a gbo nama sã, \#(a) ki wã boo.\\
        everyone REL-A lock new buy 3SG.PST key WAN have\\
        `Everyone who bought a new lock has the key.' (Lit. `... has its key.')

\ex. The village has two roads. The intersection is wide
\jf{needs transcription}

 
\end{itemize}


\end{document}