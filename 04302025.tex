\documentclass{assets/fieldnotes}

\title{Kono (Sierra Leone)}
\author{LING3020/5020}
\date{University of Pennsylvania, Spring 2025\\04/30/2025 Class Projects Week 6}

\setcounter{secnumdepth}{4} %enable \paragraph -- for subsubsubsections

\begin{document}

\maketitle

\maketitle
\tableofcontents



 \section{Mingyang}
\subsection{More on anaphoric ã}
\begin{itemize}
    \item Can ã be used in president sentences?
    \exg. sandʒa mansa wuu. jaj mansa ??(t͡ʃɛ) jansã da wã.\\
   this.year chief short next.year chief DEM tall DA WAN\\
   This year, the chief is short. Next year, the same chief will be tall.

    Does the following convey the same meaning?
    \exg. sandʒa mansa wuu. jaj ã mansa jansã da wã.\\
   this.year chief short next.year ANA chief tall DA WAN\\
   This year, the chief is short. Next year, the same chief will be tall.

   \item Can ã be used in part-whole bridging?
    \exg. n-a t͡ʃene sã. \#(a) da kene.\\
        1SG.PST house buy 3.SG door open\\
        I bought a house. \textbf{The door} was open. (Lit. `...its door is open.)

    Does the following convey the same meaning?
    \exg. n-a t͡ʃene sã. ã da kene.\\
        1SG.PST house buy ANA door open\\
         I bought a house. \textbf{The door} was open. (Lit. `...its door is open.)

    \item Can ã be used in relational bridging?
    \exg. n-a gbo nama sã. \#(a) ki ko.\\
        1SG.PST lock new buy 3.SG key big\\
        `I bought a new lock. The key is big.' (Lit. `...Its key is big.')

    Does the following convey the same meaning?
     \exg. n-a gbo nama sã. ã ki ko.\\
        1SG.PST lock new buy ANA key big\\
        `I bought a new lock. The key is big.' (Lit. `...Its key is big.')

    \item Can ã be used in a question?

\exg. Kaine kantea, ja ã kaind͡ʒe somfaŋ\\
(needs glossing)
`There is a boy in the learning place, do you know that boy'

\exg. já kaind͡ʒe somfaŋ ?\\
(Needs glossing)
`Do you know the boy (whole question set here)'
    
\exg. Kandé kant͡ʃe teá, Bondu-a kaind͡ʒe somfaŋ
`There is a student in the place of learning. Does Bondu know the student?'

\item Can be exchanged for 2nd phrase in above statement
\exg. Bondu-a kandet͡ʃe somfaŋ\\
(Needs glossing)
` Does Bondu know the student' 

\exg. Bondu ã kandet͡ʃe somfaŋ\\
(Needs glossing)
`Does Bondu know the student'

\ex. * Bondu ã kande somfaŋ\\
    
\end{itemize}


\section{Daniel}
\exg. Baiama-mu Kai-a tombwe dõ-da. \# Sefadu-mu (wéé) Kai-a tombwe dõ da.\\
Baiama-COP Kai-3SG dance \textit{v}-FUT {} Sefadu-COP also Kai-3SG dance \textit{v}-FUT\\
`It is in Baiama that Kai will dance. It is (also) in Sefadu that Kai will dance.'

\exg. Kai tombwe dõ-da (wã) Baiama ni Sefadu (wéé)\\
Kai dance \textit{v}-FUT wã Baiama and Sefadu also\\
`Kai will dance in Baiama and (also) Sefadu'

\exg. Bondu-aN-a tombwe do. Kai-aN-a *(wé-a) tombwe do (mfã).\\
Bondu-FOC-3SG dance \texti{v} Kai-FOC-3SG also dance \textit{v} FOC?\\
`\textsc{Bondu} danced. \textsc{Kai} also danced.'

\exg. Bondu-a tat\textipa{S}e-t\textipa{S}e i-wã téa\\
Bondu-3SG walk-? 2SG-wã towards\\
`Was it to you that Bondu walked?'

\exg. i-wã téa *(nì) Bondu-a tat\textipa{S}e t\textipa{S}e\\
2SG-wã towards \textit{ni} Bondu-3SG walk ?\\
`It was to you that Bondu walked'

\exg. i téa (ni) Bondu-a tat\textipa{S}e t\textipa{S}e\\
2SG towards \textit{ni} Bondu-3SG walk ?\\
`It was to you that Bondu walked'

\exg. Bondu-a tat\textipa{S}e-t\textipa{S}e i téa\\
Bondu-3SG walk-? 2SG towards\\
`Was it to you that Bondu walked?'

\exg. Bondu-a tat\textipa{S}a-t\textipa{S}a i-wã téa\\
Bondu-3SG walk-? 2SG towards\\
`Was it to you that Bondu walked?'

\exg. i-wã téa mu Bondu \textipa{tatSa} \textipa{tSa} (*wã)\\
2SG-wã towards COP Bondu walk-? wã\\
`Will it be towards you that Bondu will walk?'\\
`It will be towards you that Bondu will walk'\\
\ds{Q or A, in neither case is outer wã available?}

\exg. Bondu-a tombwe do kunũ fã ko jai-mu?\\
Bondu-3SG dance \textit{v} yesterday wã or last.year-COP\\
`Was it yesterday or last year that Bondu danced?'

\exg. Bondu-a tombwe dõ-da sinã-fã ko jai?\\
Bondu-3SG dance \textit{v}-FUT tomorrow-wã or next.year\\
`Will it be tomorrow or next year that Bondu will dance?'

\exg. Bondu-a tombwe dõ-da sinã-wã ko jai?\\
Bondu-3SG dance \textit{v}-FUT tomorrow-wã or next.year\\
`Will it be tomorrow or next year that Bondu will dance?'
\jf{double check this alternation in recording}

\exg. Bondu tweni swee ana\\
(Needs glossing)
`Was it meat that Bondu wanted?'

\exg. Bondu toa wã swee ijaj\\
(Needs glossing)\\
`Will Bondu want meat next year?'

\exg. Bondu toa wã swee ijaj\\
(Needs glossing)\\
`Will Bondu want meat next year?'

\section{Giang}
\exg. B\textipa{\`ond\'u} \textipa{\textltailn\'ond\`o} \textipa{t\'is\'a} \textipa{\`a} \textipa{s\`en\`a}?\\
Bondu who ask \textsc{fut?} tomorrow?\\
``Who will Bondu ask tomorrow?''

\exg. B\textipa{\`ond\'u} \textipa{\`a} \textipa{(w)\'and\`o} \textipa{t\'is\'a} \textipa{\`a} \textipa{s\`en\`a}.\\
Bondu \textsc{3SG.PST} \textsc{foc-obj} ask \textsc{fut} tomorrow\\
``Bondu will ask HIM tomorrow.''

\exg. B\textipa{\`ond\'u} \textipa{\`o} \textipa{t\'is\'a} \textipa{w\'{\~a}} \textipa{s\`en\`a}.\\
Bondu \textsc{obj} ask \textsc{foc} tomorrow\\
``Bondu will ask him tomorrow.''\\
\g{Can be answer to question. Did I also hear "Bondu a o..."????}

\ex. B\textipa{\`ond\'u} \textipa{\`e\'and\`o} \textipa{t\'is\'a} \textipa{w\'{\~a}} \textipa{s\`en\`a}.
Bondu will ask them tomorrow.\\
\g{What the hell did I just hear. Is ando and wang okay in the same sentence now?? and also why is Tony unable to give me an answer without wa. What the HECK?}

\g{a-series object}

\exg. B\textipa{\`ond\'u} \`a \textipa{f\'{\~e}}-\textipa{n\`a} \textipa{t\'{\~o}} \textipa{k\`un\`u}?\\
Bondu \textsc{3SG} what-\textsc{foc} close yesterday\\
``What did Bondu close yesterday?''

\exg. B\textipa{\`ond\'u} \`a \textipa{k\'aNg\'an-\`a} \textipa{t\'{\~o}} \textipa{k\`un\`u}.\\
Bondu \textsc{3SG.PST} door-\textsc{obj} close yesterday\\
``Bondu closed the door yesterday.''

\exg. B\textipa{\`ond\'u} \`a \textipa{k\'aNg\'an\`e} \textipa{\'an\`a} \textipa{t\'{\~o}} \textipa{k\`un\`u}.\\
Bondu \textsc{3SG.PST} door \textsc{foc} close yesterday\\
``Bondu closed the door yesterday.''

\exg. B\textipa{\`ond\'u} \`a \textipa{k\'aNg\'an\`e}-\textipa{tS\`a} \textipa{t\'{\~o}} \textipa{k\`un\`u}.\\
Bondu \textsc{3SG.PST} door-\textsc{dem-obj} close yesterday\\
``Bondu closed this/that door yesterday.''

\exg. B\textipa{\`ond\'u} \`a \textipa{k\'aNg\'an\`e}-\textipa{tS\`e} \textipa{\'an\`a} \textipa{t\'{\~o}} \textipa{k\`un\`u}.\\
Bondu \textsc{3SG.PST} door-\textsc{dem} \textsc{foc} close yesterday\\
``Bondu closed this/that door yesterday.''



\exg. B\textipa{\`ond\'u} \`a \textipa{t\'{\~o}} \textipa{k\`un\`u}.\\
Bondu \textsc{3SG.PST} close yesterday\\
Bondu closed it yesterday.\\
\g{The a-series doesn't have the object marker when there isnt a lexical object, I think. He also said this can be an answer, but he didn't put any focus anywhere.}

\g{object-wh, o-series}

\exg. B\textipa{\`ond\'u} a \textipa{f\'e\~{\'e}} \textipa{d\`ot\'e} \textipa{k\`un\`u}\\
Bondu \textsc{\textsc{3sg}} what break yesterday\\
What did Bondu break yesterday?

\exg. B\textipa{\`ond\'u} a \textipa{s\'E\'E} \textipa{\'En} \textipa{d\`ot\'e} \textipa{k\`un\`u}\\
Bondu \textsc{\textsc{3SG.PST} see \textsc{foc} break yesterday\\
``Bondu broke \textsc{the see} yesterday.''

\exg. B\textipa{\`ond\'u} \`o t\'e k\`un\`u.\\
Bondu \textsc{obj} break yesterday\\
``Bondu broke it yesterday.''\\
\g{WHERE IS THE FOCUS?}


\section{Jan}

\exg. e(i) o te a wã sina \\
2SG.NPST 3SG.OBJ cut AUX FUT tomorrow \\
`You will break it tomorrow.'

\exg. Wo te a wã sina\\
(Needs glossing)
`You all will break it tomorrow.'


\exg. Wo teya/ Wo te kunu\\
(Needs glossing)
`You all broke it yesterday.'

\exg. meme i te kunu\\
(Needs glossing)
`The mirror cut you yesterday.'

\exg. meme o wo te kunu\\
(Needs glossing)
`The mirror cut you yesterday.'

\item Can't say 'meme o e te kunu'

\exg. meme i tea\\
(Needs glossing)
`The mirrors are cutting you.'

\exg. meme wo te a
(Needs glossing)
`the mirror is cutting multiple people'

\exg. meme o te a wã sina\\
(Needs glossing)
`The mirror will cut you tomorrow.'

\exg. meme mbe o te a wã sina\\
(Needs glossing)
`The mirrors will cut you tomorrow.'

\exg. meme mbe i teya wã sina\\
(needs glossing)
`The mirrors will cut you'

\exg. Bondu ŋm-gbò kunu\\
(Needs glossing)
`Bondu cursed me yesterday'

\exg. io te Bondu ŋm-gbò kunu\\
(Needs glossing)
`You said Bondu cursed me yesterday.'

\exg. ja fɔ̀-n-d͡ʒè Bondu ŋm-gbò kunu\\
(Needs glossing)
`You told me Bondu cursed me yesterday'

\exg. Wo te Bondu ŋm-gbò kunu \\
(Needs glossing)
` Y'all said Bondu cursed me yesterday.'

\exg. Wa fɔ̀-n-d͡ʒè Bondu ŋm-gbò kunu.\\
(Needs Glossing)
`Y'all told me Bondu cursed me'

\exg. Ando te Bondu ŋm-gbò kunu\\
(Needs glossing)
`They said Bondu cursed me yesterday.'


\jmt{`Let him teach you.'}

\exg. ani kã\\
(Needs glossing)
`Let him teach you.'

\exg. ni kã
(Needs glossing)
`Let me teach you.'

\exg. a kã
(Needs glossing)
`Teach him!'

\exg. Saa kandie kã
(Needs glossing)
`Saa, Teach him!'

\exg. Bondu an kã
(Needs glossing)
`Bondu, Teach them!'

\ex. Bondu n'kã
(Needs glossing)
`Bondu, teach me'


[n] + [i]

\exg. Kone i te\\
(needs glossing)
`The tree cut you.'

\exg. koneni te\\
(Needs glossing)
`let the tree cut you'

\item *kon te, not acceptable



\section{Lex}

\jf{testing nasal coda as part of word stem or suffix}

\exg. n-á fìán ɔ̀ káí\\
1SG.PST broom    OBJ break\\
`I broke the broom'

\exg. n-á fìán ndɔ̀ káí\\
1SG.PST broom   3Pl.OBJ break\\
`I broke the brooms'

\exg. n-á fìá-jàwá ɔ̀ káí\\
1SG.PST broom-red OBJ break\\
`I broke the red broom'

\exg. n-á fìá-jàwá ndɔ̀ káí\\
1SG.PST broom-red 3PL.OBJ break\\
`I broke the red brooms'

\exg. n-á fìám-ba ɔ̀ káí\\
1SG.PST broom-big 3SG.OBJ break\\
`I broke the big broom'

\exg. n-á fìá-tʃénàmà ɔ̀ káí\\
1SG.PST broom-big 3SG.OBJ break\\
`I broke the big broom'

\exg. n-á fìá-wá ɔ̀ káí\\
1SG.PST broom-big 3SG.OBJ break\\
`I broke the big broom'

\exg.  n-á fìám-bá ndɔ̀ káí\\
 1SG.PST broom-big 3PL.OBJ break\\
`I broke the big brooms'

\exg.  n-á fìá-tʃénàmà-nù ndɔ̀ káí\\
 1SG.PST broom-big-PL 3PL.OBJ break\\
`I broke the big brooms'

\jf{Tony attempted to say 'ná fìámbanu ándɔ̀ káí' but ended up correcting to ná fìám-bá ndɔ̀ káí in the end. }
\jf{Does this result in a double plural marker?}

\exg.  n-á fìá-wá ndɔ̀ káí\\
 1SG.PST broom-big 3PL.OBJ break\\
`I broke the big brooms'

\jk{you can use this sentence frame to systematically manipulate plural, j vs. w. For example, big broom vs. red broom, brooms vs. broom}
\ex. sénɛ̀
`stone'

\jf{Trying to find other adjective that starts with j, to strengthen 'red' evidence}


\exg. sénɛ̀ ɔ̀ tɛ́-à\\
stone 3SG.OBJ broke-A\\
`The stone broke'

\exg. sénɛ̀ ndɔ̀ tɛ́-à\\
stone 3PL.OBJ broke-A\\
`the stones broke'

\exg. n-á sénɛ̀ ɔ̀ tɛ́\\
1SG.PST stone 3SG.OBJ break\\
`I broke the stone'

\exg. n-á sénɛ̀ ndɔ̀ tɛ́\\
1SG.PST stone 3PL.OBJ break\\
`I broke the stones'

\exg. n-á sén-dʒámá ɔ̀ tɛ́\\
1SG.PST stone-red 3SG.OBJ break\\
`I broke the red stone'

\exg. n-á sén-dʒámá ndɔ̀ tɛ́\\
1SG.PST stone-red 3PL.OBJ break\\
`I broke the red stones'

\ex. kenɛ
` musical instrument'

\exg. kénɛ̀ ɔ̀ tɛ́-à\\
kene 3SG.OBJ broke-A\\
`The kene broke'

\exg. kén ɔ̀ tɛ́-à\\
kene 3SG.OBJ broke-A\\
`The kene broke'

\exg. kénɛ̀ ndɔ̀ tɛ́-à\\
kene 3Pl.OBJ broke-A\\
`The kenes broke'

\exg. n-á kén ɔ̀ tɛ́\\
1SG.PST kene 3SG.OBJ break\\
`I broke the kene'

\exg. n-á kénɛ̀ ndɔ̀ tɛ́\\
1SG.PST kene 3PL.OBJ break\\
`I broke the kenes'

\exg. n-á kénjáwá ɔ̀ tɛ́\\
1SG.PST kene-red 3SG.OBJ break\\
`I broke the red kene'

\exg. n-á kénjáwá ndɔ̀ tɛ́\\
1SG.PST kene-red 3PL.OBJ break\\
`I broke the red kenes'

\jf{Understandable w/ kéndʒáwá, but doesn't love it and emphasizes the natural pronunciation}

\exg. n-á kénwá ɔ̀ tɛ́\\
1SG.PST kene-big 3SG.OBJ break\\
`I broke the big kene'

\exg. n-á kén-tʃénàmà ɔ̀ tɛ́\\
1SG.PST kene-big 3SG.OBJ break\\
`I broke the big kene'

\exg. n-á kém-bá ɔ̀ tɛ́\\
1SG.PST kene-big 3SG.OBJ break\\
`I broke the big kene'

\jf{Anthony said understandable and works, but to be cafeful with macho meaning}

\exg. n-á kén-tʃénàmà-nù ndɔ̀ tɛ́\\
1SG.PST kene-big-PL 3PL.OBJ break\\
`I broke the big kenes'

\jf{I think the above one he  said to give me the plural example}

\jf{}

\ex.jɛlɛn\\
`slow,slowly, quielty' 

\jf{Usually used for people, describing actions, I stretched it hear for pronunciation purposes}

\exg. n-á séŋ-jɛ́lɛ́ ɔ̀ tɛ́\\
1SG.PST stone-slow 3SG.OBJ break\\
`I broke the slow stone'

\exg. n-á séŋ-jɛ́lɛ́ ndɔ̀ tɛ́\\
1SG.PST stone-slow 3Pl.OBJ break\\
`I broke the slow stones'

\exg. n-á kéŋ-jɛ́lɛ́ ɔ̀ tɛ́\\
1SG.PST kene-slow 3SG.OBJ break\\
`I broke the slow kene'

\jf{Anthony said inserted dʒɛ́lɛ́ would not be understood}

\exg. n-á kɛ́nɛ́ sá jɛ́lɛ́\\
1SG.PST kene lay slowly\\
`I laid the kene slowly'

\jf{Above example is a natural use of the word jɛlɛn }



\section{Alex}

\item[ Context: Bondu like to do many things for fun, like break bottles. She breaks bottles for hours. Someone asks, "What did Bondu do yesterday?"]

\exg. Bondu-a sani n'fando te kunu.\\
(Needs glossing)
`Bondu broke bottles yesterday'

\exg. Bondu-a sani ndo te kunu.\\
(Needs glossing)
`Bondu broke bottles yesterday'

\exg. Bondu-a sani ndo te kunu, hour dondo.\\
(needs glossing)
`Bondu broke bottles yesterday for an hour.'

\exg. Bondu-a sani o te kunu, hour dondo\\
(needs glossing)
`Bondu broke 1 bottle for an hour'

\exg. * Bondu-a sani ndo teya kunu\\
(Needs glossing)
` Doesn't mean broke, means actively breaking,'

\item[ Context: I had 10 bottles that I wanted to use for a party. I'm looking for them, and I just saw them 30 minutes or 2 hours ago. I can't find them. Then someone tells you...]

\exg. Bondu ja sani ndo te (Ja sani ndo te)\\
(Needs glossing)
` Sorry, Bondu broke the bottles'

\exg. Bondu-a sanit͡ʃe ndo te kunu\\
(Needs glossing)
` Sorry, Bondu broke the bottles'

\exg. Bondu-a sani t͡ʃendo te kunu, hour dondwe n'koa\\
(Needs glossing)
`It took Bondu 1 hour to break the bottles'

\item[Context: I closed the door.]

\exg. na kanganena toŋ\\
(needs glossing)
` I closed the door'

\exg. Na kangana toŋ, hour dondwe koa\\
(Needs glossing)
`I closed the door for an hour'

\exg. Na kangana tonda, hour dondwe koa\\
(needs glossing)
`My door closed for 1 hour'


\section{Joey}



\subsection{Negation with te:}

\jf{NOTE: We had a hard time eliciting negatives with te, which might be something we need to try to do again to see if the fo info is worth using.}\\


\exg. ǹ-dɔ́ té bòndú á swèè sàŋ kùnù\\
1SG-PST.3SG.OBJ Bondu 3SG.PST buy yesterday\\
`I said that Bondu bought meat yesterday' \\

\exg. ní má fɔ̀-ní bòndú á swèè sàŋ kùnù\\
1SG NEG say-NEG Bondu 3SG.PST buy yesterday\\
`I didn't say that Bondu bought meat yesterday' \jf{(Impossible with te)}\\



\subsection{Non-CP DOs:}

\exg. tàí\\
'story` \\

\exg. bàí\\
'word` \\

\exg. bòndú á tàí fɔ̀-n-d͡ʒè\\
Bondu 3SG.PST story say-1SG-to\\
`Bondu told (me) the story' \jf{(fo)}\\

\exg. bòndú á tàí ɔ́ ìjàn-à\\
Bondu 3SG.PST story OBJ explain-A\\
`Bondu told the story' \jf{Explain is more common predicate here}\\

\exg. ìjã̀\\
'explain` \\


\exg. bòndú á bàí-ɱ fɔ̀(-n-d͡ʒè)\\
Bondu 3SG.PST word-PL say(-1SG-to)\\\
`Bondu said the words (to me)' \jf{(fo)}\\


\exg. íwán á bàí-ɱ fɔ̀-n-d͡ʒè\\
2SG.STRONG 3SG.PST word-PL say-1SG-to\\\
`you told me the words' \jf{(fo)}\\


\exg. bòndú ɔ́ té\\
Bondu 3SG.OBJ say\\
`Bondu said (implied: the words)' \jf{(te - not possible to have this with DP DO, apparently)}\\

\exg. bòndú á bàí ɔ́ té\\
Bondu 3SG.POSS word 3SG.OBJ say\\
`Bondu's word(s) said' \jf{(te - can't say, Bondu said the words)}\\

\exg. bòndú á tàí ɔ́ té\\
Bondu 3SG.POSS story 3SG.OBJ say\\
`Bondu's story said' \jf{(te)}\\

\exg. j-á bàí-n àmí\\
2SG-PST word-PL hear\\
`you heard the words' \\

\exg. j-á tàí àmí\\
2SG-PST story hear\\
`you heard the story' \\

\exg. tìsá\\
`question' \\

\exg. j-á tìsá-t͡ʃè\\
2SG-PST question/ask-TSE\\
`you asked the question' \\

\exg. íwán á tìsá-t͡ʃè\\
2SG.STRONG 3SG.PST question/ask-TSE\\\
`you asked the question/It was you who asked the question' \\

\exg. j-ɔ̀ tìsá\\
2SG-PST.3SG.OBJ ask\\
`you asked it' \\

\exg. j-á fènè ɔ̀ tìsá\\
2SG-PST thing 3SG.ɔSER ask\\
`you asked it' \\

\exg. j-á tìsá ɔ̀ tìsá\\
2SG-PST question 3SG.ɔSER ask\\
`you asked the question (a question)' \jf{(ie: You asked a question to the abstract conceptualization of a question)}\\


\exg. ìnàt͡ʃìì\\
`the thought' \\

\exg. j-á ìnàt͡ʃìì á t͡ʃìì\\
2SG-PST thought/think A TSII\\
`you thought the thought' \\

\exg. bòndú á ìnàt͡ʃìì á t͡ʃìì\\
Bondu 3SG.PST thought/think A TSII\\
`Bondu thought the thought' \\


    
\section{Wesley}

\wml{Binding---checking to see if the correlative is base-generated high (Bhatt 2003); if so, then we predict the following binding relations since Bondu would not C-command he and himself at any point:}

\exg. Bondu-a awandi je meme-o \\
(Needs glossing)
`Bondu saw himself in the mirror'

\exg. Bondu-a meme o te (a) ana wandie mind-o\\
(Needs Glossing)\\
`Bondu broke the mirror that he saw himself in.'

\exg. Saa, Kai, Safa, Tambam Kombam Eja \\
`Male Kono Names'

\exg. Ejaj, Bondu, Finda, Kumbam Fea\\
`Female Kono Names'

\exg. Bondu-a wandie meme mind-o, a o te\\
(Needs glossing)
` The mirror that Bondu saw herself in, she broke it.'

\exg *A wandi meme je mind-o, Bondu o te\\
(Not glossing)
`the mirror that Bondu saw herself in, she broke it.'

\exg. A wandie meme min-do, Bondu o te\\
(Needs glossing)
` He saw himself in the mirror, then Bondu broke it.'

\exg. Bondu a wandie meme-o-mindo, mwe o te\\
(Needs glossing)
`He saw himself in the mirror, then someone else broke it.'

\item if ended in a o te - could've been Bondu or someone else



\wml{Testing PPs with REL-DP complements---in Southeastern Mande languages, the REL-DP can be clause-internal, i.e., before the PP (Nikitina 2012). Can this be done in Kono, a West Mande language?}

\exg. Bondu-a jabasi t͡ʃè dao\\
(Needs glossing)
`Bondu put the onion in the pot'

\exg. Kai-a da saŋ
(needs glossing)
` Kai bought the pot'

\exg. Kai-a da mi saŋ, Bondu-a jabasi t͡ʃo\\
(Needs glossing)
`Bondu put the onion in the pot that Kai bought'

\exg. Bondu-a jabasi t͡ʃe da a mind-o, Kai ana saŋ\\
(Needs glossing)
`Bondu put the onion in the pot that Kai bought'

\exg. Bondu-a jabasi t͡ʃe dao, Kai-a mi saŋ\\
(Needs glossing)
`Bondu put the onion in the pot that Kai bought'

\exg. * Bondu-a jabasi t͡ʃe dao, Kai-a da mi saŋ t͡ʃo/o\\
(Needs glossing)
`Bondu put the onion in the pot that Kai bought'

\wml{Check if this sentence is ok---predicted ungrammatical}

\exg. ?Bondu à {[} úú mín-à sɛ̀ɛ̀ ɔ̀ té {]} jẽ̀ẽ̀ kunu tʃ\'{e}n-\`{a}\\
Bondu \textsc{3SG.PST} {} dog \textsc{mi-aux} sɛɛ \textsc{3sg.obj} break {} see yesterday house-in\\
`Yesterday, in the house, Bondu saw the dog who broke the sɛɛ.’

\exg. Kunu, t͡ʃena Bondu-a wuuje mina see o te\\
(Needs glossing)
`Yesterday, in the house, Bondu saw the dog who broke the sɛɛ.’

\exg. Wuu mina se o te, Bondu-a jẽ̀ẽ̀ kunu t͡ʃena\\
(Needs glossing)
`Yesterday, in the house, Bondu saw the dog who broke the sɛɛ.’

\exg. Bondu-a jẽ̀ẽ̀ t͡ʃena kunu wuu mina, see o te \\
(Needs glossing)
`Yesterday, in the house, Bondu saw the dog who broke the sɛɛ.’

\exg. Bondu-a wuu mina see o te, kunu t͡ʃena. \jf{ Grammatical, but could be referencing any house not just Bondu's}
(Needs glossing)
`Yesterday, in the house, Bondu saw the dog who broke the sɛɛ.’

\section{Chun-Hung} 

\chs{\textbf{Note: I can take lesser time this week (if needed), as I ran over by almost 5 minutes last week.}} \newline

\chs{\textbf{0. Some checking}} \newline

\chs{1. morpheme \textit{mb-a}: to see whether \textit{mb-a} is a combination of \textit{mbe} + \textit{wa}}

\exg. W\'{u} \`{m}b\'{e} w\`{a} B\`{o}nd\'{u}-b\'{o}\`{o}. \\ 
dog PL WA Bondu-hand \\
`Bondu has dogs.' \chs{predicative possessives}

\exg. *W\'{u} \`{m}b-\^{a} B\`{o}nd\'{u}-b\'{o}\`{o}. \\ 
dog PL-A Bondu-hand \\
`Bondu has dogs.' \chs{mbe and wa cannot be contracted}

\exg. K\`{a}nd\'{i}\`{e} mb\'{e} w\'{a} ã̀-\textipa{\textltailn}ṍ-b\'{o}\`{o}. \\
student PL WA 3PL.SER1-each.other-hand \\
The students have each other. (in the context of `They can support each other.')

\exg. K\`{a}nd\'{i}\`{e} mb-\^{a} \textipa{\textltailn}ṍ-b\'{o}\`{o}. \\
student PL-A each.other-hand \\
The students have each other. (in the context of `They can support each other.') \chs{The contraction may be derived with 3PL.SER1 and mbe when the wa is omitted}

\chs{1. Condition C --- the violation is seen on a par with transitives, but more restrictions are found with predicative possessives. I'm going to check the following sentences again.}

\exg. B\`{o}nd\'{u}\textsubscript{i}  (\textipa{\`{E}}) \`{a}\textsubscript{i}-k\'{o}-t\textipa{S}\`{\textipa{E}}n\`{a}m\'{a}-b\'{o}\`{o}. \\
Bondu (3SG.SER3) 3SG.POSS-brother-big-hand \\
`Her\textsubscript{i} older brother has Bondu\textsubscript{i}.' = `Bondu\textsubscript{i}'s older brother has her\textsubscript{i}.' \chs{Unlike literal English translation, no Condition C violation is present in Kono, which suggests the possessum is higher.}

\exg. \`{A}\textsubscript{*i/k}-k\'{i} w\`{a} B\`{o}nd\'{u}\textsubscript{i}-b\'{o}\`{o}. \\
3SG.POSS-key WA Bondu-hand \\
`Bondu\textsubscript{i} has her\textsubscript{*i/k} keys.' \chs{Tony may not like R-expressions are weakly bound, similar to transitive clause above.}

\exg. B\`{o}nd\'{u}\textsubscript{i}-\`{a}-k\'{i} w\'{a} \`{a}\textsubscript{i/*k}-b\'{o}\`{o}. \\
Bondu-A-key WA 3SG.POSS-hand \\
`She\textsubscript{i/*k} has Bondu\textsubscript{i}'s keys.' 
= `Bondu has her own keys.' \chs{It is surprising that the pronoun cannot refer to someone else (unless the possessor is structurally higher at some point of derivation), and this needs to be double-checked and be asked how to say `He/She\textsubscript{k} has Bondu\textsubscript{i}'s key.'}

\exg. B\`{o}nd\'{u}\textsubscript{i}-\`{a}-k\'{i} \textipa{\`{E}} w\'{a} \`{a}\textsubscript{i/k}-b\'{o}\`{o}. \\
Bondu-A-key 3SG WA 3SG.POSS-hand \\
`He/she\textsubscript{i/k} has Bondu\textsubscript{i}'s key.' (Context: Bondu asks us to take something in her house, and we arrive at her house. However, I forgot to take Bondu's key with me, and my friend pointing to other guy says that: Don't worry; he/she has Bondu's key.) \chs{I'm not sure how this sentence compared with the above can make the reference different, espeically in both sentences `Bondu' doesn't seem to be bound.}

\exg. \`{\textipa{E}}\textsubscript{*i/k} B\`{o}nd\'{u}\textsubscript{i}-k\'{o}-t\textipa{S}\'{\textipa{E}}n\`{a}m\`{a}-b\'{o}\`{o}. \\
3SG.SER3 Bondu-brother-big-hand \\
`Bondu\textsubscript{i}'s older brother has it\textsubscript{*i/k}.' \chs{The pronoun is interpreted most likely as an inanimate object and also as a baby according to Tony, but even if Bondu is a baby, the pronoun cannot refer to Bondu, and hence the Condition C violation may be present.}

\exg. B\`{o}nd\'{u}\textsubscript{i}, \`{a}\textsubscript{k}-k\'{i} w\'{a} \`{a}\textsubscript{i}-b\'{o}\`{o}. \\
Bondu A-key WA 3SG.POSS-hand \\
`Bondu\textsubscript{i} has her\textsubscript{k} keys.' \chs{This sentence is identical to (7) but with a unexpected interpretation, and I can only think of topicalization (with the segmentation here) that may derive this meaning, but also needs double-checked.}


\chs{\textbf{B. Covert postposition in predicative possessives} --- to test whether there's a covert postposition or the postposition \textit{o} `in' (thanks to Wesley) after `hand' in predicative possessives or it's `hand' that serves as a postposition} \newline

\chs{0. Baseline} 

\exg. D\`{u}m\'{i}\`{i} (w\'{a}) \`{m}-b\'{o}\`{o}. \\
orange (WA) 1SG.POSS-hand \\
`I have an orange.'  \\
`The orange is in my hand.'

\exg. D\`{u}m\'{i}\`{i} sṍ \`{m}-b\'{o}\`{o}. \\ 
orange lay 1SG.SER1-hand \\
`The orange is in my hand.' \chs{The word sṍ  comes from sa `lay', but I don't know what the vowel change it is here.}

\exg. D\`{u}m\'{i}\`{i} (*w\'{a}) \`{m}-b\'{o}\`{o}-k\`{u}m\`{a}. \\
orange (*WA) 1SG.SER1-hand-on \\
`The orange is on my hand.' \chs{I don't know why wa is not okay here.}

\chs{1. Plural marking}

\ex. I have oranges. (so many oranges that I have that I may need to carry them by both hands)

\exg. D\`{u}m\'{i}\`{i} (w\'{a}) \`{m}-b\'{o}\`{o}-f\textipa{\'{E}}\`{A}. \\
orange (WA) 1SG.SER1-hand-two \\
`The orange is in both of my hand.'

\chs{2. Demonstratives}

\ex. The orange is in this hand of mine. 


\chs{\textbf{C. Predicative possessives, PP predication, NP predication} --- to argue that predicative possessives pattern with PP predication (in line with Creissels (2025) who described `hand' as an adposition) not NP predication} \newline

\chs{0. Baseline}

\exg. W\'{u} w\`{a} B\`{o}nd\'{u}-b\'{o}\`{o}. \\ 
dog WA Bondu-hand \\
`Bondu has a dog.' \chs{predicative possessives}

\exg. W\'{u}  B\`{o}nd\'{u}-t\'{e}\`{a}. \\ 
dog Bondu-with \\
`The dog is with Bondu.' \chs{PP predication}

\exg. W\'{u} m\`{u} B\`{o}nd\'{u}-\`{a}-s\'{o}f\'{e}n(\`{e})-ǎn(\`{e})-\`{a}. \\
dog COP Bondu-3SG.POSS-pet-ANE-A \\
`The dog is Bondu's pet.' \chs{NP predication}

\end{document}