\documentclass{assets/fieldnotes}

\title{Kono (Sierra Leone)}
\author{LING3020/5020}
\date{University of Pennsylvania, Spring 2025\\02/05/2025 Phonetics and Phonology II: Tone }

\setcounter{secnumdepth}{4} %enable \paragraph -- for subsubsubsections

\begin{document}

\maketitle
\tableofcontents

\newpage

Potential tonal targets: H, L, F, HH, LL, LF, HF

\section{CV (compared with CV:) --- Giang and Wesley}

%\wml{We have a few minimal and near-minimal pairs ready; we will try to elicit these as individual words and then ask him to say them in sequence (both orders).}

%\wml{For those incomplete pairs, we will ask him for the word that we \textit{do} have and then either lengthen or shorten the vowel to ask if that gives a new word.}

%\jal{remember as a pair you need enough material to elicit for 20min. It's fine to ask him if he can think of a word sometimes, but we don't want to rely on that too heavily.  The dissertations should also have some words to pursue.}

%\wml{Thank you, Julie! Sorry, we had more material that we compiled offline but hadn't added here.}

%\g{I've listed the data (mainly) in order of elicitation here, with minimal pairs noted. I've also kept the original table we prepared. Please remove it if it bothers you.}

\ex. dà\\
`an opening' \g{Tony: any open space, like your mouth}

\exg. cénɛ̂ dà\\
house opening\\
`door of your house/home'\\
\g{Tony also gave us another word for `door' káŋɡánɛ̂}

\ex. dàà\\
`pot'\\
\g{Minimal pair with `an opening'.}

\ex. táà\\
`calabash'\\
\g{Not sure if this is low or falling...}

\ex. tâ\\
`fire'\\
\g{Note that the diss. has this as dáá.}

\ex. dí\\
`sweet'

\ex. díì(â)\\
`to get used to something'\\
\g{I am extremely uncertain of how to transcribe this.}

\ex. fà\\
`father'
\wml{Anthony says you might use f\'{a}\'{a} if you are calling your father from afar.}

\ex. fàà\\
`heart'\\
\g{Minimal pair here for vowel length contrast.}

\ex. fá\\
`swollen', `eternity'\\
\g{Another minimal pair, in terms of tones! Tony added an "a" in front of these words sometimes, might be relevant for morphology later.}

\ex. kòò\\
`big'\\
\g{Could also say wà. I'm not sure about the tone. Tony says it drags here but I don't know if it's a long vowel or a falling/low tone, or both.}
\wml{He says it drags, and I do think it's a low tone too.}

\ex. kó\\
`shower'

\ex. màà\\
`banana'

\exg. dàkà\\
mouth open\\
`open door'\\
\g{At first, Tony gave this as the translation for `open,' then he analyzed it as having dà `opening' that we learned right before, plus kà `to open.'}

\ex. ɲá(n\`{e})\\
`make', `to be rich'

\ex. kàà\\
`snake'\\
\g{Minimal pair with open}

\ex. kà\\
`open'\\

\ex. sá\\
`put'

\ex. sàà\\
`Sahr' (a name given to the first son of a mother)\\
\g{Tony's name! First son of a mother is Sahr! Though I'm not sure if we have a length contrast here...}

\ex. sí\\
`music'\\
\g{Tony said this has a higher tone!}

\ex. sì\\
`sit'\\Got invited to USC visit the wknd after a conference that he's attending so I want 
\g{Tony said neither of these have a dragged vowel}

\ex. sìì\\
`go into a relationship'\\
\g{He said this is a dragged version of `sit.'}

\ex. cé\\
`this'

\ex. céé\\
`husband'

\ex. céè\\
`war'\\
\g{He said this sounds NEARLY like `husband' but not entirely, and he said the difference is towards the ending side! Amazing! `War' has a falling tone while `husband' has a high tone throughout the long vowel.}

\ex. cì\\
`tie'

\ex. cíí\\
`egg''\\
\g{Again, falling tone or just end-of-utterance contour?}

\ex. tû\\
`oil'

\ex. tóò\\
`ear'/`jealousy'\\
\g{He said it drags but not really. I hear that it's longer than `oil.'}

\ex. úù\\
`dog'

\ex. kúndú\\
`short'

\ex. kúndúú\\
`iron'
\g{Length contrast on second syllable with previous word.}
%And we didn't even plan this one. How wonderful.

\ex. ú\\
`short'

\ex. bùù\\
`farm'

\ex. bù\\
`stomach'\\
\g{Minimal pair detected.}

\wml{Sentences:}

\ex. n-d\'{i}\`{i}-\'{a}\\
\textsc{1sg}-get.used.to-\textsc{obj}\\
`I'm used to it.'

\ex. n-d\'{i}\`{i}-\'{a} ma\ng gw-a\\
\textsc{1sg}-get.used.to-\textsc{obj} mango-\textsc{obj}(?)\\
`I'm used to mango.'

\ex. n-d\'{i}\`{i}-\'{a} putu-a\\
\textsc{1sg}-get.used.to-\textsc{obj} pepper-\textsc{obj}(?)\\
`I'm used to pepper.'

\ex. m-be k\'{o}a\\
1\textsc{SG}-NPST shower\\
`I'm taking a shower.'

\ex. mba \textltailn \'{a}nda\\
? ?\\
`I'm making something.'\\
\g{mba seems like a variation of 1\textsc{sg}, and second part seems to be some conjugated form of the verb `to make' ɲá(n\`{e}).}

\ex. mb\'{e} f\'{e}n\`{e} \textltailn\`{a}nda\\
1\textsc{SG}-NPST something make\\
`I'm making something.'

\ex. mb\'{e} p\'{u}s\'{u} \textltailn\`{a}nda\\
1\textsc{SG}-NPST sauce make\\
`I'm making sauce.'


\begin{table}[h!tbp]
    \centering
    \begin{tabular}{c|c|l}
    \hline
    \textbf{CV}	& \textbf{CVV}	& \textbf{Notes} \\
    \hline
    & bùù `farm' & \\
    \hline
    dà `door' & dáá `fire'	& \\
    \hline
    di `sweet' &	dii `get used to' & \\
    \hline
    fa `father' & faa `heart' & \\	
    \hline
    &faa `death'& \\
    \hline
    & faa 'swollen' & \\
    \hline
    gbo 'big' & gbòò `padlock'&\\
    \hline
    já `eye'& jáá `peanut'&\\	
    \hline
    kà `open' & kàà `snake'&\\
    \hline
    & koo `big'&\\	
    \hline
    má `make' & máá `banana'&\\
    \hline
    sa `put' & saa `a name' & \makecell{Should probably elicit 'put'\\and then lengthen to ask if "Saa" exists}\\
    \hline
    si `music' & sii `sit' & \\	
    \hline
    tâ `fire' & táà `calabash'&\\	
    \hline
    tè `break', `story' & &\\
    \hline
    té `day', `say' & &\\
    \hline
    &tʃee `see' & \\
    \hline
    tʃɛ́ `this' & tʃɛ́ɛ́ `husband', `war' &\\	
    \hline
    tʃi `tie' & tʃii `egg' &\\	
    \hline
    tû `oil' & túù `ear' &\\
    \hline
    u `short' & uu `dog' &\\
    \hline
    \end{tabular}
    \caption{Planned/possible minimal pairs, not elicited data}
    \label{tab:my_label}
\end{table}

\section{CV: (compared with CVV) -- Alex and Mingyang}
\begin{itemize}
    \item ai, ia, aa, ii
        \ex. jai\\
            `name of the fourth child'
            
        \ex. jaa\\
            `lion'

        \ex. jii\\
            `water'

        \ex. wái\\
            `resting place in a farm'

        \ex. wàí\\
            `work'
        
        \ex. wáà\\
            `mat'

        \ex. wii\\
            `blood'

        \ex. swia\\
            `to suspect'

        
    \item ua, uu, aa
        \ex. wua\\
            `big'

        \ex. wuu\\
            `dog'

        See waa `mat' in the last bullet point.

        \ex. tua\\
            `to hit it'

        \ex. tuu\\
            `fat'

        \ex. táá\\
            `to go out' (cf. `fire')

        \ex. tàá\\
            `calabash'

    \item uɛ, ɛɛ, ɛa
        \ex. súɛ́\\
            `beans'

        \ex. sɛ́ɛ́\\
            `musical instrument'

        \ex. sɛá\\
            `to climb it'

    \item ɔa, ɔɔ, aa
        \ex. kɔa\\
            `monkey'
        
        \ex. kɔɔ\\
            `to become fat'

        \ex. kàâ\\
            `snake'

        \ex. káà\\
            `gun'

    \item ui, uu, ii, ia
        \ex. fuinɛ\\
            `desert'

        \ex. fìì\\
            `to mistake a person'

        \ex. fìâ\\
            `wind'

        \ex. fíá\\
            `to sweep'

        \ex. fíâ\\
            `bush'

        \ex. fúù\\
            `to marry'

    \item ea, aa
        \ex. fèà\\
            `two' (cf. `to inflate it')

        \ex. féà\\
            a girl's name

        \ex. fáá\\
            `heart'

        \ex. fèè\\
            `sifter'

    \item oa, oo
        \ex. gbóà\\
            `to curse it'

        \ex. bóó\\
            `hand'
        
\end{itemize}

\section{CVN (compared with CV) -- Daniel and Joey}
\ds{Look at the (near)-minimal pairs first to establish existence of CVN syllables, then ask if non-minimal pair examples can be modified to involve CV/CVN counterpart, finally try to test syllabification (looking for how ŋ(g) behaves in particular)}

\jf{NB: I'll be using ȁ, à, ā, á, a̋, â, ǎ to indicate extra low (1), low (2), mid (3), high (4), extra high (5), falling, and rising tones below}

\ex. kō\\
`to wash'

\ex. kòő\\
`to grow big'

\ex. kóŋkóŋ\\
`knock-knock'

\ex. kònɛ̄\\
`tree'

\ex. tó\\
`to stay here'

\ex. tòné\\
stay.here.IMP\\
`stay here!'

\jal{use Leipzig glossing rules; I assume you meant imperative so I changed it}

\ex. tónē\\
`process of burial'

\ex. tǒŋ \jf{(for the rising tone, I had the tones noted as 2 and 4)}\\
`bury/close'

\ex. t͡ʃǐ\\
`to tie'

\ex. dāːùn\\
`to eat'
\ds{I think he also pronouned it as \textit{dóŋ} at some point, not sure about the diphthong}

\ex. dɔ́n\\
`eating process'

\ex. dɔ́\\
`to eat?' (Comment: `it's in my mouth')

\ex. dɔ̀ndɔ̀\\
`one (not for counting)'

\ex. dɔ̄ndá\\
`agree'

\ex. \textsuperscript{n}t͡ʃelí\textsuperscript{n}\\
`one' (for counting)

\ex. ɲinɛ\\
`Tooth'

\ex. ɲin(g)ba\\
`big tooth'

\ex.ɲint͡ʃɪnɛma\\
`big tooth'

\jf{I didn't get a good reading on the tones for the previous three}

\ex. gbɛ̄gbɛ̄(a)\\
`stubborn'

\ex. gbèŋgbɛ́\\
`boat'
\jf{near minimal pair with gbɛgbɛ = ‘stubborn’}

\ex. gbáŋ\\
`firmly/tightly'

\ex. gbáŋ-gbáŋ\\
`very firmly/tightly'

\ex. sàmbà\\
`basket'

\ex. sāŋgbà\\
`drum'

\ex. gbangba\\
`oven'

\ex. sɪ̀nɛ̄\\
`stone'

\ex. sɛ̀nɛ̄\\
`Rice farm'

\ex. mbà͡ʊ\\
`To heal'
\ds{I thought he also pronounced it as \textit{bã} at some point, again diphthong unclear}

\ex. bàː\\
`Goat'

\jf{I didnt’t get a good read on the tones of the following entry}

\exg. m-baŋda\\
1SG refuse\\
`I refuse'

\ex. baŋda\\
`refusal process'

\ex. fūìnɛ\\
`To beg/ask for/borrow'

\ex. fūĩ\\
`To beg/ask for/pinch' (comment: possible shortening of \textit{fūìnɛ})

\jf{I didnt’t get a good read on the tones of the following entry}

\ex. fuinda\\
`The process of begging/asking for/borrowing'

\section{CVVN (compared with CVV) -- Lex and Jan}

%\jal{many of these you have written as short vowels; what you're looking for is whether the tone patterns on long vowels differ from those on a long vowel with a final nasal.  (could be that the nasal is a tone bearing unit or not, so important to compare)}

\ex. dene \\
`child'

\ex. denenu \\
`children'


\exg. ii a wii a \\ 
water 3SG boil-a \\
`water is boiling' 

\exg. ii a wii-n\textipa{E} \\
water 3SG boil-? \\
`water has been boiled'


\ex. \textipa{da\'ON} \\
`(to) eat'

\exg. \textipa{\texttoptiebar{mb}\^a da\`On-da} \\
1SG-NPST(??) eat \\
`I eat it'

\exg. \textipa{\^a d\`a\`ON} \\
OBJ eat \\
`You (2SG) eat it'

\exg. \textipa{\'e wa \'en \texttoptiebar{tS}e} \\
2SG.NPST do work \\
`You (2SG) do work'

\ex. \textipa{k\`on\'e} \\
`tree'

\ex. \textipa{kOneamba} \\
`leaf'

\ex. \textipa{k\'on\'e\texttoptiebar{tS}\'e} \\
`this tree'

\ex. buìː \\
`medicine'

\ex. buîa \\
`run'

\ex. buìː nàmâ \\
`medicine'

\ex. fá \\
`father'

\ex. fàâ \\
`heart'

\ex. fàâ nàmâ\\
`new heart'

\ex. kùu \\
`seed'

\ex. kùumimbulɛt, kùumimbɔn,kɔnkùu \\
`seed of a tree'

\ex. kuû \\
`waist'

\ex. boː \\
`hand'

\ex. boːtʃ⁠e  \\
`this hand'

\ex. gboo \\
`book or body'

\ex. gboo nama \\
`new book'

\ex. gboo mansa \\ 
`paramount chief'

\ex. mansa nama \\ 
`new chief' (a new chief)

\ex. maa \\
`banana'

\exg. maa nama \\
banana new\\
`new banana' 

\ex. boo \\
`hand'

\exg. boo nama \\
hand new \\
`new hand'

\ex. dâ \\
`pot'

\exg. dâ ma ca\\
pot NEG ? \\
`this is not a pot'

\ex. \textipa{tiN} \\
`to jump' 

\ex. \textipa{w\`In\`e} \\ 
`deer' 

\ex. wííne \\
`hot'

\exg. \textipa{wiin-musuu-(nE)} \\
deer female- \\
`female deer' 

\exg. \textipa{wiin-kai-(nE)} \\
deer male- \\
`female deer' 


\section{CVCV, also CVCVV, CVNCV (compared with CVV) - Chun-Hung} 

\chs{CVCV vs. CVV}


\exg. b\`{a}\`{i} \\
quarrel \\
`quarrel, disagreement' 

\exg. b\'{a}\`{i} \\
cutlass \\
`cutlass, machete' 

\exg. f\`{e}\`{a} \\
two \\
`two' (FN)

\exg. f\'{e}\`{a} \\
name \\
`name of a girl' (FN)

\exg. f\'{i}\^{a} \\
bush \\
`bush'

\exg. f\`{i}\^{a} \\
wind \\
`wind'

\exg. f\'{i}\'{a}\textipa{N}\\
sweep \\
`sweep'

\exg. k\'{u}\'{u} \\
bone \\
`bone, seed' (FN)

\exg. k\'{u}n\'{u} \\
yesterday \\
`yesterday' (FN)

\exg. k\'{u}n\`{\textipa{E}} \\
grunt \\
`grunt'

\exg. k\`{u}n\^{\textipa{E}} \\
head \\
`head'

\exg. k\`{u}m\^{a} \\
on.top \\
`on top'

\exg. k\'{o}w\`{a} \\
shoe \\
`shoe'

\newpage 

\end{document}
