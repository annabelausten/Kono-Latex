\documentclass{assets/fieldnotes}

\title{Kono (Sierra Leone)}
\author{LING3020/5020}
\date{University of Pennsylvania, Spring 2025\\04/24/2025 Class Projects Week 5}

\setcounter{secnumdepth}{4} %enable \paragraph -- for subsubsubsections

\begin{document}

\maketitle

\maketitle
\tableofcontents

\section{Alex}

\paragraph*{Focus on subject of intransitives/inchoatives}

\exg.
sani-n      dɔ    ni   te-a      kunu,        bɛ     sa   ana  \\
bottle-PL   OBJ   NI   break-A   yesterday,   then   Sah   came \\%
`The bottles were breaking yesterday, then Sah came.'

\exg.
sani     tʃɛ-nu,    a-n    dɔ    ni   te-a      kunu,        bɛ     sa    ana  \\
bottle   this-PL,   3-PL   OBJ   NI   break-A   yesterday,   then   Sah   came \\%
`These bottles, they were breaking, then Sah came.' \label{47025}

\exg.
sani     tʃɛ-n     dɔ    ni   te-a...     \\
bottle   this-PL   OBJ   NI   break-A... \\%
`These bottles were breaking...' \label{81039}

\alex{\ref{47025} and \ref{81039} form a nice minimal pair: in \ref{47025} we have a hanging topic, while in \ref{81039}, it seems like the \textit{sani tʃɛn(u)} `these bottles' is indeed the grammatical subject.}

\exg.
tʃɛ-n     dɔ    ni   te-a      kunu,        bɛ     sa    ana  \\
this-PL   OBJ   NI   break-A   yesterday,   then   Sah   came \\%
`These were breaking yesterday, then Sah came.'

\exg.
fɛn     dɔ    ni   te-a,      bɛ     sa    ana  \\
thing   OBJ   NI   break-A,   then   Sah   came \\%
`What was breaking when sa came?'

\exg.
sani-an      dɔ    ni   (te)      te-a    \\
bottle-FOC   OBJ   NI   (break)   break-A \\%
A: `The bottle was breaking...' \label{67927}

\alex{Reduplication of verb is possible (for emphasis), but not obligatory, in \ref{67927}. Though, see comment after \ref{87494}.}

\exg.
sani-an      dɔ    te-a    \\
bottle-FOC   OBJ   break-A \\%
A: `The bottle broke...' (answer to `What broke?')

\exg.
sani-n-fan      dɔ    ni   (te)      te-a    \\
bottle-PL-FOC   OBJ   NI   (break)   break-A \\%
A: `The bottles were breaking...' \label{42653}

\exg.
sani     tʃɛ-n-fan     dɔ    ni   (te)      te-a    \\
bottle   this-PL-FOC   OBJ   NI   (break)   break-A \\%
A: `These bottles were breaking...'

\exg.
tʃɛ-n-fan     dɔ    ni   (te)      te-a    \\
this-PL-FOC   OBJ   NI   (break)   break-A \\%
A: `These were breaking...' \label{27512}

\exg.
sani     tʃɛ-n-fan-da-a          ni   ton     te-a,    bɛ     sa    ana  \\
bottle   this-PL-FOC-mouth-OBJ   NI   close   $v$-A,   then   Sah   came \\%
`These bottles were closing, then Sah came.' \label{34876}

\alex{Tony reported that the \textit{da} in \ref{34876} refers to `mouth'. However, I am not entirely sure this is the case. I predicted that with the verb `close', the `a' object marker should travel with/occur with the surface subject, and supposed that this `(d)a' was a reflex of that. What is surprising is the /d/ consonant showing up here, since we only really see this /d/ get inserted before an /ɔ/ vowel in this kind of context. However, upon closer inspection, it sounds like there is indeed another vowel after /da/, which I have glossed as the object marker. If this is correct, then the prediction about the object marker on the subject is born out.}

\exg.
sani     tʃɛ,    a-n-fan-da-a         ni   ton     te-a,    bɛ     sa    ana  \\
bottle   this,   3-PL-FOC-mouth-OBJ   NI   close   $v$-A,   then   Sah   came \\%
`This bottle, they were closing, then Sah came.' \label{42253}

\alex{Not sure if \ref{42253} is completely correct. I would have expected \textit{sani tʃɛnu}.}

\exg.
fɛn-an-ni      ton     te-a,    bɛ     sa    ana  \\
thing-FOC-NI   close   $v$-A,   then   Sah   came \\%
`What was closing when Sah came?' \label{87494}

\alex{Also note that the verb for `close' here appears to be distinct from previous sessions. Previously, we've heard \textit{toN} or \textit{tond-a}. Here, we have \textit{ton te-a}. For now, I am assuming that this is a complex predicate, similar to others we have seen before, where \textit{toN} is the non-verbal element `close', and \textit{te(-a)} is a light verb/verbalizer/etc. If this is correct, then it makes we wonder whether the ``reduplicated'' verbs with `break' are indeed reduplication (see \ref{42653}-\ref{27512}), or a case where the first \textit{te} is the non-verbal element `break' and the second \textit{te-a} is the same light verb we see with `close'.}

\exg.
sani-n-fan-a        ni   ton     te-a \\
bottle-PL-FOC-OBJ   NI   close   $v$-A  \\%
A: `The bottles were closing...' \label{49392}

\exg.
sani-n-fan-ni      ton     te-a \\
bottle-PL-FOC-NI   close   $v$-A  \\%
A: `The bottles were closing...' \label{25653}

\alex{Tony gave both \ref{49392} and \ref{25653} as options for the response to the question. They seem to form a nice minimal pair, where in \ref{49392}, the object marker shows up overtly on the subject, and in \ref{25653}, there is no object marker. This tracks with other observations we have made, where in transitive clauses, the /-a/ object marker is sometimes optional (depending on the verb, but I believe we have data showing that /-a/ is optional with `close'?). Tony said the difference is that \ref{25653} is the ``shorter version'', whereas \ref{49392} is the ``longer version''.} 

\exg.
sani     tʃɛ-nu,    a-n-fan-a      ni   ton     te-a \\
bottle   this-PL,   3-PL-FOC-OBJ   NI   close   $v$-A  \\%
A: `These bottles, they were closing...'

\exg.
sani     tʃɛ-n-fan-ni     ton     te-a \\
bottle   this-PL-FOC-NI   close   $v$-A  \\%
A: `These bottles were closing...'

\exg.
sani-n      dɔ.ɔ      te-a      a     waN   sina     \\
bottle-PL   OBJ.AUX   break-A   FUT   FOC   tomorrow \\%
`The bottles will break tomorrow.'

\exg.
sani     tʃɛ-n     dɔ.ɔ      te-a      a     waN   sina     \\
bottle   this-PL   OBJ.AUX   break-A   FUT   FOC   tomorrow \\%
`These bottles will break tomorrow.'

\exg.
tʃɛ-n     dɔ.ɔ      te-a      a     waN   sina     \\
this-PL   OBJ.AUX   break-A   FUT   FOC   tomorrow \\%
`These will break tomorrow.'

\exg.
fen     dɔ.ɔ      te-a      a     sina     \\
thing   OBJ.AUX   break-A   FUT   tomorrow \\%
`What will break tomorrow?'

\exg.
sani     ɔɔ        te-a      waN   sina     \\
bottle   OBJ.AUX   break-A   FOC   tomorrow \\%
A: `The bottle will break tomorrow.' \label{96773}

\alex{I don't really hear the same vowel (glossed as FUT in previous future tense example), in \ref{96773}.}

\exg. 
sani-an      dɔɔ       te-a      waN   sina     \\
bottle-FOC   OBJ.AUX   break-A   FOC   tomorrow \\%
A: `The bottle will break tomorrow.'

\exg.
sani-n-fan      dɔɔ       te-a      waN   sina     \\
bottle-PL-FOC   OBJ.AUX   break-A   FOC   tomorrow \\%
A: `The bottles will break tomorrow.'

\exg.
sani-n-fan      dɔ    te-a      (\#sina)     \\
bottle-PL-FOC   OBJ   break-A   (\#tomorrow) \\%
A: `The bottles broke (\#tomorrow).' \label{69764}

\alex{In the past tense example in \ref{69764}, I am still not sure whether and where the AUX/tense marker /-a/ occurs.}

\exg.
sani     tʃɛ-n-fan     dɔɔ       te-a      waN   sina     \\
bottle   this-PL-FOC   OBJ.AUX   break-A   FOC   tomorrow \\%
A: `These bottles will break tomorrow'

\exg.
tʃɛ-n-fan     dɔɔ       te-a      waN   sina     \\
this-PL-FOC   OBJ.AUX   break-A   FOC   tomorrow \\%
A: `These will break tomorrow'

\exg.
fɛnɛ    tʃɛ-n-fan     dɔɔ       te-a      waN   sina     \\
thing   this-PL-FOC   OBJ.AUX   break-A   FOC   tomorrow \\%
A: `These (things) will break tomorrow'


\paragraph*{Focus on object of transitives:}

% PAST PROG

\exg.
bondu   ni   taa-n         dɔ    te-a      kunu,        bɛ     sa    ana  \\
Bondu   NI   calabash-PL   OBJ   break-A   yesterday,   then   Sah   came \\%
`Bondu was breaking the calabashes, then Sah came.'

\exg.
bondu   ni   sani-n-fan      dɔ    te-a      kunu,        bɛ     sa    ana  \\
Bondu   NI   bottle-PL-FOC   OBJ   break-A   yesterday,   then   Sah   came \\%
`Bondu was breaking the bottles, then Sah came.'

\exg.
bondu   ni   sani-n      dɔ    te-a      kunu,        bɛ     sa    ana  \\
Bondu   NI   bottle-PL   OBJ   break-A   yesterday,   then   Sah   came \\%
`Bondu was breaking the bottles, then Sah came.'

\exg.
bondu   ni   sani-an      dɔ    te-a      kunu,        bɛ     sa    ana  \\
Bondu   NI   bottle-FOC   OBJ   break-A   yesterday,   then   Sah   came \\%
`Bondu was breaking the bottle, then Sah came.'

\exg.
bondu   ni   sani     ɔ     te-a      kunu,        bɛ     sa    ana  \\
Bondu   NI   bottle   OBJ   break-A   yesterday,   then   Sah   came \\%
`Bondu was breaking the bottle, then Sah came.'

\exg.
bondu   ni   sani     tʃɛ-n-fan     dɔ    te-a      kunu,        bɛ     sa    ana  \\
Bondu   NI   bottle   this-PL-FOC   OBJ   break-A   yesterday,   then   Sah   came \\%
`Bondu was breaking these bottles, then Sah came.'

\exg.
bondu   ni   sani     tʃɛ-n     dɔ    te-a      kunu,        bɛ     sa    ana  \\
Bondu   NI   bottle   this-PL   OBJ   break-A   yesterday,   then   Sah   came \\%
`Bondu was breaking these bottles, then Sah came.'

\exg.
bondu   a   sani     ɔ     te    \\
Bondu   3SG.PST   bottle   OBJ   break \\%
`Bondu broke the bottle.'

\exg.
bondu   a   sani     tʃɔ    ɔ     te    \\
Bondu   3SG.PST   bottle   this   OBJ   break \\%
`Bondu broke this bottle.'

\exg.
bondu   a   fɛn     dɔ    te    \\
Bondu   3SG.PST   thing   OBJ   break \\%
`What did Bondu break?'

\exg.
a-a      sani-an      dɔ    te    \\
3SG.PST-A   bottle-FOC   OBJ   break \\%
A: `She broke the bottle.'

\exg.
bondu   a   sani     tʃɔ    ɔ     te    \\
Bondu   3SG.PST   bottle   this   OBJ   break \\%
A: `Bondu broke this bottle.'

\exg.
bondu   a   sani     tʃa-an     dɔ    te    \\
Bondu   3SG.PST   bottle   this-FOC   OBJ   break \\%
A: `Bondu broke this bottle.'

\exg.
bondu   a   sani     tʃɛ-n-fan     dɔ    te    \\
Bondu   3SG.PST   bottle   this-PL-FOC   OBJ   break \\%
A: `Bondu broke these bottles.'

\exg.
bondu   a   sani     tʃɛ    da      a     toN   \\
Bondu   3SG.PST   bottle   this   mouth   OBJ   close \\%
`Bondu closed this bottle.'

\exg.
bondu   a   sani     tʃa    a     toN   \\
Bondu   3SG.PST   bottle   this   OBJ   close \\%
`Bondu closed this bottle.'

\alex{In careful/slow speech, Tony produced \textit{sani tʃɛ a}, which supports the idea that there is indeed an object marker that affects the vowel quality of the preceding segment (in the case that preceding element ends in a vowel).}

\exg.
bondu   a   sani     a     toN   \\
Bondu   3SG.PST   bottle   OBJ   close \\%
`Bondu closed a/the bottle.'

\exg.
bondu   a   sani     tʃɛ-n     a     toN   \\
Bondu   3SG.PST   bottle   this-PL   OBJ   close \\%
`Bondu closed these bottles.'


\section{Jan}

\jmt{From Last Week}:

\ex. bondu a ko ja kune \\
           3SG.PST
`Bondu's older brother woke you up yesterday.'

\ex. bondu ko ea kune wã sina \\
`Bondu's older brother will wake you up tomorrow.'

\exg. bondu ko íí \\
bondu brother.older brain \\
`Bondu's older brother's brain'

\exg. bondu ko íí wá \\
bondu brother.older brain big \\
`Bondu's older brother's brain is big.'

\exg. bondu ko íí a waj wã tʃe kunu \\
bondu brother.older brain AUX.3SG work FOC DEM yesterday \\
`Bondu's older brother's brain worked hard yesterday.'

\subsection{[$\varnothing$ɔ] + [$\varnothing$í]}

\ex. meme í te kunu \\
`The mirror cut you yesterday.'

\ex. a i te kunu \\
3SG.PST 2SG cut  yesterday
`It cut you yesterday.'

\exg. meme i te a sandʒe \\
mirror 2SG cut 3SG PROG \\
`The mirror is cutting you.'

\exg. memem-bi te a saandʒe \\
mirror.PL-2SG cut 3SG  \\
`The mirrors are cutting you.'

\ex. a-m-b-i te a saandʒe \\
3-PL-NPST-2SG cut OBJ PROG \\
`They are cutting you.'

\subsection{N + [$\varnothing$í] }

\jmt{Talking about Bondu and Fea.}

\ex. ambe a \\
`They fell.'

\exg. a-m-be a i ma \\
3-PL-NPST   ?  2SG  ?
`They fell on you.'

\exg. i beja a-n-u ma \\
2SG fall 3-PL-? COP \\
`You fell on them.'

\exg. a-n-dʒii wa nu \\
3-PL-brain big PL
`Their brains are big.'

\ex. saa kune \\
`Saa, wake up.'

\exg. saa ja kune\\
(Needs glossing)\\
`Saa woke you up'

\exg. Saa, Bondu ni fea fea ana kune\\
(needs glossing)\\
`Saa, wake up Bondu and Fea.'

\exg. Saa, Bondu ni fea ana kune\\
(needs glossing)\\
`Saa, wake up Bondu and Fea.'

\exg. Saa, ana kune\\
(needs glossing)\\
`Saa, wake them up.'

\exg. i gbo\\
(needs glossing)
` Your hand'

\exg. na tʃena toŋ\\
(Needs glossing)
`I closed the door'

\exg. Na tʃena da toŋ\\
(Needs glossing)
`I closed the door'

\exg. na tʃena toŋ, i bo ma
(Needs glossing)
`I closed the door on your hand.'

\exg. na kanganena toŋ, i bo ma
(Needs glossing)
`I closed the door on your hand.'


\subsection{[$\varnothing$í] + [$\varnothing$á]}

\ex. `Saa pranked Bondu yesterday.'
\jf{Needs t.}

\ex. `Saa was pranked by Bondu yesterday.'
\jf{Needs t}

\jmt{Bondu pranked Saa yesterday.}

\exg. e bondu jaa wã sina \\
2SG.NPST bondu prank FUT tomorrow \\
`You will prank Bondu tomorrow.'
\jf{Needs t}

\ex. j-a jaa wã sina \\
2PL.FUT-3SG.OBJ prank FUT tomorrow \\
`You will prank her tomorrow.'
\jf{needs t}


\ex. `She will prank you tomorrow.'
\jf{Needs t}



\subsection{[$\varnothing$í] + [$\varnothing$ó]}

\ex. `You broke the mirror yesterday.'
\jf{needs t}

\ex. `The mirror was broken by you yesterday.'
\jf{needs t}

\ex. `You broke it yesterday.'
\jf{needs t}

\ex. `It was broken by you yesterday.'
\jf{needs t}

\ex. `You will break the mirror tomorrow.'
\jf{needs t}

\ex. `You will break it tomorrow.'
\jf{needs t}

\exg. e o te a wã sina \\
2SG.NPST 3SG.OBJ cut 3SG FUT tomorrow \\
`You will break it tomorrow.'

\ex. j-o te a wã sina \\
2SG-OBJ cut 3SG FUT tomorrow
`It will be broken by you tomorrow.'

\subsection{Miscellaneous}


\exg. Saa o te, Bondu a gbo kunu\\
(Needs glossing)\\
`Saa said Bondu cursed me yesterday.'
\jmt{Expected: saa o te bondu n-gbô kunu}

\exg. Saa fɔ̀-n-d͡ʒè, Bondu a gbo kunu\\
(Needs glossing)\\
`Saa said Bondu cursed me yesterday.'

\exg. j-o te bondu a gboo kunu \\
2SG-OBJ cut bondu 3SG.PST curse yesterday \\
`You said Bondu cursed me yesterday.'

\exg. j-a fɔ̀-n-d͡ʒè, bondu a gboo kunu \\
2SG-PST cut bondu 3SG.PST curse yesterday \\
`You said Bondu cursed me yesterday.'

\exg. Saa o te Bondu a i gbo kunu\\
(Needs glossing)
` Saa said Bondu cursed her yesterday'

\exg. Saa a fɔ̀-n-d͡ʒè Bondu a i gbo kunu\\
(Needs glossing)
` Saa said Bondu cursed her yesterday'

\exg. Saa o te Bondu, a i gbo kunu\\
(Needs glossing)
` Saa said she cursed her yesterday'

\exg. Saa o te, eja gbo kunu\\
(Needs glossing)
` Saa said you cursed her yesterday'


\jmt{They are talking about Saa having taught you to dance.}

\exg. Saa a kaŋ \\
(Needs glossing)\\
`Saa taught me'

\exg. Sai kaŋ\\
(Needs glossing)
`Saa taught you'

\exg. a i kaŋ\\
(Needs glossing)\\
`He taught you.'

\exg. ja kaŋ\\
(Needs glossing)\\
`You taught him'

\exg. anɛ kaŋ\\
(needs glossing)
`They taught me'


\exg. a-n-a i kã \\
3-PL-3SG.PST 2SG teach \\
`They taught you.'

\ex. he taught you
\jf{Needs transcription}


\jmt{Ask if it's possible to say an + í. [andzi]}

\exg. a-n-i kã \\
3-PL-2SG teach \\
`Let them teach you.'

\ex. `You taught him.'
\jf{Needs transcription}

\ex. they taught you
\jf{Needs transcription}

\ex. `You taught them.'
\jf{Needs transcription}


\section{Lex}

\ex. gbò\\
`skin'

\ex. ŋm-gbò \\
1SG-skin\\
`my skin'

\ex. sàwáŋ\\
` to drag / pull a thing'

\ex. gbòndò\\
`to roll on ground/ drag a person'

\exg. à tʃìnɛ́ sáwá \jal{nasal final vowel?}\\ 
 3SG feet drag \\
`the feet dragged'

\ex. tʃaŋ\\
` to curse'

\ex. gbo\\
` to curse'

\exg. mòkámá tʃ-á ŋm-gbò \\
 man  this-3SG.PST 1SG-curse\\
` This man cursed me'

\exg. mòkámá  ŋm-gbò \\
man  1SG-curse \\
` the man cursed me'

\exg. à sínɛ́ ŋm-gbò \jal{I'm guessing homorganic}\\
3SG.PST rock 1SG-curse\\
`The man cursed the rock'

\exg. à féŋ gbò\\
3SG What curse\\
` What did he curse'


\ex. twígbɔfwɛ́\\
`frog on land'

\ex. kpèkpè\\
`frog in water'

\ex. baijama\\
` to cuss'

\ex. maínɛ́\\
`to cuss/say a bad word'

\exg. à máí\\
\\
`curse me'

\ex. táwátʃɛ̀\\
` to cook'

\ex. kpò\\
`to cook slowly/softly like rice'

\exg. à kwɛ̀ ɱ-kpò \jal{can't have been /n/}\\
3SG.PST rice  1SG-overcook\\
`The rice was overcooked'

\ex. daŋá
` to believe'

\exg. n-dán dí mwá í dí-tʃá wá sénà\\
1SG-believe  ?  ?  2SG  cry FOC  tomorrow\\
`I believe you will cry tomorrow'

\ex. tʃiɔbai
` to have a conversation'

\ex. kwijɛ\\
` language'

\ex. tʃiuɛ\\
`talking'

\ex. bai\\
`kiss'

\ex. bai\\
`word'

\exg. n-tù-mù mání tʃiɔbɔ̀\\
1SG- ??            conversation\\
` I want us to have a conversation'

\exg. tʃínùɛ́ɛ́-n dʒá\\
sleep-1SG eye\\
`sleepy'

\exg. tʃìnɛ́-tʃɛ́ swɛ́ nà\jf{ Sometimes Anthony pronounced tʃinuɛ}\\
sleep-this ?    1SG.    \\
`I want to sleep'

\exg. sén tʃénámà\\
stone big\\
` big stone'

\exg. sén dʒénámà \jf{Anthony said it was acceptable, sometimes happened when spoken too quickly}\\
 stone big\\
` big stone'

\ex. mù\\
` to run'

\ex. tʃɛ\\
`pour liquids'

\exg. íí tʃɛ̀\\
water pour\\
`pour water'

\exg. tù tʃɛ̀\\
oil pour\\
`pour oil'

\exg. Kai à féŋ tʃɛ̀\\
Kai 3SG what pour\\
` What did Kai pour?'

\jf{ Asked if Kai à féŋ dʒɛ̀ was okay, said not really because easily mistaken as Kai's item}

\exg. Kai à tù tʃɛ̀\\
Kai 3SG.PST oil pour\\
' Kai poured oil'

\jf{Anthony fully rejected attempted dʒɛ̀ insertion here}

\ex. jéé\\
` to see'

\exg. Kai à féŋ dʒéé\\
Kai 3SG what see\\
` What did Kai see?'

\exg. Kai à kùndé n-dʒee\\
Kai 3SG.PST bird .foc??-see\\
` Kai saw a bird'

\exg. Kai à tɛ́ n-dʒee\\
Kai 3SG.PST chicken foc??-see\\
` Kai saw a chicken'

\end{document}